\documentclass[a4paper]{book}

\usepackage{geometry}
% make full use of A4 papers
\geometry{margin=1.5cm, vmargin={0pt,1cm}}
\setlength{\topmargin}{-1cm}
\setlength{\paperheight}{29.7cm}
\setlength{\textheight}{25.1cm}

% auto adjust the marginals
\usepackage{marginfix}

\usepackage{amsmath}
\usepackage{amssymb}
\usepackage{amsthm}
\usepackage{amsfonts}
%\usepackage{CJKutf8}   % for Chinese characters
\usepackage{ctex}
\usepackage{enumerate}
\usepackage{graphicx}  % for figures
\usepackage{layout}
\usepackage{multicol}  % multiple columns to reduce number of pages
\usepackage{mathrsfs}  
\usepackage{fancyhdr}
\usepackage{subfigure}
\usepackage{tcolorbox}
\usepackage{tikz-cd}
\usepackage{listings}
\usepackage{xcolor} %代码高亮
\usepackage{braket}
\usepackage{algorithm} 
\usepackage{algorithmicx}  
\usepackage{algpseudocode}  
\usepackage{amsmath}  
\usepackage{bm}

\floatname{algorithm}{算法}  
\renewcommand{\algorithmicrequire}{\textbf{输入:}}  
\renewcommand{\algorithmicensure}{\textbf{输出:}}  
\renewcommand{\algorithmicrequire}{\textbf{Input : }}
\renewcommand{\algorithmicrequire}{\textbf{Precondition : }}
\renewcommand{\algorithmicensure}{\textbf{Output : }}
\renewcommand{\algorithmicensure}{\textbf{Postcondition : }}
%------------------
% common commands %
%------------------
% differentiation
\newcommand{\gen}[1]{\left\langle #1 \right\rangle}
\newcommand{\dif}{\mathrm{d}}
\newcommand{\difPx}[1]{\frac{\partial #1}{\partial x}}
\newcommand{\difPy}[1]{\frac{\partial #1}{\partial y}}
\newcommand{\Dim}{\mathrm{D}}
\newcommand{\avg}[1]{\left\langle #1 \right\rangle}
\newcommand{\sgn}{\mathrm{sgn}}
\newcommand{\Span}{\mathrm{span}}
\newcommand{\dom}{\mathrm{dom}}
\newcommand{\Arity}{\mathrm{arity}}
\newcommand{\Int}{\mathrm{Int}}
\newcommand{\Ext}{\mathrm{Ext}}
\newcommand{\Cl}{\mathrm{Cl}}
\newcommand{\Fr}{\mathrm{Fr}}
% group is generated by
\newcommand{\grb}[1]{\left\langle #1 \right\rangle}
% rank
\newcommand{\rank}{\mathrm{rank}}
\newcommand{\Iden}{\mathrm{Id}}

% this environment is for solutions of examples and exercises
\newenvironment{solution}%
{\noindent\textbf{Solution.}}%
{\qedhere}
% Define a Solution environment like proof
\makeatletter
\newenvironment{sol}[1][\solname]{\par
  \pushQED{\qed}
  \normalfont \topsep6\p@\@plus6\p@\relax
  \trivlist
  \item[\hskip\labelsep
        \itshape
    #1\@addpunct{.}]\ignorespaces
}{\popQED\endtrivlist\@endpefalse}
\providecommand{\solname}{Solution}
\makeatother
% the following command is for disabling environments
%  so that their contents do not show up in the pdf.
\makeatletter
\newcommand{\voidenvironment}[1]{%
\expandafter\providecommand\csname env@#1@save@env\endcsname{}%
\expandafter\providecommand\csname env@#1@process\endcsname{}%
\@ifundefined{#1}{}{\RenewEnviron{#1}{}}%
}
\makeatother

%---------------------------------------------
% commands specifically for complex analysis %
%---------------------------------------------
% complex conjugate
\newcommand{\ccg}[1]{\overline{#1}}
% the imaginary unit
\newcommand{\ii}{\mathbf{i}}
%\newcommand{\ii}{\boldsymbol{i}}
% the real part
\newcommand{\Rez}{\mathrm{Re}\,}
% the imaginary part
\newcommand{\Imz}{\mathrm{Im}\,}
% punctured complex plane
\newcommand{\pcp}{\mathbb{C}^{\bullet}}
% the principle branch of the logarithm
\newcommand{\Log}{\mathrm{Log}}
% the principle value of a nonzero complex number
\newcommand{\Arg}{\mathrm{Arg}}
\newcommand{\Null}{\mathrm{null}}
\newcommand{\Range}{\mathrm{range}}
\newcommand{\Ker}{\mathrm{ker}}
\newcommand{\Iso}{\mathrm{Iso}}
\newcommand{\Aut}{\mathrm{Aut}}
\newcommand{\ord}{\mathrm{ord}}
\newcommand{\Res}{\mathrm{Res}}
%\newcommand{\GL2R}{\mathrm{GL}(2,\mathbb{R})}
\newcommand{\GL}{\mathrm{GL}}
\newcommand{\SL}{\mathrm{SL}}
\newcommand{\Dist}[2]{\left|{#1}-{#2}\right|}

\newcommand\tbbint{{-\mkern -16mu\int}}
\newcommand\tbint{{\mathchar '26\mkern -14mu\int}}
\newcommand\dbbint{{-\mkern -19mu\int}}
\newcommand\dbint{{\mathchar '26\mkern -18mu\int}}
\newcommand\bint{
{\mathchoice{\dbint}{\tbint}{\tbint}{\tbint}}
}
\newcommand\bbint{
{\mathchoice{\dbbint}{\tbbint}{\tbbint}{\tbbint}}
}





%----------------------------------------
% theorem and theorem-like environments %
%----------------------------------------
% \numberwithin{equation}{chapter}
% \theoremstyle{definition}

\newtheorem{thm}{Theorem}[chapter]
\newtheorem{axm}[thm]{Axiom}
\newtheorem{alg}[thm]{Algorithm}
\newtheorem{asm}[thm]{Assumption}
\newtheorem{defn}[thm]{Definition}
\newtheorem{prop}[thm]{Proposition}
\newtheorem{rul}[thm]{Rule}
\newtheorem{coro}[thm]{Corollary}
\newtheorem{lem}[thm]{Lemma}
\newtheorem{exm}{Example}[chapter]
\newtheorem{rem}{Remark}[chapter]
\newtheorem{exc}[exm]{Exercise}
\newtheorem{frm}[thm]{Formula}
\newtheorem{ntn}{Notation}
\newtheorem{pro}{Problem}

% for complying with the convention in the textbook
\newtheorem{rmk}[thm]{Remark}


%----------------------
% the end of preamble %
%----------------------

\begin{document}

%\tableofcontents
%\clearpage

\pagestyle{fancy}
%\lhead{Qinghai Zhang}
%\chead{Notes on Algebraic Topology}
%\rhead{Fall 2018}


\setcounter{chapter}{3}
\pagenumbering{arabic}
% \setcounter{page}{0}


% each chapter is factored into a separate file.

\section{PDEhw4 12235005 谭焱}
\paragraph*{1. }
Prove the following Proposition. 
\begin{prop}
    A linear form $u$ on $\mathcal{D}(\Omega)$ is continuous 
    ($u(\phi_j) \rightarrow 0$ for every sequence $\phi_j \in 
    \mathcal{D}(\Omega)$ converging to 0) iff it verifies the
    following property: for any compact set $K \subset \Omega$ 
    there exists an integer $k$ and a constant $C = C_{K, k}$
    such that 
    \[| \langle u, \phi\rangle | \leq Cp_{K, k}(\phi), \forall \phi \in C_c^\infty(K). \] 
\end{prop}
\begin{sol}
    \begin{itemize}
        \item Sufficiency: Let $\phi_j \in \mathcal{D}(\Omega)$ be 
        a sequence converging to 0, then the definition of the 
        topology of $\mathcal{C}_c^\infty(\Omega)$ yields there is a compact set $K \subset \Omega$,
            supp $\phi_j \subset K$, for all $j \geq 1$.
           and for any $k$, 
            \[ p_{K, k} := \sup_{|\alpha| \leq k} \sup_{x \in K} 
            |\partial^\alpha \phi_j(x)| \rightarrow 0 \text{ as } j \rightarrow \infty.\]
        Combining with assumption 
        \[|u(\phi_j)| = | \langle u, \phi_j \rangle | 
        \leq Cp_{K, k}(\phi_j) \rightarrow 0 \text{ as } j \rightarrow \infty.\]

        \item Necessity: Assuming $u$ is continuous and
        \[\exists K \subset \Omega, \forall k > 0,
        \exists \phi_j \in \mathcal{C}_c^\infty(K) s.t. 
        u(\phi_k) > Cp_{K, k}.\]
        Choosing $C = j^2$ and $ \Phi_j = \frac{\phi_j}{j p_{K, k}(\phi_j)}
        \in \mathcal{C}_c^\infty(K)$. Then 
        \begin{align*}
            p_{K, k}(\Phi_j) = \frac{p_{K, k}(\phi_j)}{j p_{K, k}(\phi_j)} = \frac{1}{j} \rightarrow 0, \text{ as } j \rightarrow \infty \\
            u(\Phi_j) = \frac{u(\phi_j)}{j p_{K, k}(\phi_j)} \geq 
            \frac{C p_{K, k}(\phi_j)}{jp_{k, k}(\phi_j)} = j
        \end{align*}
        Which is conflict with $u$ is continuous.
    \end{itemize}
\end{sol}

\paragraph*{2. }
Prove the following lemma 
\begin{lem}
    Let $g \in L^1(\mathbb{R})$ with  $\int_{\mathbb{R}} gdx = 1$, 
    then $g_\epsilon(x) = \epsilon^{-1} g(\epsilon^{-1} x)$ 
    converges to $\delta$ as $\epsilon \rightarrow 0+$, in $\mathcal{D}'(\mathbb{R})$. 
\end{lem}
\begin{sol}
    Since $\int_{\mathbb{R}} g(x) dx = \epsilon^{-1} 
    \int_{\mathbb{R}} g(\epsilon^{-1} (\epsilon x)) d(\epsilon x)
    = \int_{\mathbb{R}}g_\epsilon(x) dx$. By definition of 
    converges, considering
    \begin{align*}
        \langle g_\epsilon - \delta, \phi \rangle 
        &= \int_{\mathbb{R}} g_\epsilon(x)\phi(x) dx -
        \int_{\mathbb{R}} g(x)\phi(0) dx  = 
        \int_{\mathbb{R}} g(x)(\phi(\epsilon x) - \phi(0)) dx \\
        &\leq \left(\int_{\mathbb{R}}|g(x)|^2 dx \int_{\mathbb{R}} |\phi(\epsilon x) - \phi(0)|^2 dx\right)^{1/2} \\
        &= C \left(\int_{\mathbb{R}} |\phi(\epsilon x) - \phi(0) | \right)^{1/2}
        \rightarrow 0 \qquad \text{as } \epsilon \rightarrow 0+.
    \end{align*}
    Therefore, $g_\epsilon(x)$ converges to $\delta$.
\end{sol}

\paragraph*{3. }
As $\delta \rightarrow 0 +$,
\[ K_{\delta}(\xi) = |\xi|^{-2} e^{-\delta|\xi|} \rightarrow |\xi|^{-2}, \text{ in }
S'(\mathbb{R}^3).\]
\begin{sol}
    By definition of converges, considering 
    \begin{align*}
        \langle K_\delta(\xi) - |\xi|^{-2}, \phi(\xi)\rangle 
        &= \int_{\mathbb{R}^3} (e^{-\delta |\xi|} - 1) |\xi|^{-2} \phi(\xi) d\xi
        = \int_0^{2\pi} \int_{0}^{2\pi} \int_0^\infty (-r\delta + \mathcal{O}(\delta^2))\phi d\theta_1 d\theta_2 dr
        \\& \rightarrow 0 \qquad \text{as } \delta \rightarrow 0+.
    \end{align*}
\end{sol}
    
\end{document}
