\documentclass[a4paper]{book}

\usepackage{geometry}
% make full use of A4 papers
\geometry{margin=1.5cm, vmargin={0pt,1cm}}
\setlength{\topmargin}{-1cm}
\setlength{\paperheight}{29.7cm}
\setlength{\textheight}{25.1cm}

% auto adjust the marginals
\usepackage{marginfix}

\usepackage{amsfonts}
\usepackage{amsmath}
\usepackage{amssymb}
\usepackage{amsthm}
%\usepackage{CJKutf8}   % for Chinese characters
\usepackage{ctex}
\usepackage{enumerate}
\usepackage{graphicx}  % for figures
\usepackage{layout}
\usepackage{multicol}  % multiple columns to reduce number of pages
\usepackage{mathrsfs}  
\usepackage{fancyhdr}
\usepackage{subfigure}
\usepackage{tcolorbox}
\usepackage{tikz-cd}
\usepackage{listings}
\usepackage{xcolor} %代码高亮
\usepackage{braket}
\usepackage{algorithm} 
\usepackage{algorithmicx}  
\usepackage{algpseudocode}  
\usepackage{amsmath}
\usepackage{lmodern}  

\floatname{algorithm}{算法}  
\renewcommand{\algorithmicrequire}{\textbf{输入:}}  
\renewcommand{\algorithmicensure}{\textbf{输出:}}  
\renewcommand{\algorithmicrequire}{\textbf{Input : }}
\renewcommand{\algorithmicrequire}{\textbf{Precondition : }}
\renewcommand{\algorithmicensure}{\textbf{Output : }}
\renewcommand{\algorithmicensure}{\textbf{Postcondition : }}
%------------------
% common commands %
%------------------
% differentiation
\newcommand{\gen}[1]{\left\langle #1 \right\rangle}
\newcommand{\dif}{\mathrm{d}}
\newcommand{\difPx}[1]{\frac{\partial #1}{\partial x}}
\newcommand{\difPy}[1]{\frac{\partial #1}{\partial y}}
\newcommand{\Dim}{\mathrm{D}}
\newcommand{\avg}[1]{\left\langle #1 \right\rangle}
\newcommand{\sgn}{\mathrm{sgn}}
\newcommand{\Span}{\mathrm{span}}
\newcommand{\dom}{\mathrm{dom}}
\newcommand{\Arity}{\mathrm{arity}}
\newcommand{\Int}{\mathrm{Int}}
\newcommand{\Ext}{\mathrm{Ext}}
\newcommand{\Cl}{\mathrm{Cl}}
\newcommand{\Fr}{\mathrm{Fr}}
% group is generated by
\newcommand{\grb}[1]{\left\langle #1 \right\rangle}
% rank
\newcommand{\rank}{\mathrm{rank}}
\newcommand{\Iden}{\mathrm{Id}}

% this environment is for solutions of examples and exercises
\newenvironment{solution}%
{\noindent\textbf{Solution.}}%
{\qedhere}
% the following command is for disabling environments
%  so that their contents do not show up in the pdf.
\makeatletter
\newcommand{\voidenvironment}[1]{%
\expandafter\providecommand\csname env@#1@save@env\endcsname{}%
\expandafter\providecommand\csname env@#1@process\endcsname{}%
\@ifundefined{#1}{}{\RenewEnviron{#1}{}}%
}
\makeatother

%---------------------------------------------
% commands specifically for complex analysis %
%---------------------------------------------
% complex conjugate
\newcommand{\ccg}[1]{\overline{#1}}
% the imaginary unit
\newcommand{\ii}{\mathbf{i}}
%\newcommand{\ii}{\boldsymbol{i}}
% the real part
\newcommand{\Rez}{\mathrm{Re}\,}
% the imaginary part
\newcommand{\Imz}{\mathrm{Im}\,}
% punctured complex plane
\newcommand{\pcp}{\mathbb{C}^{\bullet}}
% the principle branch of the logarithm
\newcommand{\Log}{\mathrm{Log}}
% the principle value of a nonzero complex number
\newcommand{\Arg}{\mathrm{Arg}}
\newcommand{\Null}{\mathrm{null}}
\newcommand{\Range}{\mathrm{range}}
\newcommand{\Ker}{\mathrm{ker}}
\newcommand{\Iso}{\mathrm{Iso}}
\newcommand{\Aut}{\mathrm{Aut}}
\newcommand{\ord}{\mathrm{ord}}
\newcommand{\Res}{\mathrm{Res}}
%\newcommand{\GL2R}{\mathrm{GL}(2,\mathbb{R})}
\newcommand{\GL}{\mathrm{GL}}
\newcommand{\SL}{\mathrm{SL}}
\newcommand{\Dist}[2]{\left|{#1}-{#2}\right|}

\newcommand\tbbint{{-\mkern -16mu\int}}
\newcommand\tbint{{\mathchar '26\mkern -14mu\int}}
\newcommand\dbbint{{-\mkern -19mu\int}}
\newcommand\dbint{{\mathchar '26\mkern -18mu\int}}
\newcommand\bint{
{\mathchoice{\dbint}{\tbint}{\tbint}{\tbint}}
}
\newcommand\bbint{
{\mathchoice{\dbbint}{\tbbint}{\tbbint}{\tbbint}}
}





%----------------------------------------
% theorem and theorem-like environments %
%----------------------------------------
\numberwithin{equation}{chapter}
\theoremstyle{definition}

\newtheorem{thm}{Theorem}[chapter]
\newtheorem{axm}[thm]{Axiom}
\newtheorem{alg}[thm]{Algorithm}
\newtheorem{asm}[thm]{Assumption}
\newtheorem{defn}[thm]{Definition}
\newtheorem{prop}[thm]{Proposition}
\newtheorem{rul}[thm]{Rule}
\newtheorem{coro}[thm]{Corollary}
\newtheorem{lem}[thm]{Lemma}
\newtheorem{exm}{Example}[chapter]
\newtheorem{rem}{Remark}[chapter]
\newtheorem{exc}[exm]{Exercise}
\newtheorem{frm}[thm]{Formula}
\newtheorem{ntn}{Notation}

% for complying with the convention in the textbook
\newtheorem{rmk}[thm]{Remark}


%----------------------
% the end of preamble %
%----------------------

\begin{document}
\pagestyle{plain}
\pagenumbering{roman}

%\tableofcontents
%\clearpage

\pagestyle{fancy}
%\fancyhead{}
\chapter{PDEhw2 12235005 谭焱}

\paragraph*{1. }
Specify condition on $f$ , and solve the initial boundary value problem of heat
equations
\begin{equation}
    \left\{
    \begin{aligned}
         & u_t = u_{xx} + t \cos x, \qquad x \in [0, 1], t > 0. \\
         & u_x(0, t) = u_x(1, t) = 0,                           \\
         & u(x, 0) = f(x)
    \end{aligned}
    \right.
\end{equation}
by the method of separation variables.

\begin{solution}
    Define $u_1, u_2$ are solution for follow equations separately,
    \begin{equation*}
        \left\{ \begin{aligned}
             & u_t = u_{xx} \qquad \qquad \qquad & u_t = u_{xx} + t \cos x \\
             & u(x, 0) = f(x)                    & u(x, 0) = 0             \\
             & u_x(0, t) = u_x(1, t) = 0                                   \\
        \end{aligned}
        \right.
    \end{equation*}.
    So $u = u_1 +u_2$ solve origin equations, assume $u_1 = F(x)G(t)$ get
    \begin{equation*}
        \begin{array}{l}
            \frac{G'}{G} = \frac{F''}{F}
        \end{array}
        \Rightarrow
        \left\{ \begin{array}{l}
            G(t) = e^{-m^2t} G(0)                                             \\
            F(x) = (A_m \cos(2m \pi x) + B_m \sin(2m \pi x)) \\
        \end{array}
        \right.
    \end{equation*}.
    By boundary condition,
    \begin{equation*}
        \left\{
        \begin{array}{l}
            F(x)G(0) = f(x) \\
            F'(0)G(t) = F'(1)G(t) = 0
        \end{array}
        \right.
        \Rightarrow
        \left\{
        \begin{array}{l}
            G(0) = 1                               \\
            A_m = 2\int_0^1 f(x) \cos( m \pi x) dx \\
            B_m = 0                                \\
        \end{array}
        \right.
    \end{equation*}.
    Hence induce $u_1 = A_0/2 + \sum_m A_m \cos(m\pi x)e^{-(m\pi)^2 t}$. Assume
    $w$ solve
    \begin{equation*}
        \left\{
        \begin{array}{l}
            u_t = u_{xx}              \\
            u(x, \tau) = \tau \cos x        \\
            u_x(0, t) = u_x(1, t) = 0 \\
        \end{array}
        \right.
    \end{equation*}.
    Similarly, $w(x, t, \tau) = A'_0/2 + \sum_m  A'_m \cos(m\pi x)e^{-(m\pi)^2 t}, $ there is
    \[ A'_m = 2\int_0^1 \tau e^{(m\pi)^2 \tau} \cos x \cos(m \pi x) dx\].
    According to Duhamel's principle, we know 
    \[ u = u_1 + u_2 = u_1 + \int_0^t w(x, t, \tau) d\tau  \]
\end{solution}

\paragraph*{2. }
Solve the boundary value problem of the Laplace equation in disc:
\begin{equation}
    \left\{
    \begin{aligned}
         & \Delta u = u_{xx} + u_{yy} = 0, \qquad x^2 + y^2 < R^2, \\
         & u(R \cos \theta, R \sin \theta) = f(\theta)
    \end{aligned}
    \right.
\end{equation}
by the method of separation of variables (in polar coordinates), in the form
\[ u(r, \theta) = \int_0^{2x} G(r, \theta, R, \phi) f(\phi) d\phi .\]

\begin{solution}
    Rewrite Laplace equation as polar coordinates 
    \[\Delta u = u_{rr} +u_{r}/r + u_{\theta \theta} / r^2 = 0 \].
    Assume $u = F(r)G(\theta)$ and replace $u$ in previous equation get 
    $(r^2 F'' + r F') / F = - G''/G$.
    By $G$ is period in $2\pi$ since polar coordinates induce 
    \begin{align*}
        G(\theta) = A_m \cos(mx) + B_m \sin(mx). \\
       F(r) = r^m  \text{or } F(r) = r^{-m}  \qquad m \neq 0 \\
       F(r) = 0 \text{  or } F(r) = \log r   \qquad m = 0
    \end{align*}
    Since we find meaning $ u(0, \theta)$ solution, 
    only $F(r) = r^m$ remain, then by boundary condition 
    \[u(R, \theta) = A_0 / 2 + \sum_{m} (A_m R^m \cos m\theta + B_m R^m \sin m\theta) = f(\theta), \]
    therefore,
    \[A_m = \frac{1}{\pi R^m} \int_0^{2 \pi} f(\tau) \cos m \tau d \tau, 
     B_m = \frac{1}{\pi R^m} \int_0^{2 \pi} f(\tau) \sin m \tau d \tau\]
     In summary,
     \[u(r, \theta) = \frac{1}{\pi} \int_0^{2 \pi} \left(
         1/2 + \sum_{m} r^m \cos m(\theta - \tau) / R^m
     \right)  d\tau \]
\end{solution}

\paragraph*{3. }
Solve the initial boundary value problem of the equation
\begin{equation}
    \left\{
    \begin{aligned}
         & \partial_t^2 u = u_{xx} +u_{yy}, \qquad (x, y) \in [0,1] \times [0, \pi], \\
         & u(0, y, t) = u(1, y, t) = u_y(x, 0, t) = u_y(x, \pi, t) = 0               \\
         & u(x, y, 0) = f(x, y), \qquad u(x, y, 0) = 0.
    \end{aligned}
    \right.
\end{equation}

\begin{solution}
Assume $u = F(x) G(y) K(t)$, replace $u$ obtain $FGK'' = F''GK + FG''K \rightarrow K''/K = F''/F +
G''/G =: \lambda_1 + \lambda_2 $. 
    When $F''/F = \lambda_1$, $\lambda_1 > 0$ will be conflict with 
    $u(0, y, t) = u(1, y, t) = 0 \Rightarrow F(0) = F(1) = 0$, and $\lambda = 0$
    get $u = 0$, thus $\lambda_1 < 0$. Similarly, $\lambda_2 < 0$. 
    Combining with boundary condition, we have 
    \[F(x) = \sin (n \pi x), G(y) = \cos (m y)\].
    That means $K''/K = -(n\pi)^2 - m^2 =: \lambda_3$ and 
    \begin{align*}
        &K(t) = A_{n, m} \cos \sqrt{((n\pi)^2 + m^2) } t + B_{n, m} \sin \sqrt{((n\pi)^2 + m^2) } t
    \end{align*}.
    The coefficients $A_{n, m}, B_{n, m}$ get from double Fourier series 
    \begin{align*}
        A_{n, m} &= \frac{4}{\pi} \int_0^1 \int_0^{\pi} f(x, y)\sin (n \pi x) \cos (m y)  dy dx \\
        B_{n, m} &= \frac{4}{\pi} \int_0^1 \int_0^{\pi} 0 *\sin (n \pi x) \cos (m y)  dy dx = 0 \\
        \Rightarrow u(x, y, t) = F(x)G(y)K(t) &= \sum_n \sum_m \sin (n \pi x) \cos (m y)  A_{n, m} \cos \sqrt{((n\pi)^2 + m^2) } t  
    \end{align*}
\end{solution}

\end{document}
