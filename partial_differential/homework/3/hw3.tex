\documentclass[a4paper]{book}

\usepackage{geometry}
% make full use of A4 papers
\geometry{margin=1.5cm, vmargin={0pt,1cm}}
\setlength{\topmargin}{-1cm}
\setlength{\paperheight}{29.7cm}
\setlength{\textheight}{25.1cm}

% auto adjust the marginals
\usepackage{marginfix}

\usepackage{amsfonts}
\usepackage{amsmath}
\usepackage{amssymb}
\usepackage{amsthm}
%\usepackage{CJKutf8}   % for Chinese characters
\usepackage{ctex}
\usepackage{enumerate}
\usepackage{graphicx}  % for figures
\usepackage{layout}
\usepackage{multicol}  % multiple columns to reduce number of pages
\usepackage{mathrsfs}  
\usepackage{fancyhdr}
\usepackage{subfigure}
\usepackage{tcolorbox}
\usepackage{tikz-cd}
\usepackage{listings}
\usepackage{xcolor} %代码高亮
\usepackage{braket}
\usepackage{algorithm} 
\usepackage{algorithmicx}  
\usepackage{algpseudocode}  
\usepackage{amsmath}  

\floatname{algorithm}{算法}  
\renewcommand{\algorithmicrequire}{\textbf{输入:}}  
\renewcommand{\algorithmicensure}{\textbf{输出:}}  
\renewcommand{\algorithmicrequire}{\textbf{Input : }}
\renewcommand{\algorithmicrequire}{\textbf{Precondition : }}
\renewcommand{\algorithmicensure}{\textbf{Output : }}
\renewcommand{\algorithmicensure}{\textbf{Postcondition : }}
%------------------
% common commands %
%------------------
% differentiation
\newcommand{\gen}[1]{\left\langle #1 \right\rangle}
\newcommand{\dif}{\mathrm{d}}
\newcommand{\difPx}[1]{\frac{\partial #1}{\partial x}}
\newcommand{\difPy}[1]{\frac{\partial #1}{\partial y}}
\newcommand{\Dim}{\mathrm{D}}
\newcommand{\avg}[1]{\left\langle #1 \right\rangle}
\newcommand{\sgn}{\mathrm{sgn}}
\newcommand{\Span}{\mathrm{span}}
\newcommand{\dom}{\mathrm{dom}}
\newcommand{\Arity}{\mathrm{arity}}
\newcommand{\Int}{\mathrm{Int}}
\newcommand{\Ext}{\mathrm{Ext}}
\newcommand{\Cl}{\mathrm{Cl}}
\newcommand{\Fr}{\mathrm{Fr}}
% group is generated by
\newcommand{\grb}[1]{\left\langle #1 \right\rangle}
% rank
\newcommand{\rank}{\mathrm{rank}}
\newcommand{\Iden}{\mathrm{Id}}

% this environment is for solutions of examples and exercises
\newenvironment{solution}%
{\noindent\textbf{Solution.}}%
{\qedhere}
% Define a Solution environment like proof
\makeatletter
\newenvironment{sol}[1][\solname]{\par
  \pushQED{\qed}
  \normalfont \topsep6\p@\@plus6\p@\relax
  \trivlist
  \item[\hskip\labelsep
        \itshape
    #1\@addpunct{.}]\ignorespaces
}{\popQED\endtrivlist\@endpefalse}
\providecommand{\solname}{Solution}
\makeatother
% the following command is for disabling environments
%  so that their contents do not show up in the pdf.
\makeatletter
\newcommand{\voidenvironment}[1]{%
\expandafter\providecommand\csname env@#1@save@env\endcsname{}%
\expandafter\providecommand\csname env@#1@process\endcsname{}%
\@ifundefined{#1}{}{\RenewEnviron{#1}{}}%
}
\makeatother

%---------------------------------------------
% commands specifically for complex analysis %
%---------------------------------------------
% complex conjugate
\newcommand{\ccg}[1]{\overline{#1}}
% the imaginary unit
\newcommand{\ii}{\mathbf{i}}
%\newcommand{\ii}{\boldsymbol{i}}
% the real part
\newcommand{\Rez}{\mathrm{Re}\,}
% the imaginary part
\newcommand{\Imz}{\mathrm{Im}\,}
% punctured complex plane
\newcommand{\pcp}{\mathbb{C}^{\bullet}}
% the principle branch of the logarithm
\newcommand{\Log}{\mathrm{Log}}
% the principle value of a nonzero complex number
\newcommand{\Arg}{\mathrm{Arg}}
\newcommand{\Null}{\mathrm{null}}
\newcommand{\Range}{\mathrm{range}}
\newcommand{\Ker}{\mathrm{ker}}
\newcommand{\Iso}{\mathrm{Iso}}
\newcommand{\Aut}{\mathrm{Aut}}
\newcommand{\ord}{\mathrm{ord}}
\newcommand{\Res}{\mathrm{Res}}
%\newcommand{\GL2R}{\mathrm{GL}(2,\mathbb{R})}
\newcommand{\GL}{\mathrm{GL}}
\newcommand{\SL}{\mathrm{SL}}
\newcommand{\Dist}[2]{\left|{#1}-{#2}\right|}

\newcommand\tbbint{{-\mkern -16mu\int}}
\newcommand\tbint{{\mathchar '26\mkern -14mu\int}}
\newcommand\dbbint{{-\mkern -19mu\int}}
\newcommand\dbint{{\mathchar '26\mkern -18mu\int}}
\newcommand\bint{
{\mathchoice{\dbint}{\tbint}{\tbint}{\tbint}}
}
\newcommand\bbint{
{\mathchoice{\dbbint}{\tbbint}{\tbbint}{\tbbint}}
}





%----------------------------------------
% theorem and theorem-like environments %
%----------------------------------------
\numberwithin{equation}{chapter}
\theoremstyle{definition}

\newtheorem{thm}{Theorem}[chapter]
\newtheorem{axm}[thm]{Axiom}
\newtheorem{alg}[thm]{Algorithm}
\newtheorem{asm}[thm]{Assumption}
\newtheorem{defn}[thm]{Definition}
\newtheorem{prop}[thm]{Proposition}
\newtheorem{rul}[thm]{Rule}
\newtheorem{coro}[thm]{Corollary}
\newtheorem{lem}[thm]{Lemma}
\newtheorem{exm}{Example}[chapter]
\newtheorem{rem}{Remark}[chapter]
\newtheorem{exc}[exm]{Exercise}
\newtheorem{frm}[thm]{Formula}
\newtheorem{ntn}{Notation}

% for complying with the convention in the textbook
\newtheorem{rmk}[thm]{Remark}


%----------------------
% the end of preamble %
%----------------------

\begin{document}

%\tableofcontents
%\clearpage

\pagestyle{fancy}
\pagenumbering{roman}
%\lhead{Qinghai Zhang}
%\chead{Notes on Algebraic Topology}
%\rhead{Fall 2018}


\setcounter{chapter}{2}
\pagenumbering{arabic}
% \setcounter{page}{0}


% each chapter is factored into a separate file.

\chapter*{PDEhw3 12235005 谭焱}
\paragraph*{1. }
Find a solution to the following Dirichlet problem for the 
Laplace equation, by suing the Fourier transform:
\[ (\partial_x^2 + \partial_y^2)u = 0, 
(x,y) \in \mathbb{R} \times \mathbb{R}_+, u(x, 0) = f(x) \in 
\mathcal{S}(\mathbb{R})\]
\begin{sol}
    By taking Fourier transform get 
    \[ (\partial_y^2 - \xi^2)\hat{u} = (\partial_y + |\xi|)(\partial_y - |\xi|)\hat{u} = 0, 
(\xi, y) \in \mathbb{R} \times \mathbb{R}_+, \hat{u}(\xi, 0) = \hat{f}(\xi) \in 
\mathcal{S}(\mathbb{R}).\]
The equation's general solutions is $\hat{u} = A e^{-|\xi|y} +Be^{|\xi|y}$.
However, since $\hat{u} \in L^1 \Rightarrow \lim_{y \rightarrow \infty} \hat{u} = 0$,
$B = 0$. Combining with boundary condition
\[\hat{u}(\xi, 0) = A = \hat{f}(\xi).\]
According to 
\begin{align*}
     2 \pi \mathcal{F}^{-1}(e^{-|\xi|y}) &= \int e^{-|\xi|y + ix \cdot \xi} d\xi
     = \frac{1}{ix - y} - \frac{1}{-ix - y} = \frac{2y}{x^2 + y^2}
\end{align*}
\begin{align*}
    u(x, y) &= \mathcal{F}^{-1} \left( \hat{f} e^{-|\xi|y} d\xi\right) 
    =  \mathcal{F}^{-1} \left( \hat{f} \cdot \mathcal{F} (\frac{y}{\pi(x^2 +y^2)}) \right) \\
    &=\mathcal{F}^{-1} \mathcal{F} \left(\int \frac{y}{\pi(\xi^2 + y^2)} f(x - \xi) d \xi\right)
    =\int \frac{y}{\pi(\xi^2 + y^2)} f(x - \xi) d \xi
\end{align*} 
\end{sol}

\paragraph*{2. }
Check that any polynomial $P(x) \in \mathcal{S}'(\mathbb{R}^n)$,
however, $f(x) = e^{x^2} \notin \mathcal{S}'(\mathbb{R}), 
g(x) = e^s \notin \mathcal{S}'(\mathbb{R})$. Hint: you may want
to use test functions like $e^{-\sqrt{1 +x^2}}$.
\begin{sol}
    \begin{itemize}
        \item For polynomial $P(x)$, without lost general, assuming 
        the highest item is order $N$, so we have 
        \[ \int_{\mathbb{R}^n} (1 + \| x\|^2)^{-N - n} |P(x)| dx = C < \infty.\]
        Then 
        \begin{align*}
            \lim_{m \rightarrow \infty}|\int_{\mathbb{R}^n} P(x) (\phi_m 0 \phi) dx | &\leq 
            \int_{\mathbb{R}^n} (1 + \|x\|^2)^{-N -n} |P(x)| |(\lim_{m \rightarrow \infty} (1 + \|x\|^2)^{N+n} (\phi_m - \phi) ) dx | \\
            &\leq C \lim_{m \rightarrow \infty} P_{2(N+n)}(\phi_m - \phi)
            =0.
        \end{align*}
        It means $P(x) \in \mathcal{S}'$.

        \item Take $\phi_m = e^{-\sqrt{m + x^2}}$, we know 
        \[ \int \phi_m < \int e^{-|x|} = 2, \text{ and } 
        \lim_{m \rightarrow \infty} \phi_m = 0.\]
        However take $L_m$ large enough $x^{3/2} > \sqrt{m + x^2}$, $\lim_{m \rightarrow \infty}\int e^{x^2} \phi_m dx 
        \leq \int_{L_m}^{\infty} e^{x^{1/2}} dx + C = \infty$ indicate $f(x) \notin \mathcal{S}'(\mathbb{R})$.

        \item Take $\phi_m = e^{-\sqrt{m + |x|}}$, $\lim_{m \rightarrow \infty}\int e^{x} \phi_m dx 
        \leq \int e^{x^{1/3}} dx + C = \infty$ have $g(x) \notin \mathcal{S}'(\mathbb{R})$.
    \end{itemize}
\end{sol}

\paragraph*{3. }
Based on the formula 
\[ K_t(x) = (4\pi t)^{-1/2} e^{-|x|^2 / (4t)}, \mathcal{F}
(K_t)(\xi ) = e^{-t |\xi |^2}, t > 0, x \in \mathbb{R}.\]
\begin{itemize}
    \item Prove the formula holds for $t \in \mathbb{C}$ 
    with $\mathfrak{R}t > 0$.

    \item With $t = \epsilon  + i\lambda, \epsilon > 0, 
    \lambda \in \mathbb{R} \backslash \{0\}$, By considering
    limit in $\mathcal{S}'(\mathbb{R})$ as $\epsilon \rightarrow
    0+$, calculate $\mathcal{F}(K_{i\lambda})$.
\end{itemize}

\begin{sol}
    \begin{itemize}
        \item For $ t \in \mathbb{C}, \mathfrak{R} t > 0$,
        \begin{align*}
            \mathcal{F}(K_t)(\xi) &= \frac{1}{(4\pi t)^{1/2}} \int e^{-\frac{|x|^2}{4t} - ix\xi} dx 
            = \frac{1}{\pi^{1/2}} \int e^{-y^2 - i2 t^{1/2}y \xi} dy \\
            &= \frac{e^{-t \xi^2}}{\pi^{1/2}} \int e^{-(y - it^{1/2}\xi)^2}dy
            = e^{-t |\xi|^2} \qquad  \text{by }\mathfrak{R}(y-it^{1/2}\xi) = \mathfrak{R}\frac{|x|}{2t^{1/2}} > 0
        \end{align*}

        \item The same as above, replace $t$ with $\epsilon + i \lambda$ get 
        \begin{align*}
            \lim_{\epsilon \rightarrow 0}\mathcal{F}(K_{\epsilon + i\lambda})(\xi) 
            &= \lim_{\epsilon \rightarrow 0} \frac{1}{(4\pi (\epsilon +i\lambda))^{1/2}} \int e^{-\frac{|x|^2}{4(\epsilon +i\lambda)} - ix\xi} dx \\
            &= \lim_{\epsilon \rightarrow 0} \frac{1}{\pi^{1/2}} \int e^{-(\frac{x}{2\sqrt{\epsilon + i\lambda}} + i\sqrt{\epsilon + i\lambda}\xi)^2 -i\lambda\xi^2} d(\frac{x}{2\sqrt{\epsilon + i\lambda}}) \\
            &= \lim_{\epsilon \rightarrow 0}e^{-(\epsilon + i\lambda)|\xi|^2} \qquad \text{by } \mathfrak{R}(\frac{x}{2\sqrt{\epsilon + i\lambda}} + i\sqrt{\epsilon + i\lambda}\xi) > 0 \\
            &= e^{-i\lambda|\xi|^2}
        \end{align*}
    \end{itemize}
\end{sol}

\end{document}
