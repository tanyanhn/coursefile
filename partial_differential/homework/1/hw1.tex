\documentclass[a4paper]{book}

\usepackage{geometry}
% make full use of A4 papers
\geometry{margin=1.5cm, vmargin={0pt,1cm}}
\setlength{\topmargin}{-1cm}
\setlength{\paperheight}{29.7cm}
\setlength{\textheight}{25.1cm}

% auto adjust the marginals
\usepackage{marginfix}

\usepackage{amsfonts}
\usepackage{amsmath}
\usepackage{amssymb}
\usepackage{amsthm}
%\usepackage{CJKutf8}   % for Chinese characters
\usepackage{ctex}
\usepackage{enumerate}
\usepackage{graphicx}  % for figures
\usepackage{layout}
\usepackage{multicol}  % multiple columns to reduce number of pages
\usepackage{mathrsfs}  
\usepackage{fancyhdr}
\usepackage{subfigure}
\usepackage{tcolorbox}
\usepackage{tikz-cd}
\usepackage{listings}
\usepackage{xcolor} %代码高亮
\usepackage{braket}
\usepackage{algorithm} 
\usepackage{algorithmicx}  
\usepackage{algpseudocode}  
\usepackage{amsmath}
\usepackage{lmodern}  

\floatname{algorithm}{算法}  
\renewcommand{\algorithmicrequire}{\textbf{输入:}}  
\renewcommand{\algorithmicensure}{\textbf{输出:}}  
\renewcommand{\algorithmicrequire}{\textbf{Input : }}
\renewcommand{\algorithmicrequire}{\textbf{Precondition : }}
\renewcommand{\algorithmicensure}{\textbf{Output : }}
\renewcommand{\algorithmicensure}{\textbf{Postcondition : }}
%------------------
% common commands %
%------------------
% differentiation
\newcommand{\gen}[1]{\left\langle #1 \right\rangle}
\newcommand{\dif}{\mathrm{d}}
\newcommand{\difPx}[1]{\frac{\partial #1}{\partial x}}
\newcommand{\difPy}[1]{\frac{\partial #1}{\partial y}}
\newcommand{\Dim}{\mathrm{D}}
\newcommand{\avg}[1]{\left\langle #1 \right\rangle}
\newcommand{\sgn}{\mathrm{sgn}}
\newcommand{\Span}{\mathrm{span}}
\newcommand{\dom}{\mathrm{dom}}
\newcommand{\Arity}{\mathrm{arity}}
\newcommand{\Int}{\mathrm{Int}}
\newcommand{\Ext}{\mathrm{Ext}}
\newcommand{\Cl}{\mathrm{Cl}}
\newcommand{\Fr}{\mathrm{Fr}}
% group is generated by
\newcommand{\grb}[1]{\left\langle #1 \right\rangle}
% rank
\newcommand{\rank}{\mathrm{rank}}
\newcommand{\Iden}{\mathrm{Id}}

% this environment is for solutions of examples and exercises
\newenvironment{solution}%
{\noindent\textbf{Solution.}}%
{\qedhere}
% the following command is for disabling environments
%  so that their contents do not show up in the pdf.
\makeatletter
\newcommand{\voidenvironment}[1]{%
\expandafter\providecommand\csname env@#1@save@env\endcsname{}%
\expandafter\providecommand\csname env@#1@process\endcsname{}%
\@ifundefined{#1}{}{\RenewEnviron{#1}{}}%
}
\makeatother

%---------------------------------------------
% commands specifically for complex analysis %
%---------------------------------------------
% complex conjugate
\newcommand{\ccg}[1]{\overline{#1}}
% the imaginary unit
\newcommand{\ii}{\mathbf{i}}
%\newcommand{\ii}{\boldsymbol{i}}
% the real part
\newcommand{\Rez}{\mathrm{Re}\,}
% the imaginary part
\newcommand{\Imz}{\mathrm{Im}\,}
% punctured complex plane
\newcommand{\pcp}{\mathbb{C}^{\bullet}}
% the principle branch of the logarithm
\newcommand{\Log}{\mathrm{Log}}
% the principle value of a nonzero complex number
\newcommand{\Arg}{\mathrm{Arg}}
\newcommand{\Null}{\mathrm{null}}
\newcommand{\Range}{\mathrm{range}}
\newcommand{\Ker}{\mathrm{ker}}
\newcommand{\Iso}{\mathrm{Iso}}
\newcommand{\Aut}{\mathrm{Aut}}
\newcommand{\ord}{\mathrm{ord}}
\newcommand{\Res}{\mathrm{Res}}
%\newcommand{\GL2R}{\mathrm{GL}(2,\mathbb{R})}
\newcommand{\GL}{\mathrm{GL}}
\newcommand{\SL}{\mathrm{SL}}
\newcommand{\Dist}[2]{\left|{#1}-{#2}\right|}

\newcommand\tbbint{{-\mkern -16mu\int}}
\newcommand\tbint{{\mathchar '26\mkern -14mu\int}}
\newcommand\dbbint{{-\mkern -19mu\int}}
\newcommand\dbint{{\mathchar '26\mkern -18mu\int}}
\newcommand\bint{
{\mathchoice{\dbint}{\tbint}{\tbint}{\tbint}}
}
\newcommand\bbint{
{\mathchoice{\dbbint}{\tbbint}{\tbbint}{\tbbint}}
}





%----------------------------------------
% theorem and theorem-like environments %
%----------------------------------------
\numberwithin{equation}{chapter}
\theoremstyle{definition}

\newtheorem{thm}{Theorem}[chapter]
\newtheorem{axm}[thm]{Axiom}
\newtheorem{alg}[thm]{Algorithm}
\newtheorem{asm}[thm]{Assumption}
\newtheorem{defn}[thm]{Definition}
\newtheorem{prop}[thm]{Proposition}
\newtheorem{rul}[thm]{Rule}
\newtheorem{coro}[thm]{Corollary}
\newtheorem{lem}[thm]{Lemma}
\newtheorem{exm}{Example}[chapter]
\newtheorem{rem}{Remark}[chapter]
\newtheorem{exc}[exm]{Exercise}
\newtheorem{frm}[thm]{Formula}
\newtheorem{ntn}{Notation}

% for complying with the convention in the textbook
\newtheorem{rmk}[thm]{Remark}


%----------------------
% the end of preamble %
%----------------------

\begin{document}
\pagestyle{plain}
\pagenumbering{roman}

%\tableofcontents
%\clearpage

\pagestyle{fancy}
%\fancyhead{}
\chapter{PDEhw1 12235005 谭焱}
\paragraph*{Section 1.1}
\begin{itemize}
    \item [(2)] Find general solutions for spherically symmetric functions 
    $V = V(|x|)$ such that 
    $\Delta V = 0, x \in \mathbb{R}^n \backslash \{0\}$ with $n \geq 2$.

    \item [(3)] (minimal surface) Let $u : \Omega \rightarrow \mathbb{R}$
    with $\Omega \in \mathbb{R}^2$. For fixed function $f$ on the boundary of 
    $\Omega$, suppose $u$ is a function such that it minimizes the area
    \[ A[u] = \int_{\Omega} \sqrt{1 + | \nabla u|^2 } d\mathbf{x} ,\]
    among all surface with $u = f$ on the boundary of $Omega$. 
    Try to derive a PDE for $u$.
\end{itemize}

\begin{solution}
    \begin{itemize}
        \item [(2)] Since $| x| = (\sum x_i^2)^{1/2}$,
        we have $\partial |x| / \partial x = x / |x|$.

        So, \[V' = V'(|x|) \partial |x| / \partial x
        = V'x / |x|,\]
        and \[\Delta V = V_{x_i x_i} + V_{x_i} (n - 1) / |x|.\]
        That means 
         \[\Delta V = 0 \Longleftrightarrow V'' + (n - 1)V'/ |x| = 0. \]
         Differential $r^{n-1} V'$ get
         \[(r^{n - 1} V')' = r^{n - 1}\left( V'' + (n - 1)V' / |x|\right) = 0.\]
         hence,
            \[V' = a / r^{n - 1}. \]
            Therefore 
            \[ V = \left\{ 
                \begin{aligned}
                &a \log |x| + b  \qquad & (n = 2) \\
                &a / |x|^{n - 2} + b &(n \geq 3)
            \end{aligned} \right. .\]

            \item[(3)] Assume $u$ match the condition, For any smooth
            function $v : \Omega \rightarrow \mathbb{R}$ and Real $\delta$.
            Define $w(\delta) = u + \delta v$ satisfy $w(\delta) = f$ on the $\partial\Omega$.
            So $A[u] \leq A[w(\delta)]$.

            Hence 
            \begin{align*}
                w'(\delta) = \int_{\Omega} \frac{ (| \nabla v |^2 \delta + \nabla u \cdot \nabla v)}
                {\sqrt{1 + | \nabla u|^2 }} d\mathbf{x}.
            \end{align*}
            Set $\delta = 0$,
            \begin{align*}
                0 &= w'(0) = \int_{\Omega} \frac{\Delta u \cdot \Delta v}{\sqrt{1 + | \nabla u|^2 }} d\mathbf{x} = \int_{\Omega} v \Delta \cdot \left( \frac{\Delta u}{\sqrt{1 + | \nabla u|^2 }}\right) d\mathbf{x} \qquad \text{Since $v = 0$, on } \partial \Omega
            \end{align*}.

            By random of $v$, we know
            \[\Delta \cdot \left( \frac{\Delta u}{\sqrt{1 + | \nabla u|^2 }} \right) = 0 \]
    \end{itemize}
\end{solution}


\paragraph*{Section 1.3}
\begin{itemize}
    \item [(3)] Solve $u_t + uu_x = 0, u(0, x)= -x$.
\end{itemize}

\begin{solution}
    \begin{itemize}
        \item [(3)] Firstly,
        \[       
        \det \left( 
            \begin{array}{cc}
                \partial_x g^1 & 1 \\
                \partial_x g^2 & u
            \end{array} \right) = -1 \neq 0.
        \]

        We obtain \[t = s, x = -x_0 s + x_0, u = - x_0\].
        means $\forall (t, x) \in \mathbb{R}^2, u(t, x) = -x_0$,
        
        and since \[x_0 = \frac{x}{ 1 - s} = \frac{x}{1 - t}\].
        So
        \[ u = \frac{x}{t - 1}\].
    \end{itemize}
\end{solution}

\paragraph*{Section 1.5}
\begin{itemize}
    \item [(5)] Solve $u_t + xu_x = x$ with data $u(0, x) = 2x$,
    by applying the power series method.
\end{itemize}

\begin{solution}
    \begin{itemize}
        \item [(5)] Assume 
        \[u(t, x) = \sum \frac{c_{j,k}}{j! k!} t^j x^k + 2x.\]
        So $c_{0, 1} = 0$, substitute into $u_t + xu_x = x$ know 
\[\sum_{j \neq 0} (\frac{c_{j + 1, k} + c_{j, k}}{j! k!}) t^j x^k + \sum\frac{c_{1, k}}{ k!} x^k =
-x.\]
        Which imply 
        \[ c_{1, k} = \left\{ \begin{aligned}
            &-1 \qquad & k = 1\\
            &0  \qquad & k \neq 1
        \end{aligned}  \right.\]
        \[c_{j + 1,k} = -c_{j, k}, \qquad j \neq 0.\]
        So 
        \[c_{j, k} = \left\{
            \begin{aligned}
                &(-1)^j  \qquad & k = 1 \\
                &0                       & k \neq 1
            \end{aligned}
        \right..\]
        Therefore,
            \begin{align*}
                u(t, x) = \sum_{j \neq 0} \frac{(-t)^j x}{j!} + 2x 
            \end{align*}

    \end{itemize}
\end{solution}
    
\end{document}
