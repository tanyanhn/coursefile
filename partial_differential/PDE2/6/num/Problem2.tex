\begin{pro}
  Prove the following finite speed of propagation property for problem \eqref{eq:1},
  with homogeneous Dirichlet or Neumann boundary conditions.
  If $f=g=0$ in $B_1(\mathbf{0})\subset\Omega$,
  then there exists a region in $\mathbb{R}_+\times\Omega$
  where $u$ vanishes identically.
\end{pro}
\begin{proof}
  First we introduce some notations for convenience.
  Denote the ball $B_t$
  in $\mathbb{R}_+\times\Omega$ by
  \begin{displaymath}
    B_t = \{(\mathbf{x}, t)| \|\mathbf{x}\|\le 1-t\},
  \end{displaymath}
  and the cone $K(\mathbf{0}, 1)$ in $\mathbb{R}_+\times\Omega$ by
  \begin{displaymath}
    K(\mathbf{0}, 1) = \cup_{t\in[0,1]}B_t
    = \{(\mathbf{x},t) | 0\le t\le 1, \|\mathbf{x}\|\le 1-t\}.
  \end{displaymath}
  Define the energy $e(t)$ by
  \begin{displaymath}
    e(t) = \frac{1}{2}\int_{B_t}u_t^2+qu^2+c^2\|\nabla u\|^2\dif \mathbf{x},
  \end{displaymath}
  compute that
  \begin{align*}
    e'(t) &= \int_{B_t}u_tu_{tt} + quu_t+c^2\nabla u\cdot(\nabla u)_t\dif\mathbf{x} - \frac{1}{2}\int_{\partial B_t}u_t^2+qu^2+c^2\|\nabla u\|^2\dif S \\
          &= \int_{B_t}u_t\nabla\cdot(c^2\nabla u) + c^2\nabla u\cdot(\nabla u)_t\dif\mathbf{x} - \frac{1}{2}\int_{\partial B_t}u_t^2+qu^2+c^2\|\nabla u\|^2\dif S \\
          &= \int_{\partial B_t}u_t(c^2\nabla u)\cdot \mathbf{n}\dif S - \frac{1}{2}\int_{\partial B_t}u_t^2+qu^2+c^2\|\nabla u\|^2\dif S \\
          &\le \int_{\partial B_t}\frac{1}{2}u_t^2 + \frac{1}{2}\LP c^2\nabla u\cdot \mathbf{n}\RP^2\dif S - \frac{1}{2}\int_{\partial B_t}u_t^2+qu^2+c^2\|\nabla u\|^2\dif S \\
          &\le \int_{\partial B_t}\frac{1}{2}u_t^2 + \frac{1}{2}\|c^2\nabla u\|^2\|\mathbf{n}\|^2\dif S - \frac{1}{2}\int_{\partial B_t}u_t^2+qu^2+c^2\|\nabla u\|^2\dif S \\
          &= -\frac{1}{2}\int_{\partial B_t}qu^2\dif S - \frac{1}{2}\int_{\partial B_t}c^2(1-c^2)\|\nabla u\|^2\dif S \\
    &\le 0
  \end{align*}
  where the second step follows from \eqref{eq:1},
  the third from the integration-by-parts formula,
  the fourth from Cauchy's inequality, the fifth from
  Cauchy-Schwarz inequality and the last from the assumption that
  $p(\mathbf{x})$ is nonnegative and $0<c(\mathbf{x})<1, \forall \mathbf{x}\in\Omega$.
  We have
  \begin{displaymath}
    e(0) = \frac{1}{2}\int_{B_1(\mathbf{0})}u_t^2+qu^2+c^2\|\nabla u\|^2\dif \mathbf{x} = \frac{1}{2}\int_{B_1(\mathbf{0})}g^2+qf^2+c^2\|\nabla f\|^2\dif \mathbf{x} = 0,
  \end{displaymath}
  and
  \begin{displaymath}
    e(t) = \frac{1}{2}\int_{B_t}u_t^2+qu^2+c^2\|\nabla u\|^2\dif \mathbf{x} \ge 0
  \end{displaymath}
  $e(t)$ is non-increasing in $[0, \infty)$
  since we have shown that $e'(t)<0$ for $t\ge 0$,
  and therefore
  \begin{displaymath}
    e(t) = 0, \quad \forall t\ge 0 \Rightarrow
    u \equiv 0 \text{ in } B_t.
  \end{displaymath}
  Consequently,
  \begin{displaymath}
    u \equiv 0 \text{ in } K(\mathbf{0}, 1).
  \end{displaymath}
\end{proof}
%%% Local Variables:
%%% mode: latex
%%% TeX-master: "../hw6"
%%% End:
