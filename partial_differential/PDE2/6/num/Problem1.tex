\begin{pro}
  Let $\Omega$ be an open bounded domain in $\mathbb{R}^n$,
  $c$ and $q$ nonnegative smooth functions.
  Assume
  $\exists\delta>0 \text{ s.t. }\delta<c(\mathbf{x})<\delta^{-1}$
  for any $\mathbf{x}\in\Omega$.
  Consider the real-valued variable-coefficient wave equation
  \begin{equation}
    \label{eq:1}
    u_{tt}-\nabla\cdot(c^2(\mathbf{x})\nabla u) + q(\mathbf{x})u = 0,
    u(0,\mathbf{x})=f(\mathbf{x}),
    u_t(0,\mathbf{x}) = g(\mathbf{x}),
    \mathbf{x}\in\Omega,
  \end{equation}
  with homogoneous Neumann boundary condition
  \begin{displaymath}
    \mathbf{n}(\mathbf{x})\cdot\nabla u(t,\mathbf{x}) = 0, \mathbf{x}\in\partial\Omega,
  \end{displaymath}
  where $\mathbf{n}(\mathbf{x})$ denotes the outward unit normal.
  Multiply the equation by $u_t$ and apply the integration-by-parts formula
  to find the energy corresponding to this problem and prove that
  the energy so defined is conserved,
  assuming $u\in\mathcal{C}^2$ with compact support for any time $t$.
\end{pro}
\begin{proof}
  Multiplying the wave equation by $u_t$ and integrating over $\Omega$ yields
  \begin{displaymath}
    \int_{\Omega}u_tu_{tt}\dif\mathbf{x}
    - \int_{\Omega}u_t\nabla\cdot(c^2\nabla u)\dif\mathbf{x}
    + \int_{\Omega}u_tqu = 0.
  \end{displaymath}
  Use the fact that $\frac{1}{2}(u_t^2)_t = u_tu_{tt},
  \frac{1}{2}(u^2)_t = uu_t$ and integrate by part,
  and we have
  \begin{align*}
    0 &= \int_{\Omega}\frac{1}{2}\frac{\dif}{\dif t} u_t^2\dif\mathbf{x}
    + \int_{\Omega}\frac{q}{2}\frac{\dif}{\dif t}u^2\dif\mathbf{x}
    + \int_{\Omega}c^2\nabla u\cdot(\nabla u)_t\dif\mathbf{x} -
        \int_{\partial\Omega}u_t c^2\nabla u\cdot\mathbf{n}\dif S \\
      &=\frac{1}{2}\frac{\dif}{\dif t}\LP
        \int_{\Omega}u_t^2 + qu^2 +c^2\|\nabla u\|^2\dif\mathbf{x}\RP,
  \end{align*}
  where the second equality follows by applying the boundary condition.
  If we define the energy $E(t)$ by
  \begin{displaymath}
    E(t) = \frac{1}{2}\int_{\Omega}u_t^2+qu^2+c^2\|\nabla u\|^2\dif \mathbf{x},
  \end{displaymath}
  the above argument shows that $E'(t)\equiv 0$,
  and therefore the energy $E(t)$ is conserved.
\end{proof}
%%% Local Variables:
%%% mode: latex
%%% TeX-master: "../hw6"
%%% End:
