\begin{pro}
  Consider the (uniformly) elliptic operator with (continuous) variable-coefficient
  \begin{displaymath}
    L = \sum_{i, j}a^{ij}(\mathbf{x})\partial_i\partial_j + \sum_ib^i(\mathbf{x})\partial_i + c(\mathbf{x})
  \end{displaymath}
  Assume that $\exists\mu>0 \text{ s.t. the matrix } A-\mu I$ is positive
  semidefinite, i.e.,
  \begin{displaymath}
    \exists \mu>0 \text{ s.t. } \forall\mathbf{x}\in\Omega,
    \forall\xi\in\mathbb{R}^n, \sum_{i,j=1}^na^{ij}(\mathbf{x})\xi_i\xi_j
    \ge \mu\|\xi\|^2.
  \end{displaymath}
  Prove the following theorem.
  \begin{thm}[Weak maximum principle(version 2)]
    If $u\in\mathcal{C}^2(\Omega)\cap\mathcal{C}(\overline{\Omega})$
    such that
    \begin{displaymath}
      Lu \ge 0 \ge c(\mathbf{x}) \text{ in } \Omega,
    \end{displaymath}
    then
    \begin{displaymath}
      \max_{\overline{\Omega}} u \le \max_{\partial\Omega}u^+, 
    \end{displaymath}
    where $u^+(\mathbf{x}) = \max(u(\mathbf{x}), 0)$.
  \end{thm}
\end{pro}
\begin{proof}
  Set $V := \{\mathbf{x}\in\Omega| u(\mathbf{x})>0\}$.
  Then
  \begin{displaymath}
    Ku := Lu - cu \ge -cu \ge 0 \text{ in } V.
  \end{displaymath}
  The operator $K$ has no zeroth-order term
  and consequently the theorem in Problem 3 implies
  \begin{displaymath}
    \max_{\overline{V}}u = \max_{\partial V}u = \max_{\partial U}u^+.
  \end{displaymath}
  This gives the desired result in the case that $V\neq \emptyset$.
  Otherwise $u\le 0$ everywhere in $U$,
  and the desired result likewise follows.
\end{proof}
%%% Local Variables:
%%% mode: latex
%%% TeX-master: "../hw6"
%%% End:
