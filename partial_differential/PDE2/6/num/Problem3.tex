\begin{pro}
  Consider the (uniformly) elliptic operator with (continuous)
  variable-coefficient
  \begin{displaymath}
  L=\sum_{i,j}a^{ij}(\mathbf{x})\partial_i\partial_j +
  \sum_ib^i(\mathbf{x})\partial_i.
\end{displaymath}
Assume that $\exists \mu>0 \text{ s.t. the matrix } A-\mu I$ is positive semidefinite,
i.e.,
  \begin{displaymath}
    \exists \mu>0 \text{ s.t. } \forall \mathbf{x}\in\Omega,
    \forall \xi\in\mathbb{R}^n, \sum_{i,j=1}^na^{ij}(\mathbf{x})\xi_i\xi_j \ge
    \mu\|\xi\|^2.
  \end{displaymath}
  Try to formulate and prove the corresponding weak maximum principle in $\Omega$.
  (Hint: You may want to construct $\phi=e^{\lambda x_1}$ with
  $1\ll \lambda$.)
\end{pro}
\begin{sol}
  The following is a reiteration of the classical PDE textbook by Evans.
  \begin{thm}[Weak maximum principle]
    Let $\Omega$ be a connected open bounded set.
    Assume
    $u\in\mathcal{C}^2(\Omega)\cap\mathcal{C}(\overline{\Omega})$,
    if
    \begin{displaymath}
      Lu \ge 0 \text{ in } \Omega,
    \end{displaymath}
    then
    \begin{displaymath}
      \max_{\overline{\Omega}}u = \max_{\partial\Omega}u.
    \end{displaymath}
  \end{thm}
  \begin{proof}
    \begin{itemize}
    \item[1.]
      Let us first suppose we have the strict inequality
      \begin{equation}
        \label{eq:2}
        Lu > 0 \text{ in } \Omega,
      \end{equation}
      and yet there exists a point $\mathbf{x}_0\in\Omega$ with
      \begin{equation}
        \label{eq:3}
        u(\mathbf{x}_0) = \max_{\overline{\Omega}}u.
      \end{equation}
      Now at this maximum point $\mathbf{x}_0$,
      we have
      \begin{equation}
        \label{eq:4}
        Du(\mathbf{x}_0) = 0
      \end{equation}
      and
      \begin{equation}
        \label{eq:5}
        D^2u(\mathbf{x}_0) \le 0.
      \end{equation}

    \item[2.]
      Since the matrix $A=((a^{jk}(\mathbf{x}_0)))$ is symmetric
      and positive definite,
      there exists an orthogonal matrix $O = ((o_{ij}))$ so that
      \begin{equation}
        \label{eq:6}
        OAO^T = \text{diag}(d_1, \ldots, d_n), \quad OO^T = I.
      \end{equation}
      with $d_k>0 (k=1, \ldots, n)$.
      Write $\mathbf{y}=\mathbf{x}_0+O(\mathbf{x}-\mathbf{x}_0)$.
      Then $\mathbf{x}-\mathbf{x}_0=O^T(\mathbf{y}-\mathbf{x}_0)$,
      and so
      \begin{displaymath}
        u_{x_i} = \sum_{k=1}^nu_{y_k}o_{ki}, \quad
        u_{x_ix_j} = \sum_{k, l=1}^nu_{y_ky_l}o_{ki}o_{lj} \quad
        (i, j = 1, \ldots, n).
      \end{displaymath}
      Hence at the point $\mathbf{x}_0$,
      \begin{equation}
        \label{eq:7}
        \sum_{i,j=1}^na^{ij}u_{x_ix_j} = \sum_{k,l=1}^n\sum_{i,j=1}^na^{ij}
        u_{y_ky_l}o_{ki}o_{lj} = \sum_{k=1}^nd_ku_{y_ky_k} \text{ (by \eqref{eq:6})} \le 0,
      \end{equation}
      since $d_k>0$ and $u_{y_ky_k}(\mathbf{x}_0)\le 0 (k=1, \ldots, n)$,
      according to \eqref{eq:5}.

    \item[3.]
      Thus at $\mathbf{x}_0$
      \begin{displaymath}
        Lu = \sum_{i,j=1}^na^{ij}u_{x_ix_j} + \sum_{i=1}b^iu_{x_i} \le 0,
      \end{displaymath}
      in light of \eqref{eq:4} and \eqref{eq:7}.
      So \eqref{eq:2} and \eqref{eq:3} are incompatible,
      and we have a contradiction.

    \item[4.]
      In the general case,
      write
      \begin{displaymath}
        u^{\epsilon}(\mathbf{x}) := u(\mathbf{x}) + \epsilon e^{\lambda x_1}
        \quad (\mathbf{x}\in\Omega),
      \end{displaymath}
      where $\lambda>0$ will be selected below and $\epsilon>0$.
      Recall that the uniform ellipticity condition implies
      $a^{ii}(\mathbf{x})\ge \mu (i=1, \ldots, n, \mathbf{x}\in\Omega)$.
      Therefore
      \begin{displaymath}
        Lu^{\epsilon} = Lu + \epsilon L(e^{\lambda x_1})
        \ge \epsilon e^{\lambda x_1}(\lambda^2a^{11}+\lambda b^1)
        \ge \epsilon e^{\lambda x_1}(\lambda^2\mu - \|\mathbf{b}\|_{L^{\infty}}\lambda) > 0 \text{ in }\Omega,
      \end{displaymath}
      provided we choose $\lambda>0$ sufficiently large.
      Then according to steps 1 and 2 above
      $\max_{\overline{\Omega}}u^{\epsilon} = \max_{\partial\Omega}u^{\epsilon}$.
      Let $\epsilon\to 0$ to find
      $\max_{\overline{\Omega}}u = \max_{\partial\Omega} u$.
      This completes the proof.
    \end{itemize}
  \end{proof}
\end{sol}
%%% Local Variables:
%%% mode: latex
%%% TeX-master: "../hw6"
%%% End:
