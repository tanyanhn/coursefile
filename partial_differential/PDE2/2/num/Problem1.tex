\begin{pro}
  Solve $u_{tt}-c^2u_{xx}=F(t, x)$ for
  $t>0$ and $x\in\mathbb{R}$,
  with data $u(0, x)=0, u_t(0, x) = 0$
  for $x\in\mathbb{R}$.
\end{pro}

\begin{sol}
  First, let's review Duhamel's principle.
  
  Define $u=u(t, x; s)$ to be the solution of
  \begin{displaymath}
    \begin{cases}
      u_{tt}(s; \cdot) - c^2u_{xx}(s; \cdot) = 0 \text{ in } (s, \infty)\times\mathbb{R} \\
      u(s; \cdot) = 0, u_t(s; \cdot) = F(\cdot, s) \text{ on } \{t=s\}\times\mathbb{R}.
    \end{cases}
  \end{displaymath}

  Now set
  \begin{displaymath}
    u(t, x) := \int_0^tu(t, x; s) \dif s \quad (t\ge 0, x\in\mathbb{R}).
  \end{displaymath}

  Duhamel's principle asserts this is a solution of
  \begin{displaymath}
    \begin{cases}
      u_{tt}-c^2u_{xx} = F \text{ in } (0, \infty)\times\mathbb{R} \\
      u=0, u_t = 0 \text{ on } \{t=0\}\times\mathbb{R}.
    \end{cases}
  \end{displaymath}

  Apply d'Alembert's formula and Duhamel's principle, and we have
  \begin{displaymath}
    u(t, x; s) = \frac{1}{2c}\int_{x-c(t-s)}^{x+c(t-s)}F(s, y)\dif y.
  \end{displaymath}
  \begin{align*}
    u(t, x) = \int_0^tu(t, x; s)\dif s = \frac{1}{2c}\int_0^t\int_{x-c(t-s)}^{x+c(t-s)}F(s, y)\dif y\dif s.
  \end{align*}
  % \begin{lem}
  %   The solution of the following nonhomogeneous transport equation
  %   \begin{displaymath}
  %     \begin{cases}
  %       u_t + \mathbf{b}\cdot Du = f \text{ in } \mathbb{R}^n\times(0, \infty) \\
  %       u = g \text{ on } \mathbb{R}^n\times\{t=0\}.
  %     \end{cases}
  %   \end{displaymath}
  %   is
  %   \begin{displaymath}
  %     u(\mathbf{x}, t) = g(\mathbf{x}-t\mathbf{b}) + \int_0^t f(\mathbf{x}+(s-t)\mathbf{b}, s) \dif s
  %     \quad (x\in\mathbb{R}^n, t\ge 0).
  %   \end{displaymath}
  % \end{lem}

  % Note that the PDE can be ``factored'',
  % to read
  % \begin{equation}
  %   \label{eq:1}
  %   \LP \frac{\partial}{\partial t} - c\frac{\partial}{\partial x}\RP
  %   \LP \frac{\partial}{\partial t} + c\frac{\partial}{\partial x}\RP u
  %   = u_{tt} - c^2u_{xx} = F(t, x).
  % \end{equation}

  % Write
  % \begin{equation}
  %   \label{eq:2}
  %   v(t, x) := \LP \frac{\partial}{\partial t} + c\frac{\partial}{\partial x}\RP u(t, x).
  % \end{equation}

  % Then \eqref{eq:1} says
  % \begin{displaymath}
  %   v_t(t, x) - cv_x(t, x) = F(t, x) \quad (x\in\mathbb{R}, t>0).
  % \end{displaymath}
  
  % Applying the above lemma (with $n=1, b=c$),
  % we find
  % \begin{equation}
  %   \label{eq:3}
  %   v(t, x) = \int_0^tF(s, x-c(s-t))\dif s.
  % \end{equation}

  % Combining now \eqref{eq:1}-\eqref{eq:3},
  % we obtain
  % \begin{displaymath}
  %   u_t(t, x) + cu_x(t, x) = v(t, x)\text{ in }
  %   \mathbb{R}\times(0, \infty).
  % \end{displaymath}

  % Applying the above lemma gives
  % \begin{align*}
  %   u(t, x) &= \int_0^tv(s, x+c(s-t))\dif s =
  %             \int_0^t\int_0^sF(\tau, x+c(s-t)-c(\tau-s))\dif \tau\dif s \\
  %           &= \int_0^t\int_0^sF(\tau, x-c(t+\tau-2s))\dif\tau\dif s \\
  %           &= \int_0^t\int_0^{\tau}F(\tau, x-c(t+\tau-2s))\dif s\dif \tau \\
  %   &= \frac{1}{2c}\int_0^t\int_{x+c(t-\tau)}^{x+c(t+\tau)}F(\tau, \xi)\dif \xi\dif \tau
  % \end{align*}
\end{sol}
%%% Local Variables:
%%% mode: latex
%%% TeX-master: "../hw2"
%%% End:
