\begin{pro}[Poisson's formula]
  Solve the boundary value problem of the Laplace equation in the disc:
  \begin{displaymath}
    \begin{cases}
      \Delta u = u_{xx}+u_{yy} = 0, \quad x^2 + y^2 < R^2, \\
      u(R\cos\theta, R\sin\theta) = f(\theta)
    \end{cases}
  \end{displaymath}
  by the method of separation of variables (in polar coordinates).
  You should finally obtain the celebrated Poisson's formula
  \begin{displaymath}
    u(r, \theta) = \frac{R^2-r^2}{2\pi}\int_0^{2\pi}\frac{f(\phi)}{R^2-2Rr\cos(\theta-\phi)+r^2}\dif\phi
    = \frac{R^2-\|\mathbf{x}\|^2}{2\pi R}\int_{\|\mathbf{y}\|=R}
    \frac{u(\mathbf{y})}{\|\mathbf{x}-\mathbf{y}\|^2}\dif S(\mathbf{y}).
  \end{displaymath}
\end{pro}

\begin{sol}
  First we rewrite the Laplacian in polar coordinates.
  An application of the chain rule gives
  \begin{displaymath}
    \Delta u = \frac{\partial^2 u}{\partial r^2} +
    \frac{1}{r}\frac{\partial u}{\partial r} + \frac{1}{r^2}\frac{\partial^2u}{\partial\theta^2}.
  \end{displaymath}

  We now multiply both sides by $r^2$,
  and since $\Delta u=0$, we get
  \begin{displaymath}
    r^2 \frac{\partial^2u}{\partial r^2} + r \frac{\partial u}{\partial r}
    = -\frac{\partial^2 u}{\partial \theta^2}.
  \end{displaymath}

  Separating these variables,
  and looking for a solution of the form $u(r, \theta) = F(r)G(\theta)$,
  we find
  \begin{displaymath}
    \frac{r^2F''(r)+rF'(r)}{F(r)} = -\frac{G''(\theta)}{G(\theta)}.
  \end{displaymath}

  Since the two sides depend on different variables,
  they must both be constant, say equal to $\lambda$.
  We therefore get the following equations:
  \begin{displaymath}
    \begin{cases}
      G''(\theta) + \lambda G(\theta) = 0, \\
      r^2F''(r) + rF'(r) - \lambda F(r) = 0.
    \end{cases}
  \end{displaymath}
  Since $G$ must be periodic of period $2\pi$,
  this implies that $\lambda\ge 0$ and that $\lambda=m^2$
  where $m$ is an integer;
  hence
  \begin{displaymath}
    G(\theta) = \tilde{A}\cos m\theta + \tilde{B}\sin m\theta.
  \end{displaymath}

  % An application of Euler's identity, $e^{ix}=\cos x+i\sin x$,
  % allows one to rewrite $G$ in terms of complex exponentials,
  % \begin{displaymath}
  %   G(\theta) = Ae^{im\theta} + Be^{-im\theta}.
  % \end{displaymath}

  With $\lambda=m^2$ and $m\neq 0$,
  two simple solutions of the equation in $F$ are $F(r)=r^m$ and $F(r)=r^{-m}$.
  If $m=0$, then $F(r)=0$ and $F(r)=\log r$ are two solutions.
  If $m>0$,
  we note that $r^{-m}$ grows unboundedly large as $r$ tends to zero,
  so $F(r)G(\theta)$ is unbounded at the origin; the same occurs when $m=0$
  and $F(r)=\log r$.
  We reject these solutions as contrary to our intuition.
  Therefore,
  we are left with the following special functions
  \begin{displaymath}
    % u_m(r, \theta) = r^{|m|}e^{im\theta}, \quad m\in\mathbb{Z}.
    F(r) = r^m, \quad G(\theta) = \tilde{A}\cos m\theta + \tilde{B}\sin m\theta.
  \end{displaymath}
  We now make the important observation that the PDE is linear,
  and so we may superpose the above special solutions to obtain the
  presumed general solution:
  \begin{displaymath}
    u(r, \theta) = \frac{a_0}{2} + \sum_{m=1}^{\infty}(a_mr^m\cos m\theta +
    b_m r^m\sin m\theta).
  \end{displaymath}
  Let $r=R$ and use the boundary condition, and we have
  \begin{displaymath}
    f(\theta) = \frac{a_0}{2} + \sum_{m=1}^{\infty}(a_mR^m\cos m\theta + b_mR^m\sin m\theta).
  \end{displaymath}
  Thus
  \begin{displaymath}
    a_m = \frac{1}{\pi R^m}\int_0^{2\pi}f(\phi)\cos m\phi\dif \phi, \quad
    b_m = \frac{1}{\pi R^m}\int_0^{2\pi}f(\phi)\sin m\phi\dif \phi.
  \end{displaymath}
  \begin{align*}
    u(r, \theta) = \frac{1}{2\pi}\int_0^{2\pi}f(\phi)\dif \phi
    + \frac{1}{\pi}\sum_{m=1}^{\infty}\LP \frac{r}{R}\RP^m
    \int_0^{2\pi}f(\phi)(\cos m\phi\cos m\theta + \sin m\phi\sin m\theta)\dif \phi.
  \end{align*}
  Consider $r<\tilde{R}<R$.
  Since the series converges uniformly there,
  we can interchange the order of summation and integration,
  and obtain
  \begin{equation}
    \label{eq:7}
    u(r, \theta) = \frac{1}{\pi}\int_0^{2\pi}f(\phi)\left[ \frac{1}{2} +
    \sum_{m=1}^{\infty}\LP \frac{r}{R}\RP^m\cos m(\theta-\phi)\right]\dif\phi.
\end{equation}
The summation of the infinite series
\begin{displaymath}
  \frac{1}{2} + \sum_{m=1}^{\infty}\LP \frac{r}{R}\RP^m\cos m(\theta-\phi)
\end{displaymath}
requires a little side calculation.
Define for this purpose $z=\rho e^{i\alpha}$ and evaluate (for $\rho<1$)
the geometric sum
\begin{displaymath}
  \frac{1}{2} + \sum_{m=1}^{\infty}z^m = \frac{1}{2} + \frac{z}{1-z} =
  \frac{1-\rho^2+2i\rho \sin\alpha}{2(1-2\rho\cos\alpha+\rho^2)}.
\end{displaymath}
Since $z^m = \rho^m(\cos m\alpha + i\sin m\alpha)$,
we conclude upon separating the real and imaginary parts that
\begin{displaymath}
  \frac{1}{2} + \sum_{m=1}^{\infty}\rho^m\cos m\alpha =
  \frac{1-\rho^2}{2(1-2\rho\cos\alpha+\rho^2)}.
\end{displaymath}
Returning to \eqref{eq:7} using $\rho=r/R, \alpha=\theta-\phi$,
we obtain the Poisson formula
\begin{displaymath}
  u(r, \theta) = \frac{R^2-r^2}{2\pi}\int_0^{2\pi}\frac{f(\phi)}{R^2-2Rr\cos(\theta-\phi)+r^2}\dif\phi = \frac{R^2-\|\mathbf{x}\|^2}{2\pi R}\int_{\|\mathbf{y}\|=R}
    \frac{u(\mathbf{y})}{\|\mathbf{x}-\mathbf{y}\|^2}\dif S(\mathbf{y}).
\end{displaymath}
\end{sol}
%%% Local Variables:
%%% mode: latex
%%% TeX-master: "../hw2"
%%% End:
