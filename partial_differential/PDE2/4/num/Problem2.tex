\begin{pro}
  Let $g\in L^1(\mathbb{R}^n)$ with
  $\int_{\mathbb{R}^n}g\dif\mathbf{x} = 1$,
  then $g_{\epsilon}(\mathbf{x}) =
  \epsilon^{-n}g(\epsilon^{-1}\mathbf{x})$
  converges to $\delta$ as $\epsilon\to 0^+$,
  in $\mathcal{D}'(\mathbb{R}^n)$.
\end{pro}

\begin{proof}
  By the change of variable $\mathbf{x}\to\epsilon\mathbf{x}$
  we see that $\int_{\mathbb{R}^n}g_{\epsilon}(\mathbf{x})\dif\mathbf{x}=1$ for all $\epsilon>0$.
  Hence $\forall\phi\in\mathcal{C}_c^{\infty}(\mathbb{R}^n)$,
  \begin{align*}
    \left\langle g_{\epsilon}, \phi\right\rangle -
    \left\langle \delta, \phi\right\rangle &=
    \int_{\mathbb{R}^n}g_{\epsilon}(\mathbf{x})\phi(\mathbf{x})\dif\mathbf{x} - \phi(\mathbf{0})
                                              = \int_{\mathbb{R}^n}\epsilon^{-n}g(\epsilon^{-1}\mathbf{x})\phi(\mathbf{x})\dif\mathbf{x} - \int_{\mathbb{R}^n}g(\mathbf{x})\phi(\mathbf{0})\dif\mathbf{x} \\
                                           &= \int_{\mathbb{R}^n}g(\mathbf{x})\phi(\epsilon\mathbf{x})\dif\mathbf{x} - \int_{\mathbb{R}^n}g(\mathbf{x})\phi(\mathbf{0})\dif\mathbf{x} \\
    &= \int_{\mathbb{R}^n}g(\mathbf{x})\LP\phi(\epsilon\mathbf{x})-\phi(\mathbf{0})\RP\dif\mathbf{x} \to 0 \text{ as } \epsilon\to 0^+
  \end{align*}
  by the dominated convergence theorem.
  Therefore $g_{\epsilon}$ converges to $\delta$ as $\epsilon\to 0^+$, in
  $\mathcal{D}'(\mathbb{R}^n)$.
\end{proof}
%%% Local Variables:
%%% mode: latex
%%% TeX-master: "../hw4"
%%% End:
