\begin{pro}
  For 3D wave equation,
  perform a rigorous derivation of $E_3$ for $t>0$.
  Hint: As for Laplace operator,
  given $t>0$, we have
  \begin{displaymath}
    \hat{K}_{\delta} = \hat{E}_3e^{-\delta|\xi|^2} \to \hat{E}_3 \text{ in } \mathcal{S}'(\mathbb{R}^3),
  \end{displaymath}
  when $\delta>0$ tends to zero.
\end{pro}
\begin{sol}
  \begin{lem}
    Define $f_{\nu}(x)=\frac{1}{\pi}\frac{\sin\nu x}{x}$,
    then $f_{\nu}\to\delta$ as $\nu\to\infty$.
  \end{lem}
  \begin{proof}[Proof of Lemma]
    In fact,
    for $\forall \varphi\in C_c^{\infty}$,
    we have from Riemann-Lebesgue lemma
    \begin{displaymath}
      \langle f_{\nu}, \varphi\rangle = \frac{1}{\pi}\int_{-\infty}^{\infty}\frac{\sin\nu x}{x}\varphi(x)\dif x \to \varphi(0) = \langle \delta, \varphi\rangle.
    \end{displaymath}
  \end{proof}
  From the lecture notes,
  we know that
  \begin{displaymath}
    \hat{E}_3(t, \xi) = \frac{\sin t|\xi|}{|\xi|}.
  \end{displaymath}
  We need to apply the inverse Fourier transformation to get $E_3$:
  \begin{align*}
    E_3(t, x_1, x_2, x_3) &= \frac{1}{(2\pi)^3}\int_{\mathbb{R}^3}\frac{\sin t|\xi|}{|\xi|}
                            e^{i(\xi_1x_1+\xi_2x_2+\xi_3x_3)}\dif \xi_1\dif\xi_2\dif\xi_3 \\
    &= \frac{1}{(2\pi)^3}\int_0^{\infty}\int_{\mathbb{S}^2}\frac{\sin t\rho}{\rho}e^{i\xi\cdot \mathbf{x}}\rho^2\dif\omega\dif \rho,
  \end{align*}
  establish spherical coordinates $(\theta, \varphi)$ on the sphere
  with $x$-direction as the north direction,
  then we have
  \begin{displaymath}
    \xi\cdot\mathbf{x} = \rho r\cos\theta, \quad r = \|\mathbf{x}\| = \sqrt{x_1^2+x_2^2+x_3^2}, \quad
    \dif\omega=\sin\theta\dif\theta\dif\varphi.
  \end{displaymath}
  Thus
  \begin{align*}
    E(t, x_1, x_2, x_3) &= \frac{1}{(2\pi)^3}\int_0^{\infty}\int_0^{2\pi}\int_0^{\pi}
                          \rho\sin \rho t e^{i\rho r\cos\theta}\sin\theta\dif\theta\dif\varphi\dif\rho \\
                        &= \frac{1}{(2\pi)^2}\int_0^{\infty}\sin\rho t\LP\int_0^{\pi}e^{i\rho r\cos\theta}\rho\sin\theta\dif\theta\RP\dif\rho \\
                        &= \frac{1}{4\pi^2r}\int_0^{\infty}2\sin\rho t\cdot\sin\rho r\dif\rho \\
                        &= \frac{1}{4\pi^2r}\lim_{A\to\infty}\int_0^A[\cos\rho(r-t)-\cos\rho(r+t)]dif\rho \\
    &= \frac{1}{4\pi^2r}\lim_{A\to\infty}\LP \frac{\sin A(r-t)}{r-t} - \frac{\sin A(r+t)}{r+t}\RP.
  \end{align*}
  From the above lemma,
  we know that $\frac{1}{\pi}\frac{\sin\nu x}{x}$ converges to $\delta(x)$ as $\nu\to\infty$.
  Therefore
  \begin{displaymath}
    E(t, x_1, x_2, x_3) = \frac{1}{4\pi r}[\delta(r-t)-\delta(r+t)].
  \end{displaymath}
  Since $r+t>0$, we have $\delta(r+t)\equiv 0$.
  Hence
  \begin{displaymath}
    E(t, x_1, x_2, x_3) = \frac{1}{4\pi r}\delta(r-t).
  \end{displaymath}
\end{sol}
%%% Local Variables:
%%% mode: latex
%%% TeX-master: "../hw5"
%%% End:
