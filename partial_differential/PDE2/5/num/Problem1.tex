\begin{pro}
  Consider the operator $L=\partial_x+i\partial_y$ in $\mathbb{R}^2$,
  is it hypoelliptic?
  Explain in details.
\end{pro}
\begin{sol}
  The Cauchy-Riemann operator $L=\partial_x+i\partial_y$ is hypoelliptic.

  Consider $L$ on $\Omega\subset\mathbb{R}^2$,
  we know from complex analysis that solutions of $Lu=0$
  are homomorphic functions of $z=x+iy$ on $\Omega$.
  Therefore $L$ is hypoelliptic by definition.
 %  From Theorem 4 in the Lecture notes,
 %  we know that $L$ is hypoelliptic if
 %  \begin{equation}
 %    \label{eq:1}
 %    \exists K\in\mathcal{D}', LK = \delta \Rightarrow
 %    K\in C^{\infty}(\mathbb{R}^n\backslash\{\mathbf{0}\}).
 %  \end{equation}
 %  From Lecture notes \#4 \S 4.1,
 %  \pageref{eq:1} is a true statement.
 %  Therefore the Cauchy-Riemann operator is hypoelliptic.

 %  The following is a reiteration of \S 4.1 of the lecture notes:

 %  For the Cauchy-Riemann operator $L=\partial_x+i\partial_y$,
 %  to find a $K\in\mathcal{S}'$ with $LK=\delta$,
 %  we use Fourier tranform on $y$,
 %  to obtain
 %  \begin{displaymath}
 %    (\partial_x-\eta)\hat{K}(x, \eta) = \delta(x).
 %  \end{displaymath}
 %  Multiplying the integrating factor $e^{-x\eta}$,
 %  we obtain
 %  \begin{displaymath}
 %    \partial_x(e^{-x\eta}\hat{K}(x, \eta)) = \delta(x)e^{-x\eta} = \delta(x).
 %  \end{displaymath}
 %  View as an ODE in $x$,
 %  we get,
 %  for any fixed $\eta$,
 %  \begin{displaymath}
 %    e^{-x\eta}\hat{K}(x, \eta) = H(x) + C.
 %  \end{displaymath}
 %  But it is a PDE,
 %  we have in general
 %  \begin{displaymath}
 %    e^{-x\eta}\hat{K}(x, \eta) = H(x) + C(\eta), \quad
 %    \hat{K}(x, \eta) = (H(x) + C(\eta))e^{x\eta}.
 %  \end{displaymath}
 %  As we have seen,
 %  it is not in $\mathcal{S}'$ in general,
 %  because of the factor $e^{x\eta}$ in the case of $x\eta>0$.

 %  To avoid the bad behavior,
 %  for $x>0$,
 %  we want to set for $\eta>0, 1+C(\eta)=0$;
 %  while for $x<0$,
 %  we want to set for $\eta<0, 0+C(\eta)=0$.
 %  In conclusion,
 %  we choose to set $C(\eta)=-H(\eta)$ for $\eta\neq 0$.
 %  In general,
 %  we could set $C(\eta)=-H(\eta)+c\delta$.
 %  Here for simplicity,
 %  let us set $C(\eta)=-H(\eta)$.

 %  With this choice,
 %  we see that $\hat{K}\in L^1$ for fixed $x\neq 0$,
 %  and we can find $2\pi K$ by (classical) Fourier inversion:
 %  \begin{displaymath}
 %    \int(H(x)-H(\eta))e^{x\eta}e^{iy\eta}\dif\eta = \int_{-\infty}^0H(x)e^{x\eta}e^{iy\eta}\dif\eta
 %    + \int_0^{\infty}(H(x)-1)e^{x\eta}e^{iy\eta}\dif\eta = \frac{1}{x+iy},
 %  \end{displaymath}
 %  $K=\mathcal{O}(1/r)\in L_{\text{loc}}^1\cap \mathcal{S}'$.

 %  Let us check that $LK=\delta$.
 %  Actually,
 %  in polar coordinates $(x,y)=r(\cos\theta, \sin\theta)$,
 %  we have
 %  \begin{displaymath}
 %    r\partial_r = x\partial_x + y\partial_y, \partial_{\theta} = -y\partial_x+x\partial_y,
 %  \end{displaymath}
 %  \begin{displaymath}
 %    \partial_x = \cos\theta\partial_r-\frac{\sin\theta}{r}\partial_{\theta},
 %    \partial_y = \sin\theta\partial_r+\frac{\cos\theta}{r}\partial_{\theta},
 %  \end{displaymath}
 %  \begin{displaymath}
 %    \partial_x+i\partial_y = e^{i\theta}(\partial_r-\frac{i}{r}\partial_{\theta}),
 %  \end{displaymath}
 %  \begin{displaymath}
 %    \langle LK, \phi\rangle = -\langle K, L\phi\rangle = -\int_0^{2\pi}\int_0^{\infty}
 %    (2\pi re^{i\theta})^{-1}e^{i\theta}[(\partial_r-\frac{i}{r}\partial_{\theta})\phi]r\dif r\dif\theta
 %   = \phi(0),
 % \end{displaymath}
 % for any $\phi\in\mathcal{D}(\mathbb{R}^2)$.
\end{sol}
%%% Local Variables:
%%% mode: latex
%%% TeX-master: "../hw5"
%%% End:
