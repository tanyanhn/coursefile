\begin{pro}
  Consider the heat operator $L=\partial_t-\Delta$,
  is it hypoelliptic?
  Explain in details.
\end{pro}
\begin{sol}
  The heat operator $L=\partial_t-\Delta$ is hypoelliptic.

  Recall the regularity of solutions of the heat equation:
  \begin{thm}[Smoothness]
    Suppose $u\in C_1^2(U_T)$ solves the heat equation in $U_T$.
    Then
    \begin{displaymath}
      u\in C^{\infty}(U_T).
    \end{displaymath}
    This regularity assertion is valid even if $u$ attains nonsmooth boundary values on
    $\Gamma_T$.
  \end{thm}
  Hence the heat operator $L$ is hypoelliptic by definition.
\end{sol}
%%% Local Variables:
%%% mode: latex
%%% TeX-master: "../hw5"
%%% End:
