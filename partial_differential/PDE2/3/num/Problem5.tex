\begin{lem}[Gaussian function]
  If $G_{\lambda}=e^{-\lambda\|\mathbf{x}\|^2}$,
  where $\mathscr{R}\lambda>0$,
  then
  \begin{equation}
    \label{eq:7}
    \widehat{G}_{\lambda}(\xi) = \LP \frac{\pi}{\lambda}\RP^{n/2}
    e^{-\frac{\|\xi\|^2}{4\lambda}} = \LP \frac{\pi}{\lambda}\RP^{n/2}
    G_{1/(4\lambda)}.
  \end{equation}
\end{lem}

\begin{pro}
  Based on \eqref{eq:7} with $\lambda=\epsilon-it$,
  $\epsilon>0$,
  $t\in\mathbb{R}\backslash\{0\}$.
  By considering the limit in $\mathcal{S}'(\mathbb{R})$
  as $\epsilon\to 0^+$, deduce that
  \begin{equation}
    \mathcal{F}_{\mathbf{x}}e^{it\|\mathbf{x}\|^2}(\xi) =
    \LP \frac{\pi}{|t|}\RP^{n/2}e^{i \frac{n\pi}{4}\mathrm{sgn}t
    -\frac{i\|\xi\|^2}{4t}}.
  \end{equation}
\end{pro}

\begin{proof}
  Based on \eqref{eq:7} with $\lambda=\epsilon-it$,
  we obtain
  \begin{displaymath}
    \mathcal{F}e^{-(\epsilon-it)\|\mathbf{x}\|^2}(\xi) = \LP \frac{\pi}{\epsilon-it}\RP^{n/2}e^{-\frac{\|\xi\|^2}{4(\epsilon-it)}},
  \end{displaymath}
  considering the limit in $\mathcal{S}'(\mathbb{R})$ as $\epsilon\to 0^+$,
  we have
  \begin{align*}
    \mathcal{F}_{\mathbf{x}}e^{it\|\mathbf{x}\|^2}(\xi) &= \LP \frac{\pi}{-it}\RP^{n/2}e^{-\frac{i\|\xi\|^2}{4t}} \\
    &=
      \begin{cases}
        \LP \frac{\pi}{t}\RP^{n/2}i^{n/2}e^{-\frac{i\|\xi\|^2}{4t}} =
        \LP \frac{\pi}{|t|}\RP^{n/2}e^{i \frac{n\pi}{4}\mathrm{sgn}t-\frac{i\|\xi\|^2}{4t}} \text{ if } t>0, \\
        \LP \frac{\pi}{|t|}\RP^{n/2}(-i)^{n/2}e^{-\frac{i\|\xi\|^2}{4t}} =
        \LP \frac{\pi}{|t|}\RP^{n/2}e^{i \frac{n\pi}{4}\mathrm{sgn}t-\frac{i\|\xi\|^2}{4t}} \text{ if } t<0,
      \end{cases}
  \end{align*}
  where we have used Euler's identity $e^{ix}=\cos x+i\sin x$,
  in particular, $i = e^{i \frac{\pi}{2}}$ and $-i=e^{-i \frac{\pi}{2}}$.
  Therefore, we have the desired result.
\end{proof}
%%% Local Variables:
%%% mode: latex
%%% TeX-master: "../hw3"
%%% End:
