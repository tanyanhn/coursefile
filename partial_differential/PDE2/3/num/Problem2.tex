\begin{pro}
  Find a solution to the following Dirichlet problem for
  the Laplace equation,
  by using the Fourier transform:
  \begin{displaymath}
    \begin{cases}
      (\partial_x^2 + \partial_y^2)u = 0, (x, y)\in\mathbb{R}
      \times\mathbb{R}_+, \\
      u(x, 0) = f(x) \in \mathcal{S}(\mathbb{R}).
    \end{cases}
  \end{displaymath}
\end{pro}

\begin{sol}
  Take the Fourier transform,
  and we have
  \begin{displaymath}
    \begin{cases}
      \partial_y^2\hat{u}(\xi, y) + \xi^2\hat{u}(\xi, y) = 0, \\
      \hat{u}(\xi, 0) = \hat{f}(\xi).
    \end{cases}
  \end{displaymath}

  The general solution of this ordinary differential equation
  in $y$ (with $\xi$ fixed) takes the form
  \begin{displaymath}
    \widehat{u}(\xi, y) = A(\xi)e^{-|\xi|y} + B(\xi)e^{|\xi|y}.
  \end{displaymath}
  If we disregard the second term because of
  its rapid increase we find, after setting $y=0$, that
  \begin{displaymath}
    \widehat{u}(\xi, y) = \widehat{f}(\xi)e^{-|\xi|y}.
  \end{displaymath}
  Therefore $u$ is given in terms of the convolution of $f$
  with a kernel whose Fourier transform is $e^{-|\xi|y}$.

  \begin{lem}
    Define the Poisson kernel $\mathcal{P}_y(x)$ for the upper half-plane
    \begin{displaymath}
      \mathcal{P}_y(x) = \frac{2y}{x^2+y^2}
      \text{ where } x\in\mathbb{R} \text{ and } y>0.
    \end{displaymath}
    Then the following two identities hold:
    \begin{align*}
      \int_{-\infty}^{\infty}e^{-|\xi|y}e^{i\xi x}\dif\xi &= \mathcal{P}_y(x), \\
      \int_{-\infty}^{\infty}\mathcal{P}_y(x)e^{ix\xi}\dif x &=
                                                                     e^{-|\xi|y}.
    \end{align*}
  \end{lem}
  \begin{proof}[Proof of Lemma]
    The first formula is fairly straightforward since we can split
    the integral from $-\infty$ to $0$ and $0$ to $\infty$.
    Then, since $y>0$ we have
    \begin{displaymath}
      \int_0^{\infty}e^{-\xi y}e^{i\xi x}\dif \xi =
      \int_0^{\infty}e^{i(x+iy)\xi}\dif \xi =
      \left[\frac{e^{i(x+iy)\xi}}{i(x+iy)}\right]_0^{\infty} =
      -\frac{1}{i(x+iy)},
    \end{displaymath}
    and similarly,
    \begin{displaymath}
      \int_{-\infty}^0e^{\xi y}e^{i\xi x}\dif \xi = \frac{1}{i(x-iy)}.
    \end{displaymath}
    Therefore
    \begin{displaymath}
      \int_{-\infty}^{\infty}e^{-|\xi|y}e^{i\xi x}\dif \xi
      = \frac{1}{i(x-iy)} - \frac{1}{i(x+iy)}
      = \frac{2y}{x^2+y^2}.
    \end{displaymath}
    The second formula is now a consequence of the Fourier
    inversion theorem applied in the case when
    $f$ and $\hat{f}$ are of moderate decrease.
  \end{proof}
  Therefore
  \begin{displaymath}
    u(x, y) = (\mathcal{P}_y*f)(x) =
    \int_{\mathbb{R}}\mathcal{P}_y(x-z)f(z)\dif z
    = \int_{\mathbb{R}}\frac{2y}{(x-z)^2+y^2}f(z)\dif z.
  \end{displaymath}
%  where $\mathcal{P}_y(x)$ is the Poisson kernel for
%  the upper space.
\end{sol}

%%% Local Variables:
%%% mode: latex
%%% TeX-master: "../hw3"
%%% End:
