\begin{pro}
  Let $u\in\mathcal{S}'$,
  calculate
  $\mathcal{F}(\partial_ju)\in\mathcal{S}'(\mathbb{R}^n)$
  by definition.
\end{pro}

\begin{sol}
  $\forall \phi\in\mathcal{S}(\mathbb{R}^n)$,
  \begin{align*}
    \LI\mathcal{F}(\partial_ju), \phi\RI &= \LI\partial_ju, \mathcal{F}(\phi)\RI = -\LI u, \partial_j\mathcal{F}(\phi)\RI \\
                                           &= -\LI u, -i\mathcal{F}(\xi_j\phi)\RI
                                             = \LI \mathcal{F}(u), i\xi_j\phi\RI \\
    &= \LI i\xi_j\mathcal{F}(u), \phi\RI,
  \end{align*}
  where the first and third step follow from the definition
  of the Fourier transform on $\mathcal{S}'(\mathbb{R}^n)$,
  the second step from the definition of derivatives of
  tempered distribution and the third step follows from
  the property of Fourier transforms. Therefore
  \begin{displaymath}
    \mathcal{F}(\partial_ju) = i\xi_j\mathcal{F}(u).
  \end{displaymath}
  % \begin{align*}
  %   (\mathcal{F}(\partial_ju))(\phi) &= \int_{\mathbb{R}^n}
  %   (\partial_ju)(\mathbf{x})\widehat{\phi}(\mathbf{x})\dif\mathbf{x}
  %   =
  %                                      \int_{\mathbb{R}^n}\widehat{(\partial_ju)}(\mathbf{x})\phi(\mathbf{x})\dif\mathbf{x}
  %   \\
  %   &=
  %     \int_{\mathbb{R}^n}i\xi_j\widehat{u}(\mathbf{x})\phi(\mathbf{x})\dif\mathbf{x}
  %     = i\xi_j\int_{\mathbb{R}^n}u(\mathbf{x})\widehat{\phi}(\mathbf{x})\dif\mathbf{x} \\
  %   &= i\xi_j(\mathcal{F}(u))(\phi),
  % \end{align*}
  % where the first and the last step follow from the definition,
  % the second step from Plancherel theorem and
  % the third step from 
\end{sol}
%%% Local Variables:
%%% mode: latex
%%% TeX-master: "../hw3"
%%% End:
