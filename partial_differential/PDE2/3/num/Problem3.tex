\begin{pro}
  Check that any polynomial
  $p(\mathbf{x})\in\mathcal{S}'(\mathbb{R}^n)$,
  however,
  $f(x)=e^{x^2}\notin\mathcal{S}'(\mathbb{R})$,
  $g(x)=e^x\notin\mathcal{S}'(\mathbb{R})$.
  (Hint: you may want to use test functions like
  $e^{-\sqrt{1+x^2}}$.)
\end{pro}

\begin{sol}
  \begin{itemize}
  \item
  For a polynomial $p(\mathbf{x})$,
  define
  \begin{displaymath}
    L(\phi) = \int_{\mathbb{R}^n}p(\mathbf{x})\phi(\mathbf{x})\dif\mathbf{x}, \quad \phi\in\mathcal{S}.
  \end{displaymath}
  Now we show that $L(\phi)$
  is a continuous linear functional on the Schwartz
  space $\mathcal{S}$.
  Choose $N\in\mathbb{N}^+$ large enough so that
  \begin{displaymath}
    \int_{\mathbb{R}^n}\LP 1+\|\mathbf{x}\|^2\RP^{-N}
    |p(\mathbf{x})|\dif\mathbf{x} = C < \infty.
  \end{displaymath}
  Then
  \begin{align*}
    |L(\phi)| &= \left|\int_{\mathbb{R}^n}p(\mathbf{x})\phi(\mathbf{x})\dif\mathbf{x}\right| =
                \left|\int_{\mathbb{R}^n}\LP
                1+\|\mathbf{x}\|^2\RP^{-N}p(\mathbf{x})
                (1+\|\mathbf{x}\|^2)^N\phi(\mathbf{x})\dif\mathbf{x}\right| \\
    &\le \int_{\mathbb{R}^n}\LP 1+\|\mathbf{x}\|^2\RP^{-N}
      |p(\mathbf{x})|\dif\mathbf{x}
      \sup_{\mathbf{x}\in\mathbb{R}^n}\left\|(1+\|\mathbf{x}\|^2)^N\phi(\mathbf{x})\right\| \\
    &= C\sup_{\mathbf{x}\in\mathbb{R}^n}\left\|(1+\|\mathbf{x}\|^2)^N\phi(\mathbf{x})\right\|,
  \end{align*}
  which shows the continuity of $L$,
  $L$ is easily seen to be a linear functional.
  Therefore,
  the polynomial $p(\mathbf{x})\in\mathcal{S}'(\mathbb{R}^n)$.

\item
  Choose a function $\psi\in\mathcal{C}_c^{\infty}(\mathbb{R})$ such that
  $\int_{\mathbb{R}}\psi(x)\dif x = 1$,
  let
  \begin{displaymath}
    \phi_j(x) = \frac{\psi(x-j)}{f(x)} =
    e^{-x^2}\psi(x-j).
  \end{displaymath}
  It is easily verified that $\phi_j\to 0$ in
  $\mathcal{S}(\mathbb{R})$ as $j\to\infty$,
  but
  \begin{displaymath}
    \int_{\mathbb{R}}f(x)\phi_j(x)\dif x =
    \int_{\mathbb{R}}\psi(x)\dif x = 1
  \end{displaymath}
  for all $j$. Therefore $f(x)=e^{x^2}\notin
  \mathcal{S}'(\mathbb{R})$.

\item
  Similarly, we can show that $g(x)=e^x\notin\mathcal{S}'(\mathbb{R})$.
\end{itemize}
\end{sol}
%%% Local Variables:
%%% mode: latex
%%% TeX-master: "../hw3"
%%% End:
