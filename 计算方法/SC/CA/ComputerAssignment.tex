\documentclass[10pt,a4paper]{article}
\setlength{\paperheight}{29.7cm}
\setlength{\textheight}{25cm}

\usepackage{enumerate}
\usepackage{geometry}
\usepackage{CJKutf8}
\usepackage{amsfonts}
\usepackage{amsmath}
\usepackage{amssymb}
\usepackage{amsthm}
% \usepackage{CJKutf8}   % for Chinese characters


\usepackage{graphicx}  % for figures
\usepackage{layout}
\usepackage{multicol}  % multiple columns to reduce number of pages
\usepackage{mathrsfs}  
\usepackage{fancyhdr}
\usepackage{subfigure}
\usepackage{tcolorbox}
\usepackage{tikz-cd}

\usepackage{mathtools}
\usepackage{float}
\usepackage{bm}
\usepackage{booktabs}

\geometry{margin=2.5cm, vmargin={2cm,2cm}}
\setlength\parindent{0pt}

\newtheorem{pro}{Problem}
\newtheorem*{defn}{Definition}
\newtheorem*{thm}{Theorem}

\newcommand{\difFrac}[2]{\frac{\dif #1}{\dif #2}}
\newcommand{\pdfFrac}[2]{\frac{\partial #1}{\partial #2}}

%commands to enter adjustable parentheses.
\newcommand{\LP}{\left(}
  \newcommand{\RP}{\right)}

%add spacing between integrand and differential.
\newcommand{\dif}{\,\mathrm{d}}

\usepackage{color}
\usepackage{listings}
\definecolor{mygreen}{rgb}{0,0.6,0}
\definecolor{mygray}{rgb}{0.5,0.5,0.5}
\definecolor{mymauve}{rgb}{0.58,0,0.82}
\lstset{ %
  backgroundcolor=\color{white},   % choose the background color; you must add \usepackage{color} or \usepackage{xcolor}
  basicstyle=\footnotesize,        % the size of the fonts that are used for the code
  breakatwhitespace=false,         % sets if automatic breaks should only happen at whitespace
  breaklines=true,                 % sets automatic line breaking
  captionpos=b,                    % sets the caption-position to bottom
  commentstyle=\color{mygreen},    % comment style
  deletekeywords={...},            % if you want to delete keywords from the given language
  escapeinside={\%*}{*)},          % if you want to add LaTeX within your code
  extendedchars=true,              % lets you use non-ASCII characters; for 8-bits encodings only, does not work with UTF-8
  %frame=single,                    % adds a frame around the code
  keepspaces=true,                 % keeps spaces in text, useful for keeping indentation of code (possibly needs columns=flexible)
  keywordstyle=\color{blue},       % keyword style
  language=Matlab,                 % the language of the code
  morekeywords={*,...},            % if you want to add more keywords to the set
  numbers=left,                    % where to put the line-numbers; possible values are (none, left, right)
  numbersep=5pt,                   % how far the line-numbers are from the code
  numberstyle=\tiny\color{mygray}, % the style that is used for the line-numbers
  rulecolor=\color{black},         % if not set, the frame-color may be changed on line-breaks within not-black text (e.g. comments (green here))
  showspaces=false,                % show spaces everywhere adding particular underscores; it overrides 'showstringspaces'
  showstringspaces=false,          % underline spaces within strings only
  showtabs=false,                  % show tabs within strings adding particular underscores
  stepnumber=1,                    % the step between two line-numbers. If it's 1, each line will be numbered
  stringstyle=\color{mymauve},     % string literal style
  tabsize=2,                       % sets default tabsize to 2 spaces
  %title=\lstname,                   % show the filename of files included with \lstinputlisting; also try caption instead of title
  flexiblecolumns=true
}


% Define a Solution environment like proof
\makeatletter
\newenvironment{sol}[1][\solname]{\par
  \pushQED{\qed}
  \normalfont \topsep6\p@\@plus6\p@\relax
  \trivlist
  \item[\hskip\labelsep
        \itshape
    #1\@addpunct{.}]\ignorespaces
}{\popQED\endtrivlist\@endpefalse}
\providecommand{\solname}{Solution}
\makeatother

% Define the average integration notation: \bbint
\newcommand\tbbint{-\mkern -16mu\int}
\newcommand\dbbint{-\mkern -19mu\int}
\newcommand\bbint{
  {\mathchoice{\dbbint}{\tbbint}{\tbbint}{\tbbint}}
}


\title{Scientific Computing Homework: Computer Problems}
\author{\begin{CJK*}{UTF8}{gkai}
    李阳
    \end{CJK*}\, 11935018}

\begin{document}

\maketitle

\renewcommand\theenumi{\roman{enumi}}
\renewcommand\labelenumi{(\theenumi)}

\begin{pro}
  What is the inverse of the following matrix?
  \begin{displaymath}
    A =
    \begin{bmatrix}
      1 & 0 & 0\\
      1 & -1 & 0 \\
      1 & -2 & 1
    \end{bmatrix}
  \end{displaymath}
\end{pro}

\begin{sol}
  \begin{displaymath}
    A^{-1} =
    \begin{bmatrix}
            1 & 0 & 0\\
      1 & -1 & 0 \\
      1 & -2 & 1
    \end{bmatrix}
  \end{displaymath}
\end{sol}
%%% Local Variables:
%%% mode: latex
%%% TeX-master: "../hw1"
%%% End:


\clearpage

\begin{pro}
  Prove that the following holds in the sense of tempered distributions $\mathcal{S}'(\mathbb{R}^2)$.
  \begin{displaymath}
    (|x|^{-\epsilon}-1)\epsilon^{-1} = [\exp(\epsilon\ln|x|^{-1})-1]
    \epsilon^{-1} \to \ln(|x|^{-1}).
  \end{displaymath}
\end{pro}

\begin{proof}
  We have
  \begin{displaymath}
    \forall x\in\mathbb{R}^2, \quad
    [\exp(\epsilon\ln|x|^{-1})-1]\epsilon^{-1} - \ln(|x|^{-1}) \to 0
    \text{ as } \epsilon\to 0^+.
  \end{displaymath}
  Therefore, by the dominated convergence therem,
  $\forall \phi\in\mathcal{S}(\mathbb{R}^2)$,
  \begin{align*}
    \left\langle [\exp(\epsilon\ln|x|^{-1})-1]\epsilon^{-1}, \phi\right\rangle -
    \left\langle \ln(|x|^{-1}), \phi\right\rangle =
    \int_{\mathbb{R}^2}\LP[\exp(\epsilon\ln|x|^{-1})-1]\epsilon^{-1}-
    \ln(|x|^{-1})\RP\phi(x)\dif x \to 0 \text{ as } \epsilon\to 0^+,
  \end{align*}
  which gives the desired result.
\end{proof}
%%% Local Variables:
%%% mode: latex
%%% TeX-master: "../hw4"
%%% End:


\clearpage

\begin{pro}
  Consider the heat operator $L=\partial_t-\Delta$,
  is it hypoelliptic?
  Explain in details.
\end{pro}
\begin{sol}
  The heat operator $L=\partial_t-\Delta$ is hypoelliptic.

  Recall the regularity of solutions of the heat equation:
  \begin{thm}[Smoothness]
    Suppose $u\in C_1^2(U_T)$ solves the heat equation in $U_T$.
    Then
    \begin{displaymath}
      u\in C^{\infty}(U_T).
    \end{displaymath}
    This regularity assertion is valid even if $u$ attains nonsmooth boundary values on
    $\Gamma_T$.
  \end{thm}
  Hence the heat operator $L$ is hypoelliptic by definition.
\end{sol}
%%% Local Variables:
%%% mode: latex
%%% TeX-master: "../hw5"
%%% End:


\clearpage

\begin{pro}
  Solve using characteristics:
  \begin{itemize}
  \item[(a) ]
    $u_t + xu_x = x, u(0, x) = x^2$.
  \item[(b) ]
    $u_t + uu_x = 0, u(0, x) = -x$.
  \end{itemize}
\end{pro}

\begin{sol}
  Recall that the characteristic equations of
a first-order nonlinear PDE are defined as follows:
\begin{defn}
  \begin{subequations}
    \label{eq:characteristics}
    \begin{align}
      \dot{\mathbf{p}}(s) &= -D_{\mathbf{x}}F(\mathbf{p}(s), z(s), \mathbf{x}(s)) -
      D_z F(\mathbf{p}(s), z(s), \mathbf{x}(s))\mathbf{p}(s); \\
      \dot{z}(s) &= D_{\mathbf{p}}F(\mathbf{p}(s), z(s), \mathbf{x}(s))\cdot\mathbf{p}(s); \\
      \dot{\mathbf{x}}(s) &= D_{\mathbf{p}}F(\mathbf{p}(s), z(s), \mathbf{x}(s)).
    \end{align}
  \end{subequations}
\end{defn}

\begin{itemize}
\item[(a) ]
  Substituting
  \begin{displaymath}
    F(p_1, p_2; z; x, t) = xp_1 + p_2 - x
  \end{displaymath}
  into \eqref{eq:characteristics} yields
  \begin{align*}
    \dot{x}(s) &= x(s), \quad \dot{t}(s) = 1; \\
    \dot{z}(s) &= x(s).
  \end{align*}
  Consequently
  \begin{align*}
    x(s) &= x^0e^s, \quad t(s) = s; \\
    z(s) &= g(x^0) + \int_0^sx^0e^{\tau}\dif\tau = g(x^0) + x^0(e^s-1),
  \end{align*}
  where $x^0\in\mathbb{R}, s\geq 0$.

  Fix a point $(t, x)\in(0, \infty)\times\mathbb{R}$.
  We select $s>0$ and $x^0\in\mathbb{R}$ so that
  $(t, x) = (t(s), x(s)) = (s, x^0e^s)$;
  that is,
  $s = t, x^0 = xe^{-s}$.
  Then
  \begin{displaymath}
    u(t, x) = u(t(s), x(s)) = z(s) = g(x^0) + x^0(e^s - 1) = g(xe^{-t}) + xe^{-t}(e^t-1) = x^2e^{-2t} - xe^{-t} +x.
  \end{displaymath}
  Therefore
  \begin{equation}
    u(t, x) = x^2e^{-2t} - xe^{-t} + x.
  \end{equation}

  \item[(b) ]
  Substituting
  \begin{displaymath}
    F(p_1, p_2; z; x, t) = zp_1 + p_2
  \end{displaymath}
  into \eqref{eq:characteristics} yields
  \begin{align*}
    \dot{x}(s) &= z(s), \quad \dot{t}(s) = 1; \\
    \dot{z}(s) &= 0.
  \end{align*}
  Consequently
  \begin{align*}
    x(s) &= -x^0s + x^0, \quad t(s) = s; \\
    z(s) &= g(x^0) = -x^0,
  \end{align*}
  where $x^0\in\mathbb{R}, s\geq 0$.

  Fix a point $(t, x)\in(0, \infty)\times\mathbb{R}$.
  We select $s>0$ and $x^0\in\mathbb{R}$ so that
  $(t, x) = (t(s), x(s)) = (s, -x^0s+x^0)$;
  that is,
  $s = t, x^0 = \frac{x}{1-t}$.
  Then
  \begin{displaymath}
    u(t, x) = u(t(s), x(s)) = z(s) = g(x^0)= g\left(\frac{x}{1-t}\right) =
    -\frac{x}{1-t}.
  \end{displaymath}
  Therefore
  \begin{equation}
    u(t, x) = \frac{x}{t-1}.
  \end{equation}
\end{itemize}
\end{sol}
%%% Local Variables:
%%% mode: latex
%%% TeX-master: "../hw1"
%%% End:


\clearpage

\begin{pro}
  Prove the strengthened version of Theorem 8,
  for $(x, t)\in[0, L]^2$,
  with $f\in\mathcal{C}^2, g\in\mathcal{C}^1$ instead.
  [Hint: use the general solutions for wave equations]
\end{pro}

\begin{proof}
  Define $u(x, t)$ by d'Alembert's formula
  \begin{equation}
    u(x, t) = \frac{1}{2}[f(x+t)+f(x-t)] + \frac{1}{2}\int_{x-t}^{x+t}g(y)
    \dif y \quad (x\in\mathbb{R}, t \ge 0).
  \end{equation}

  It's quite trivial to verify that the following statements hold.
  \begin{itemize}
  \item[(i)]
    $u\in\mathcal{C}^2(\mathbb{R}\times[0, \infty])$;

  \item[(ii)]
    $u_{tt}-u_{xx}=0$ in $\mathbb{R}\infty(, \infty)$;

  \item[(iii)]
    $\lim_{(x, t)\to(x^0, 0), t>0}u(x, t) = f(x^0)$,
    $\lim_{{x,t}\to(x^0, 0), t>0}u_t(x, t) = g(x^0)$
    for each point $x^0\in\mathbb{R}$;

  \item[(iv)]
    The uniqueness of the solution follows from the energy methods
    as in Problem 3.
  \end{itemize}
  Combining the above completes the proof.
\end{proof}
%%% Local Variables:
%%% mode: latex
%%% TeX-master: "../hw2"
%%% End:


\clearpage

\begin{pro}
  Given the three data points $(-1, 1), (0, 0), (1, 1)$,
  determine the interpolating polynomial of degree two:
  \begin{itemize}
  \item[(a)]
    Using the monomial basis

  \item[(b)]
    Using the Lagrange basis

  \item[(c)]
    Using the Newton basis
  \end{itemize}
  Show that the three representations give the same polynomial.
\end{pro}
\begin{sol}
  \begin{itemize}
  \item[(a)]
    Solving the following system of linear equations
    \begin{displaymath}
      \begin{bmatrix}
        1 & -1 & 1 \\
        1 & 0 & 0 \\
        1 & 1 & 1
      \end{bmatrix}
      \begin{bmatrix}
        x_1 \\
        x_2 \\
        x_3
      \end{bmatrix}
      =
      \begin{bmatrix}
        1 \\
        0 \\
        1
      \end{bmatrix}
    \end{displaymath}
    yields $x_1 = 0, x_2 = 0, x_3 = 1$,
    so that the interpolating polynomial is
    \begin{displaymath}
      p_2(t) = t^2.
    \end{displaymath}

  \item[(b)]
    Apply Lagrange interpolation polynomial, and we have
    \begin{align*}
      p_2(t) &= 1\cdot \frac{(t-0)(t-1)}{(-1-0)(-1-1)}
      + 0\cdot \frac{(t+1)(t-1)}{(0+1)(0-1)}
               + 1\cdot \frac{(t+1)(t-0)}{(1+1)(1-0)} \\
             &= \frac{t(t-1)}{2} + \frac{t(t+1)}{2} \\
      &= t^2.
    \end{align*}

  \item[(c)]
    (a) From the table of divided differences
\begin{center}
\begin{tabular}{c|cccc}
$t$  & $y$ \\
\hline 
-1 & 1 \\
0 & 0 & -1 \\
1 & 1 & 1 & 1
\end{tabular}
\end{center}
one obtains by Newton's formula
\begin{displaymath}
  p_2(t) = 1 - (t+1) + (t+1)t = t^2.
\end{displaymath}

We can see that the above three representations give the same polynomial
\begin{displaymath}
  p_2(t) = t^2.
\end{displaymath}
  \end{itemize}
\end{sol}
%%% Local Variables:
%%% mode: latex
%%% TeX-master: "../hw3"
%%% End:


\end{document}
%%% Local Variables:
%%% mode: latex
%%% TeX-master: t
%%% End:
