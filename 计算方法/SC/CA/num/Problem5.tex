\textbf{4.3}
\begin{itemize}
\item[(a)]
  Implement inverse iteration with a shift to compute the eigenvalues nearest to 2,
  and a corresponding normalized eigenvector,
  of the matrix
  \begin{displaymath}
    A =
    \begin{bmatrix}
      6 & 2 & 1
      \\
      2 & 3 & 1
      \\
      1 & 1 & 1
    \end{bmatrix}.
  \end{displaymath}
  You may use an arbitrary starting vector.

\item[(b)]
  Use a real symmetric eigensystem library routine to compute all of the eigenvalues
  and eigenvectors of the matrix,
  and compare the results with those obtained in part (a).
\end{itemize}
  \begin{multicols}{2}
    \setlength{\columnseprule}{0.2pt}
    \begin{sol}
      \begin{itemize}
        \item[(a)]
  The code is shown as follows.
  \lstinputlisting[firstnumber=1]{matlab/InverseIter.m}
  The numerical result obtained is
\begin{verbatim}
The eigenvalue nearest to 2 is:
  2.1331e+00
a corresponding eigenvector is:

v =

    0.4974
   -0.8196
   -0.2843
\end{verbatim}

\item[(b)]
  We use the \verb|matlab| function \verb|eig| to compute
  the eigenvalue decomposition of $A$.
  \begin{displaymath}
    AV = VD,
  \end{displaymath}
  where the columns of $V$ are the eigenvectors of $A$
  and the diagonal entries of $D$ are the corresponding eigenvalues.
\begin{verbatim}
>> [V, D] = eig(A); diag(D)

ans =

    0.5789
    2.1331
    7.2880

>> V

V =

   -0.0432   -0.4974   -0.8664
   -0.3507    0.8196   -0.4531
    0.9355    0.2843   -0.2098
\end{verbatim}
  From the above two numerical results,
  we see that the two results agree with each other up to fourth digit.
\end{itemize}
\end{sol}
\end{multicols}
%%% Local Variables:
%%% mode: latex
%%% TeX-master: "../ComputerAssignment"
%%% End:
