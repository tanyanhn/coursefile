\textbf{5.9} In celestial mechanics,
\emph{Kepler's equation}
\begin{displaymath}
  M = E - e\sin(E)
\end{displaymath}
relates the mean anomaly $M$ to the eccentric anomaly $E$ of an elliptical orbit
of eccentricity $e$,
where $0<e<1$.
\begin{itemize}
\item[(a)]
  Prove that the fixed-point iteration using the iteration function
  \begin{displaymath}
    g(E) = M + e\sin(E)
  \end{displaymath}
  is locally convergent.

\item[(b)]
  Use the fixed-point iteration scheme in part (a)
  to solve Kepler's equation for the eccentric anomaly $E$
  corresponding to a mean anomaly of $M=1$ (radians)
  and an eccentricity of $e=0.5$.

\item[(c)]
  Use Newton's method to solve the same problem.

\item[(d)]
  Use a library zero finder to solve the same problem.
\end{itemize}

\begin{sol}
    \begin{multicols}{2}
      \setlength{\columnseprule}{0.2pt}
      \begin{itemize}
      \item[(a)]
        It sufficies to show that $g$ is locally a contractive mapping,
        which is true since
        \begin{displaymath}
          |g'(E)| = e|\cos(E)| \le e < 1.
        \end{displaymath}
        Therefore,
        the fixed-point iteration using the iteration function
        \begin{displaymath}
          g(E) = M + e\sin(E)
        \end{displaymath}
        is locally convergent.

      \item[(b)]
        The fixed-point iteration formula in this case is given by
        \begin{displaymath}
          E_{n+1} = M + e\sin(E_n) = 1 + \frac{\sin(E_n)}{2}
        \end{displaymath}
        \lstinputlisting[firstnumber=1]{matlab/fixedPoint.m}
      \end{itemize}
      The numerical result obtained is
\begin{verbatim}
E: 1.498701e+00
\end{verbatim}
    \item[(c)]
      Using Newton's method,
      we have
      \begin{align*}
        E_{n+1} &= E_n - \frac{E_n - e\sin(E_n)-M}{1-e\cos(E_n)} \\
        &= E_n - \frac{E_n-\sin(E_n)/2 -1}{1-\cos(E_n)/2}.
      \end{align*}
      \lstinputlisting[firstnumber=1]{matlab/Newton.m}
      The numerical result obtained is
\begin{verbatim}
E: 1.498702e+00
\end{verbatim}

    \item[(d)]
\begin{verbatim}
>> M = 1; e = 0.5;
>> E = fzero(@(x) M+e*sin(x)-x, 1)
\end{verbatim}
      The numerical result obtained is
\begin{verbatim}
E = 1.4987
\end{verbatim}
  \end{multicols}
\end{sol}
%%% Local Variables:
%%% mode: latex
%%% TeX-master: "../ComputerAssignment"
%%% End:
