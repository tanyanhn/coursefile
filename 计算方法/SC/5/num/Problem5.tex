\begin{pro}
  Implement the Total Variation Diminishing(TVD) scheme.
\end{pro}
\begin{sol}
    We have the following method for conservation law
  \begin{displaymath}
    Q_i^{n+1} = Q_i^n - \frac{\Delta t}{\Delta x}\left[\mathcal{F}(Q_i^n, Q_{i+1}^n) - \mathcal{F}(Q_{i-1}^n, Q_i^n)\right],
  \end{displaymath}
  where $\mathcal{F}(Q_i^n, Q_{i+1}^n) \approx F_{i+\frac{1}{2}}^n =
  h(Q_{i+\frac{1}{2}}^-, Q_{i+\frac{1}{2}}^+)$.
  For a TVD scheme,
  we require that the numerical flux function $h(\cdot,\cdot)$ satisfies
  \begin{itemize}
  \item
    Lipschitz continuous;
  \item
    monotone;
  \item
    $h(a,a)=a$.
  \end{itemize}
  Here we take
  \begin{displaymath}
    h(a,b) = 0.5(f(a)+f(b)-\alpha(b-a)), \text{ with } \alpha = \max_u|f'(u)|.
  \end{displaymath}
  For Burger's equation,
  \begin{displaymath}
    f(u) = \frac{u^2}{2}.
  \end{displaymath}
  Modify the given code \verb|convection_fvm.m|,
  we obtain the following main part of the code:
\begin{verbatim}
    h1 = 0.5*(0.5*un(i).^2+0.5*un(i+1).^2-norm(un,inf)*(un(i+1)-un(i)));
    h2 = 0.5*(0.5*un(i-1).^2+0.5*un(i).^2-norm(un,inf)*(un(i)-un(i-1)));
    u(i+1) = un(i) - (dt/dx)*(h1-h2);
\end{verbatim}
\end{sol}
%%% Local Variables:
%%% mode: latex
%%% TeX-master: "../hw5"
%%% End:
