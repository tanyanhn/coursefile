\begin{pro}
  Let $A$ be an $m\times n$ matrix and $\mathbf{b}$ an $m$-vector.
  \begin{itemize}
  \item[(a)]
    Prove that a solution to the least squares problem $A\mathbf{x}\cong\mathbf{b}$
    always exists.

  \item[(b)]
    Prove that such a solution is unique if, and only if,
    $\mathrm{rank}(A)=n$.
  \end{itemize}
\end{pro}

\begin{proof}
  \begin{itemize}

    \item[(a)]
  \begin{defn}
    A continuous function $f$ on an unbounded set $S\subset\mathbb{R}^n$
    is said to be \emph{coercive} if
    \begin{displaymath}
      \lim_{\|\mathbf{x}\|\to\infty}f(\mathbf{x}) = +\infty,
    \end{displaymath}
    i.e., for any constant $M$,
    there is an $r>0$ (depending on $M$) such that $f(\mathbf{x})\ge M$
    for any $\mathbf{x}\in S$ such that $\|\mathbf{x}\|\ge r$.
  \end{defn}

  \begin{lem}
    Let $S\in\mathbb{R}^n$ be closed and unbounded,
    if $f$ is coercive on $S$, then
    $f$ has a global minimum over $S$.
  \end{lem}
  \begin{proof}[Proof of Lemma]
    Without loss of generality,
    assume $\mathbf{0}\in S$.
    Since $f$ is coercive on $S$,
    we have
    \begin{equation}
      \label{eq:1}
      \exists r>0 \text{ s.t. } \forall \mathbf{x}\in S, \|\mathbf{x}\|> r, \quad
      f(\mathbf{x}) \ge f(\mathbf{0}).
    \end{equation}
    Consider the closed and bounded (hence compact) set
    $A = \{\mathbf{x}\in S: \|\mathbf{x}\|\le r\}$,
    we know from Calculus the fact that a continuous function on a compact
    set has both maximum and minimum,
    therefore
    \begin{equation}
      \label{eq:2}
      \exists \mathbf{x}^{*}\in A, \text{ s.t. }
      \forall \mathbf{x}\in A, f(\mathbf{x})\ge f(\mathbf{x}^{*}).
    \end{equation}
    Combining \eqref{eq:1} and \eqref{eq:2} completes the proof, i.e.,
    \begin{displaymath}
      \exists \mathbf{x}^{*}\in S, \text{ s.t. } \forall \mathbf{x}\in S,
      f(\mathbf{x})\ge f(\mathbf{x}^{*}).
    \end{displaymath}
  \end{proof}

  Consider the function $\phi:\mathbb{R}^n\to\mathbb{R}$ given by
  \begin{displaymath}
    \phi(\mathbf{y}) = \|\mathbf{b}-\mathbf{y}\|_2.
  \end{displaymath}

  $\phi$ is coercive on the closed and unbounded set $\mathrm{span}(A)$,
  applying the above lemma yields
  \begin{equation}
    \label{eq:3}
    \exists \mathbf{y}^{*} \text{ s.t. } \forall \mathbf{y}\in\mathrm{span}(A),
    \quad \phi(\mathbf{y}) \ge \phi(\mathbf{y}^{*}).
  \end{equation}
  Let $\mathbf{y}^{*} = A\mathbf{x}^{*} $,
  from \eqref{eq:3}, we see that $\mathbf{x}^{*}$ is a solution to the
  least squares problem $A\mathbf{x}\cong\mathbf{b}$, i.e.,
  \begin{displaymath}
    % \forall \mathbf{x}\in\mathbb{R}^n, \quad \|\mathbf{b}-A\mathbf{x}\|
    % \ge \|\mathbf{b}-A\mathbf{x}^{*}\|.
    \|\mathbf{b}-A\mathbf{x}^{*}\|_2 = \min_{\mathbf{x}\in\mathbb{R}^n}
    \|\mathbf{b}-A\mathbf{x}\|_2.
  \end{displaymath}

\item[(b)]
  Sufficiency: If $\mathrm{rank}(A)=n$,
  then $\forall \mathbf{y}\in\mathrm{span}(A)$,
  there exists a unique $\mathbf{x}\in \mathbb{R}^n$, s.t.
  $\mathbf{y}=A\mathbf{x}$.
  Therefore, the $\mathbf{x}^{*}$ constructed in the proof of (a) is unique.

  Necessity: if $\mathrm{rank}(A)<n$,
  then $\exists \mathbf{z}\in\mathbb{R}^n
  \text{ s.t. } A\mathbf{z}=\mathbf{0}$.
  Thus
  \begin{displaymath}
    \|\mathbf{b}-A(\mathbf{x}^{*}+\mathbf{z})\|_2 =
    \|\mathbf{b}-A\mathbf{x}^{*}\|_2 = \min_{\mathbf{x}\in\mathbb{R}^n}
    \|\mathbf{b}-A\mathbf{x}\|_2,
  \end{displaymath}
  which contradicts the uniqueness of $\mathbf{x}^{*}$.
\end{itemize}
\end{proof}
%%% Local Variables:
%%% mode: latex
%%% TeX-master: "../hw2"
%%% End:
