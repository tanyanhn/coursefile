\begin{pro}
  Suppose the $n\times n$ matrix $A$ has the block upper triangular form
  \begin{displaymath}
    A =
    \begin{bmatrix}
      A_{11} & A_{12} \\
      O & A_{22}
    \end{bmatrix},
  \end{displaymath}
  where $A_{11}$ is $k\times k$ and
  $A_{22}$ is $(n-k)\times(n-k)$.
  \begin{itemize}
  \item[(a)]
    If $\lambda$ is an eigenvalue of $A_{11}$ with corresponding
    eigenvector $\mathbf{u}$,
    show that $\lambda$ is an eigenvalue of $A$.
    (Hint: Find an $(n-k)-$vector $\mathbf{v}$ such that
    $
    \begin{bmatrix}
      \mathbf{u} \\ \mathbf{v}
    \end{bmatrix}
    $ is an eigenvector of $A$ corresponding to $\lambda$.)

  \item[(b)]
    If $\lambda$ is an eigenvalue of $A_{22}$ (but not of $A_{11}$)
    with corresponding eigenvector $\mathbf{v}$,
    show that $\lambda$ is an eigenvalue of $A$.
    (Hint: Find a $k$-vector $\mathbf{u}$ such that
    $
    \begin{bmatrix}
      \mathbf{u} \\
      \mathbf{v}
    \end{bmatrix}
    $ is an eigenvector of $A$ corresponding to $\lambda$.)

  \item[(c)]
    If $\lambda$ is an eigenvalue of $A$ with corresponding eigenvector
    $
    \begin{bmatrix}
      \mathbf{u} \\ \mathbf{v}
    \end{bmatrix}
    $, where $\mathbf{u}$ is a $k$-vector,
    show that $\lambda$ is either an eigenvalue of $A_{11}$ with
    corresponding eigenvector $\mathbf{u}$ or an eigenvalue of $A_{22}$ with
    corresponding eigenvector $\mathbf{v}$.

  \item[(d)]
    Combine the previous parts of this exercise to show that
    $\lambda$ is an eigenvalue of $A$ if, and only if,
    it is an eigenvalue of either $A_{11}$ or $A_{22}$.
  \end{itemize}
\end{pro}

\begin{proof}
  \begin{itemize}
  \item[(a)]
    \begin{displaymath}
      A
      \begin{bmatrix}
        \mathbf{u} \\
        \mathbf{0}
      \end{bmatrix}
      =
      \begin{bmatrix}
        A_{11} & A_{12} \\
        O & A_{22}
      \end{bmatrix}
      \begin{bmatrix}
        \mathbf{u} \\
        \mathbf{0} 
      \end{bmatrix}
      =
      \begin{bmatrix}
        A_{11}\mathbf{u} \\
        \mathbf{0}
      \end{bmatrix}
      = \lambda
      \begin{bmatrix}
        \mathbf{u} \\
        \mathbf{0}
      \end{bmatrix},
    \end{displaymath}
    where $\mathbf{0}$ is the zero vector of dimension $n-k$.

    Therefore $\lambda$ is an eigenvalue of $A$.

  \item[(b)]
    \begin{displaymath}
      A
      \begin{bmatrix}
        \mathbf{u} \\
        \mathbf{v}
      \end{bmatrix}
      =
      \begin{bmatrix}
        A_{11} & A_{12} \\
        O & A_{22}
      \end{bmatrix}
      \begin{bmatrix}
        \mathbf{u}  \\
        \mathbf{v}
      \end{bmatrix}
      =
      \begin{bmatrix}
        A_{11}\mathbf{u} + A_{12}\mathbf{v} \\
        A_{22}\mathbf{v}
      \end{bmatrix}
      =\lambda
      \begin{bmatrix}
        \mathbf{u} \\
        \mathbf{v}
      \end{bmatrix},
    \end{displaymath}
    We need to choose a specific $\mathbf{u}$
    so that the last equality holds.
    \begin{displaymath}
      A_{11}\mathbf{u} + A_{12}\mathbf{v} = \lambda\mathbf{u} \Rightarrow (A_{11}-\lambda I)\mathbf{u} = A_{12}\mathbf{v}
      \Rightarrow \mathbf{u} = (A_{11}-\lambda I)^{-1}A_{12}\mathbf{v},
    \end{displaymath}
    The invertibility of $A_{11}-\lambda I$ is guaranteed by the fact
    that $\lambda$ is not an eigenvalue of $A_{11}$.
    Therefore, $\lambda$ is an eigenvalue of $A$
    with corresponding eigenvector
    $
    \begin{bmatrix}
      (A_{11}-\lambda I)^{-1}A_{12}\mathbf{v} & \mathbf{v}
    \end{bmatrix}^T
    $.

  \item[(c)]
    Since $\lambda$ is an eigenvalue of $A$ with corresponding
    eigenvector $
    \begin{bmatrix}
      \mathbf{u} & \mathbf{v}
    \end{bmatrix}^T
    $, we have
        \begin{displaymath}
      A
      \begin{bmatrix}
        \mathbf{u} \\
        \mathbf{v}
      \end{bmatrix}
      =
      \begin{bmatrix}
        A_{11} & A_{12} \\
        O & A_{22}
      \end{bmatrix}
      \begin{bmatrix}
        \mathbf{u}  \\
        \mathbf{v}
      \end{bmatrix}
      =
      \begin{bmatrix}
        A_{11}\mathbf{u} + A_{12}\mathbf{v} \\
        A_{22}\mathbf{v}
      \end{bmatrix}
      =\lambda
      \begin{bmatrix}
        \mathbf{u} \\
        \mathbf{v}
      \end{bmatrix},
    \end{displaymath}
    If $\mathbf{v}\neq \mathbf{0}$,
    then $A_{22}\mathbf{v}=\lambda \mathbf{v}$,
    and thus $\lambda$ is an eigenvalue of $A_{22}$
    with corresponding eigenvector $\mathbf{v}$.

    If $\mathbf{v}=\mathbf{0}$,
    then $A_{11}\mathbf{u}=\lambda\mathbf{u}$,
    and hence $\lambda$ is an eigenvalue of $A_{11}$
    with corresponding eigenvector $\mathbf{u}$.

  \item[(d)]
    The sufficiency follows from (a) and (b)
    while the necessity follows from (c).
  \end{itemize}
\end{proof}
%%% Local Variables:
%%% mode: latex
%%% TeX-master: "../hw2"
%%% End:
