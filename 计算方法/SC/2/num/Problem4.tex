\begin{pro}
  Suppose you want to annihilate the second component of a vector
  \begin{displaymath}
    \mathbf{a} =
    \begin{bmatrix}
      a_1 \\ a_2
    \end{bmatrix}
  \end{displaymath}
  using a Givens rotation, but $a_1$ is already zero.
  \begin{itemize}
  \item[(a)]
    Is it still possible to annihilate $a_2$ with a Givens rotation?
    If so, specify an appropriate Givens rotation; if not, explain
    why.

  \item[(b)]
    Under these circumstances,
    can $a_2$ be annihilated with an elementary elimination matrix?
    If so, how? If not, why?
  \end{itemize}
\end{pro}

\begin{sol}
  \begin{itemize}
  \item[(a)]
    It is possible to annihilate $a_2$ with a Givens rotation
    \begin{displaymath}
      G =
      \begin{bmatrix}
        0 & 1 \\
        -1 & 0
      \end{bmatrix}.
    \end{displaymath}

  \item[(b)]
    We cannot annihilate $a_2$ with an elementary elimination matrix,
    since adding any scalar multiple of $a_1$ to $a_2$ does not change $a_2$.
  \end{itemize}
\end{sol}
%%% Local Variables:
%%% mode: latex
%%% TeX-master: "../hw2"
%%% End:
