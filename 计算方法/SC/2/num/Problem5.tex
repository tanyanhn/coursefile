\begin{pro}
  \begin{itemize}
  \item[(a)]
    In the Gram-Schmidt procedure of Section 3.5.3,
    if we define the orthogonal projectors $P_k = \mathbf{q}_k\mathbf{q}_k^T$,
    $k=1, \ldots, n$, where $\mathbf{q}_k$ is the $k$th column of $Q$ in the
    resulting QR factorization, show that
    \begin{displaymath}
      (I-P_k)(I-P_{k-1})\cdots(I-P_1) = I-P_k-P_{k-1}-\cdots-P_1.
    \end{displaymath}

  \item[(b)]
    Show that the classical Gram-Schmidt procedure is equivalent to
    \begin{displaymath}
      \mathbf{q}_k = (I-(P_1+\cdots+P_{k-1}))\mathbf{a}_k,
    \end{displaymath}

  \item[(c)]
    Show that the modified Gram-Schmidt procedure is equivalent to
    \begin{displaymath}
      \mathbf{q}_k = (I-P_{k-1})\cdots(I-P_1)\mathbf{a}_k.
    \end{displaymath}

  \item[(d)]
    An alternative way to stablize the classical procedure is to apply it more
    than once (i.e., iterative refinement),
    which is equivalent to taking
    \begin{displaymath}
      \mathbf{q}_k = (I-(P_1+\cdots+P_{k-1}))^m\mathbf{a}_k,
    \end{displaymath}
    where $m=2$ is typically sufficient.
    Show that all three of these variations are mathematically equivalent
    (thought they may differ markedly in finite-precision arithmetic).
  \end{itemize}
\end{pro}

\begin{sol}
  \begin{itemize}
  \item[(a)]
    First we compute $\forall i\neq j$,
    \begin{displaymath}
      P_iP_j = (\mathbf{q}_i\mathbf{q}_i^T)(\mathbf{q}_j\mathbf{q}_j^T)
      = \mathbf{q}_i(\mathbf{q}_i^T\mathbf{q}_j)\mathbf{q}_j^T
      = (\mathbf{q}_i^T\mathbf{q}_j)\mathbf{q}_i\mathbf{q}_j^T
      = 0(\mathbf{q}_i\mathbf{q}_j^T) = O,
    \end{displaymath}
    where the fourth equality holds since $\mathbf{q}_i$'s
    are the columns of an orthogonal matrix.

    Now we employ a simple induction on $k$.
    \begin{enumerate}
    \item
      For $k=1$, the conclusion clearly holds.

    \item
      Suppose the conclusion holds for some $k$,
      then for $k+1$, we have
      \begin{align*}
        (I-P_{k+1})(I-P_k)\cdots(I-P_1) &= (I-P_{k+1})(I-P_k-P_{k-1}-\cdots-P_1) \\
                                        &= I - P_{k+1}-P_k-\cdots-P_1 + P_{k+1}(P_k+\cdots+P_1) \\
        &= I - P_{k+1} - P_k -\cdots -P_1,
      \end{align*}
      therefore the conlusion holds for $k+1$ as well.
    \end{enumerate}

  \item[(b)]
    The definition of the classical Gram-Schmidt procedure yields
    \begin{align*}
      \mathbf{q}_k = \mathbf{a}_k - \sum_{j=1}^{k-1}(\mathbf{q}_j^T\mathbf{a}_k)\mathbf{q}_j = \mathbf{a}_k - \sum_{j=1}^{k-1}\mathbf{q}_j(\mathbf{q}_j^T\mathbf{a}_k) = \mathbf{a}_k - \sum_{j=1}^{k-1}(\mathbf{q}_j\mathbf{q}_j^T)\mathbf{a}_k = (I-(P_1+\cdots+P_{k-1}))\mathbf{a}_k.
    \end{align*}

  \item[(c)]
    The definition of the modified Gram-Schmidt procedure yields
    \begin{align*}
      \mathbf{a}_k &\leftarrow \mathbf{a}_k - \mathbf{q}_1^T\mathbf{a}_k\mathbf{q}_1 = (I-P_1)\mathbf{a}_k \\
      \mathbf{a}_k &\leftarrow \mathbf{a}_k - \mathbf{q}_2^T\mathbf{a}_k\mathbf{q}_2 = (I-P_2)\mathbf{a}_{k} \\
                   & \cdots \\
      \mathbf{a}_k &\leftarrow \mathbf{a}_k - \mathbf{q}_{k-1}^T\mathbf{a}_k\mathbf{q}_{k-1} = (I-P_{k-1})\mathbf{a}_k \\
      \mathbf{q}_k &\leftarrow \mathbf{a}_k
    \end{align*}
    Therefore
    \begin{displaymath}
      \mathbf{q}_k = (I-P_{k-1})\cdots(I-P_1)\mathbf{a}_k.
    \end{displaymath}

  \item[(d)]
    The equivalence of (b) and (c) follows directly from (a).
    To show the equivalence of (b) and (d),
    apply a mathematical induction on $m$ and
    use the following property of $P_k$:
    \begin{displaymath}
      P_iP_j =
      \begin{cases}
        P_i \text{ if } i = j; \\
        O \text{ if } i\neq j.
      \end{cases}
    \end{displaymath}
  \end{itemize}
\end{sol}
%%% Local Variables:
%%% mode: latex
%%% TeX-master: "../hw2"
%%% End:
