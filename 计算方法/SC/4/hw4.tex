\documentclass[10pt,a4paper]{article}
\setlength{\paperheight}{29.7cm}
\setlength{\textheight}{25cm}

\usepackage{enumerate}
\usepackage{geometry}
\usepackage{CJKutf8}
\usepackage{amsfonts}
\usepackage{amsmath}
\usepackage{amssymb}
\usepackage{amsthm}
% \usepackage{CJKutf8}   % for Chinese characters

\usepackage{algorithm}
\usepackage{algorithmic}
\usepackage{graphicx}  % for figures
\usepackage{layout}
\usepackage{multicol}  % multiple columns to reduce number of pages
\usepackage{mathrsfs}  
\usepackage{fancyhdr}
\usepackage{subfigure}
\usepackage{tcolorbox}
\usepackage{tikz-cd}

\usepackage{mathtools} % For multlined
\usepackage{float}
\usepackage{bm}
\usepackage{booktabs}

\geometry{margin=2.5cm, vmargin={2cm,2cm}}
\setlength\parindent{0pt}

\newtheorem{pro}{Problem}
\newtheorem*{defn}{Definition}
\newtheorem*{lem}{Lemma}
\newtheorem{prop}{Proposition}
\newtheorem*{thm}{Theorem}

\newcommand{\difFrac}[2]{\frac{\dif #1}{\dif #2}}
\newcommand{\pdfFrac}[2]{\frac{\partial #1}{\partial #2}}

%commands to enter adjustable parentheses.
\newcommand{\LP}{\left(}
  \newcommand{\RP}{\right)}

%add spacing between integrand and differential.
\newcommand{\dif}{\,\mathrm{d}}

\usepackage{color}
\usepackage{listings}
\definecolor{mygreen}{rgb}{0,0.6,0}
\definecolor{mygray}{rgb}{0.5,0.5,0.5}
\definecolor{mymauve}{rgb}{0.58,0,0.82}
\lstset{ %
  backgroundcolor=\color{white},   % choose the background color; you must add \usepackage{color} or \usepackage{xcolor}
  basicstyle=\footnotesize,        % the size of the fonts that are used for the code
  breakatwhitespace=false,         % sets if automatic breaks should only happen at whitespace
  breaklines=true,                 % sets automatic line breaking
  captionpos=b,                    % sets the caption-position to bottom
  commentstyle=\color{mygreen},    % comment style
  deletekeywords={...},            % if you want to delete keywords from the given language
  escapeinside={\%*}{*)},          % if you want to add LaTeX within your code
  extendedchars=true,              % lets you use non-ASCII characters; for 8-bits encodings only, does not work with UTF-8
  %frame=single,                    % adds a frame around the code
  keepspaces=true,                 % keeps spaces in text, useful for keeping indentation of code (possibly needs columns=flexible)
  keywordstyle=\color{blue},       % keyword style
  language=Matlab,                 % the language of the code
  morekeywords={*,...},            % if you want to add more keywords to the set
  numbers=left,                    % where to put the line-numbers; possible values are (none, left, right)
  numbersep=5pt,                   % how far the line-numbers are from the code
  numberstyle=\tiny\color{mygray}, % the style that is used for the line-numbers
  rulecolor=\color{black},         % if not set, the frame-color may be changed on line-breaks within not-black text (e.g. comments (green here))
  showspaces=false,                % show spaces everywhere adding particular underscores; it overrides 'showstringspaces'
  showstringspaces=false,          % underline spaces within strings only
  showtabs=false,                  % show tabs within strings adding particular underscores
  stepnumber=1,                    % the step between two line-numbers. If it's 1, each line will be numbered
  stringstyle=\color{mymauve},     % string literal style
  tabsize=2,                       % sets default tabsize to 2 spaces
  %title=\lstname,                   % show the filename of files included with \lstinputlisting; also try caption instead of title
  flexiblecolumns=true
}


% Define a Solution environment like proof
\makeatletter
\newenvironment{sol}[1][\solname]{\par
  \pushQED{\qed}
  \normalfont \topsep6\p@\@plus6\p@\relax
  \trivlist
  \item[\hskip\labelsep
        \itshape
    #1\@addpunct{.}]\ignorespaces
}{\popQED\endtrivlist\@endpefalse}
\providecommand{\solname}{Solution}
\makeatother

% Define the average integration notation: \bbint
\newcommand\tbbint{-\mkern -16mu\int}
\newcommand\dbbint{-\mkern -19mu\int}
\newcommand\bbint{
  {\mathchoice{\dbbint}{\tbbint}{\tbbint}{\tbbint}}
}


\title{Scientific Computing Homework \#4}
\author{\begin{CJK*}{UTF8}{gkai}
    李阳
    \end{CJK*}\, 11935018}

\begin{document}

\maketitle

\renewcommand\theenumi{\roman{enumi}}
\renewcommand\labelenumi{(\theenumi)}

\begin{pro}
  Set up the linear least squares system $A\mathbf{x}\cong \mathbf{b}$
  for fitting the model function $f(t, \mathbf{x}) = x_1t+x_2e^t$
  to the three data points $(1, 2), (2, 3), (3,5)$.
\end{pro}
\begin{sol}
  \begin{displaymath}
    A\mathbf{x} =
    \begin{bmatrix}
      1 & e \\
      2 & e^2 \\
      3 & e^3
    \end{bmatrix}
    \begin{bmatrix}
      x_1 \\
      x_2
    \end{bmatrix}
    \cong
    \begin{bmatrix}
      2 \\
      3 \\
      5
    \end{bmatrix}
    = \mathbf{b}.
  \end{displaymath}
\end{sol}
%%% Local Variables:
%%% mode: latex
%%% TeX-master: "../hw2"
%%% End:


\begin{pro}
  Let $g\in L^1(\mathbb{R}^n)$ with
  $\int_{\mathbb{R}^n}g\dif\mathbf{x} = 1$,
  then $g_{\epsilon}(\mathbf{x}) =
  \epsilon^{-n}g(\epsilon^{-1}\mathbf{x})$
  converges to $\delta$ as $\epsilon\to 0^+$,
  in $\mathcal{D}'(\mathbb{R}^n)$.
\end{pro}

\begin{proof}
  By the change of variable $\mathbf{x}\to\epsilon\mathbf{x}$
  we see that $\int_{\mathbb{R}^n}g_{\epsilon}(\mathbf{x})\dif\mathbf{x}=1$ for all $\epsilon>0$.
  Hence $\forall\phi\in\mathcal{C}_c^{\infty}(\mathbb{R}^n)$,
  \begin{align*}
    \left\langle g_{\epsilon}, \phi\right\rangle -
    \left\langle \delta, \phi\right\rangle &=
    \int_{\mathbb{R}^n}g_{\epsilon}(\mathbf{x})\phi(\mathbf{x})\dif\mathbf{x} - \phi(\mathbf{0})
                                              = \int_{\mathbb{R}^n}\epsilon^{-n}g(\epsilon^{-1}\mathbf{x})\phi(\mathbf{x})\dif\mathbf{x} - \int_{\mathbb{R}^n}g(\mathbf{x})\phi(\mathbf{0})\dif\mathbf{x} \\
                                           &= \int_{\mathbb{R}^n}g(\mathbf{x})\phi(\epsilon\mathbf{x})\dif\mathbf{x} - \int_{\mathbb{R}^n}g(\mathbf{x})\phi(\mathbf{0})\dif\mathbf{x} \\
    &= \int_{\mathbb{R}^n}g(\mathbf{x})\LP\phi(\epsilon\mathbf{x})-\phi(\mathbf{0})\RP\dif\mathbf{x} \to 0 \text{ as } \epsilon\to 0^+
  \end{align*}
  by the dominated convergence theorem.
  Therefore $g_{\epsilon}$ converges to $\delta$ as $\epsilon\to 0^+$, in
  $\mathcal{D}'(\mathbb{R}^n)$.
\end{proof}
%%% Local Variables:
%%% mode: latex
%%% TeX-master: "../hw4"
%%% End:


\begin{pro}
  Let $f = \ln|x|\in\mathcal{D}'(\mathbb{R})$,
  check that $f'\in\mathcal{D}'(\mathbb{R})$ is given by
  \begin{displaymath}
    \left\langle \mathrm{pv}\frac{1}{x}, \phi\right\rangle =
    \lim_{\epsilon\to 0^+}\int_{|x|\ge \epsilon}\frac{\phi(x)}{x}\dif x,
    \quad \forall \phi\in\mathcal{C}_c^{\infty}(\mathbb{R}).
  \end{displaymath}
\end{pro}

\begin{proof}
  \begin{align*}
    \int_{|x|\ge\epsilon}\frac{\phi(x)}{x}\dif x &= \int_{-\infty}^{-\epsilon}\frac{\phi(x)}{x}\dif x + \int_{\epsilon}^{\infty}\frac{\phi(x)}{x}\dif x \\
                                                 &= \phi(-\epsilon)\ln\epsilon - \int_{-\infty}^{-\epsilon}\phi'(x)\ln|x|\dif x - \phi(\epsilon)\ln\epsilon - \int_{\epsilon}^{\infty}\phi'(x)\ln|x|\dif x \\
    &= -(\phi(\epsilon)-\phi(-\epsilon))\ln\epsilon - \int_{|x|\ge\epsilon}\phi'(x)\ln|x|\dif x,
  \end{align*}
  since $(\phi(\epsilon)-\phi(-\epsilon))\ln\epsilon\to 0$
  (Taylor expansion),
  therefore
  \begin{displaymath}
    \lim_{\epsilon\to 0}\int_{|x|\ge\epsilon}\frac{\phi(x)}{x}\dif x =
    -\int_{-\infty}^{\infty}\phi'(x)\ln|x|\dif x,
  \end{displaymath}
  which completes the proof.
\end{proof}
%%% Local Variables:
%%% mode: latex
%%% TeX-master: "../hw4"
%%% End:


\begin{pro}
  Suppose you are using the secant method to find a root $x^{*}$ of a
  nonlinear equation $f(x)=0$.
  Show that if at any iteration it happens to be the case that either
  $x_k=x^{*}$ or $x_{k-1}=x^{*}$
  (but not both),
  then it will also be true that $x_{k+1}=x^{*}$.
\end{pro}
\begin{sol}
  \begin{itemize}
  \item
    If $x_k=x^{*}$, then
    \begin{displaymath}
      x_{k+1} = x_k - f(x_k)\frac{x_k-x_{k-1}}{f(x_k)-f(x_{k-1})}
      = x^{*} - f(x^{*})\frac{x^{*}-x_{k-1}}{f(x^{*})-f(x_{k-1})}
      = x^{*} - 0 = x^{*}.
    \end{displaymath}

  \item
    If $x_{k-1}=x^{*}$, then
    \begin{align*}
      x_{k+1} &= x_k - f(x_k)\frac{x_k-x_{k-1}}{f(x_k)-f(x_{k-1})}
                = x_k - f(x_k)\frac{x_k - x^{*}}{f(x_k)-f(x^{*})} \\
      &= x_k - f(x_k) \frac{x_k-x^{*}}{f(x_k)} = x_k - (x_k - x^{*})
      \\
      &= x^{*}.
    \end{align*}
  \end{itemize}
\end{sol}
%%% Local Variables:
%%% mode: latex
%%% TeX-master: "../hw3"
%%% End:


\begin{lem}[Gaussian function]
  If $G_{\lambda}=e^{-\lambda\|\mathbf{x}\|^2}$,
  where $\mathscr{R}\lambda>0$,
  then
  \begin{equation}
    \label{eq:7}
    \widehat{G}_{\lambda}(\xi) = \LP \frac{\pi}{\lambda}\RP^{n/2}
    e^{-\frac{\|\xi\|^2}{4\lambda}} = \LP \frac{\pi}{\lambda}\RP^{n/2}
    G_{1/(4\lambda)}.
  \end{equation}
\end{lem}

\begin{pro}
  Based on \eqref{eq:7} with $\lambda=\epsilon-it$,
  $\epsilon>0$,
  $t\in\mathbb{R}\backslash\{0\}$.
  By considering the limit in $\mathcal{S}'(\mathbb{R})$
  as $\epsilon\to 0^+$, deduce that
  \begin{equation}
    \mathcal{F}_{\mathbf{x}}e^{it\|\mathbf{x}\|^2}(\xi) =
    \LP \frac{\pi}{|t|}\RP^{n/2}e^{i \frac{n\pi}{4}\mathrm{sgn}t
    -\frac{i\|\xi\|^2}{4t}}.
  \end{equation}
\end{pro}

\begin{proof}
  Based on \eqref{eq:7} with $\lambda=\epsilon-it$,
  we obtain
  \begin{displaymath}
    \mathcal{F}e^{-(\epsilon-it)\|\mathbf{x}\|^2}(\xi) = \LP \frac{\pi}{\epsilon-it}\RP^{n/2}e^{-\frac{\|\xi\|^2}{4(\epsilon-it)}},
  \end{displaymath}
  considering the limit in $\mathcal{S}'(\mathbb{R})$ as $\epsilon\to 0^+$,
  we have
  \begin{align*}
    \mathcal{F}_{\mathbf{x}}e^{it\|\mathbf{x}\|^2}(\xi) &= \LP \frac{\pi}{-it}\RP^{n/2}e^{-\frac{i\|\xi\|^2}{4t}} \\
    &=
      \begin{cases}
        \LP \frac{\pi}{t}\RP^{n/2}i^{n/2}e^{-\frac{i\|\xi\|^2}{4t}} =
        \LP \frac{\pi}{|t|}\RP^{n/2}e^{i \frac{n\pi}{4}\mathrm{sgn}t-\frac{i\|\xi\|^2}{4t}} \text{ if } t>0, \\
        \LP \frac{\pi}{|t|}\RP^{n/2}(-i)^{n/2}e^{-\frac{i\|\xi\|^2}{4t}} =
        \LP \frac{\pi}{|t|}\RP^{n/2}e^{i \frac{n\pi}{4}\mathrm{sgn}t-\frac{i\|\xi\|^2}{4t}} \text{ if } t<0,
      \end{cases}
  \end{align*}
  where we have used Euler's identity $e^{ix}=\cos x+i\sin x$,
  in particular, $i = e^{i \frac{\pi}{2}}$ and $-i=e^{-i \frac{\pi}{2}}$.
  Therefore, we have the desired result.
\end{proof}
%%% Local Variables:
%%% mode: latex
%%% TeX-master: "../hw3"
%%% End:


\textbf{11.2}
\begin{itemize}
\item[(a)]
  Use the method of lines and an ODE solver of your choice to solve the heat equation
  \begin{displaymath}
    u_t = u_{xx}, \quad 0\le x \le 1, \quad t\ge 0,
  \end{displaymath}
  with initial condition
  \begin{displaymath}
    u(0, x) = \sin(\pi x), \quad 0\le x \le 1,
  \end{displaymath}
  and Dirichlet boundary conditions
  \begin{displaymath}
    u(t, 0) = 0, \quad u(t, 1) = 0, \quad t\ge 0.
  \end{displaymath}
  Integrate from $t=0$ to $t=0.1$.
  Plot the computed solution,
  preferably as a three-dimensional surface over the $(t, x)$ plane.
  If you do not have three-dimensional plotting capability,
  plot the solution as a function of $x$ for a few values of $t$,
  including the initial and final times.
  Determine the maximum error in the computed solution by
  comparing with the exact solution
  \begin{displaymath}
    u(t, x) = \exp(-\pi^2t)\sin(\pi x).
  \end{displaymath}
  Experiment with various spatial mesh sizes $\Delta x$,
  and try to characterize the error as a function of $\Delta x$.
  On a log-log scale,
  plot the maximum error as a function of $\Delta x$.

\item[(b)]
  Repeat part (a),
  but this time with initial condition
  \begin{displaymath}
    u(0, x) = \cos(\pi x), \quad 0\le x \le 1,
  \end{displaymath}
  and Neumann boundary conditions
  \begin{displaymath}
    u_x(t, 0) = 0, \quad u_x(t, 1) = 0, \quad t\ge 0,
  \end{displaymath}
  and compare with the exact solution.
\end{itemize}
\begin{itemize}
\begin{multicols}{2}
  \setlength{\columnseprule}{0.2pt}
  \item[(a)]
    Discretize the spatial domain
    \begin{displaymath}
      x_i = i\Delta x, \quad i=0, 1, \ldots, m+1,
    \end{displaymath}
    where the spatial mesh size $\Delta x=\frac{1}{m+1}$.
  Discretizing the heat equation with respect to spatial variables only,
  we obtain
  \begin{displaymath}
    \frac{\dif u_i(t)}{\dif t} = \frac{u_{i-1}(t)-2u_i(t)+u_{i+1}(t)}{(\Delta x)^2},
    \quad i=1, \ldots, m,
  \end{displaymath}
  and
  \begin{displaymath}
    u_i(0) = \sin(\pi i\Delta x), \quad i=1, \ldots, m,
  \end{displaymath}
  the Dirichlet boundary conditions are discretized into
  \begin{displaymath}
    u_0(t) = 0, \quad u_{m+1}(t) = 0, \quad t\ge 0.
  \end{displaymath}
  The code is shown as follows.

  \verb|func_mol.m|
  \lstinputlisting[firstnumber=1]{matlab/func_mol.m}

  \verb|mol.m|
  \lstinputlisting[firstnumber=1]{matlab/mol.m}
\end{multicols}

\begin{figure}[H]
  \centering
  \subfigure[$m = 10$]{
    \includegraphics[width=0.305\linewidth]{png/m10}
  }
  \hfill
  \subfigure[$m = 20$]{
    \includegraphics[width=0.305\linewidth]{png/m20}
  }
  \hfill
  \subfigure[$m = 40$]{
    \includegraphics[width=0.305\linewidth]{png/m40}
  }
  \hfill
  \subfigure[$m = 80$]{
    \includegraphics[width=0.305\linewidth]{png/m80}
  }
  \hfill
  \subfigure[$m = 160$]{
    \includegraphics[width=0.305\linewidth]{png/m160}
  }
  \hfill
  \subfigure[$m = 320$]{
    \includegraphics[width=0.305\linewidth]{png/m320}
  }
\end{figure}
  To characterize the error as a function of $\Delta x$,
  we modify the ODE solver used in \verb|mol.m| as follows.
\begin{verbatim}
options = odeset('RelTol',1e-9,'AbsTol',1e-12);
[t,u] = ode23s('func_mol',[t0, tfinal],y0,options);
\end{verbatim}

  On a log-log scale,
  the maximum error as a function of $\Delta x$ is as illustrated by the following figure.
  \begin{figure}[H]
    \centering
    \includegraphics[scale=0.55]{png/err.png}
  \end{figure}

\item[(b)]
  \begin{multicols}{2}
    \setlength{\columnseprule}{0.2pt}
  In this case,
  the discretization of the heat equation is
  \begin{align*}
    \frac{\dif u_0(t)}{\dif t} &= \frac{-2u_0(t)+2u_1(t)}{(\Delta x)^2}, \\
    \frac{\dif u_i(t)}{\dif t} &= \frac{u_{i-1}(t)-2u_i(t)+u_{i+1}(t)}{(\Delta x)^2}, i = 1,\ldots, m, \\
    \frac{\dif u_{m+1}(t)}{\dif t} &= \frac{2u_m(t)-2u_{m+1}(t)}{(\Delta x)^2},
  \end{align*}
  with initial condition
  \begin{displaymath}
    u_i(0) = \cos(\pi i\Delta x), \quad i=1, \ldots, m.
  \end{displaymath}
  The code is shown as follows.

    \verb|func_mol2.m|
  \lstinputlisting[firstnumber=1]{matlab/func_mol2.m}

  \verb|mol2.m|
  \lstinputlisting[firstnumber=1]{matlab/mol2.m}
\end{multicols}

\begin{figure}[H]
  \centering
  \subfigure[$m = 10$]{
    \includegraphics[width=0.305\linewidth]{png/2m10}
  }
  \hfill
  \subfigure[$m = 20$]{
    \includegraphics[width=0.305\linewidth]{png/2m20}
  }
  \hfill
  \subfigure[$m = 40$]{
    \includegraphics[width=0.305\linewidth]{png/2m40}
  }
  \hfill
  \subfigure[$m = 80$]{
    \includegraphics[width=0.305\linewidth]{png/2m80}
  }
  \hfill
  \subfigure[$m = 160$]{
    \includegraphics[width=0.305\linewidth]{png/2m160}
  }
  \hfill
  \subfigure[$m = 320$]{
    \includegraphics[width=0.305\linewidth]{png/2m320}
  }
\end{figure}
  To characterize the error as a function of $\Delta x$,
  we modify the ODE solver used in \verb|mol2.m| as follows.
\begin{verbatim}
options = odeset('RelTol',1e-9,'AbsTol',1e-12);
[t,u] = ode23s('func_mol',[t0, tfinal],y0,options);
\end{verbatim}

  On a log-log scale,
  the maximum error as a function of $\Delta x$ is as illustrated by the following figure.
  \begin{figure}[H]
    \centering
    \includegraphics[scale=0.55]{png/herr2.png}
  \end{figure}
\end{itemize}
%%% Local Variables:
%%% mode: latex
%%% TeX-master: "../ComputerAssignment"
%%% End:


\begin{pro}[Programming assignment]
  Consider a horizontal cantilevered beam that is
  clamped at one end but free along the remainder of its length.
  A discrete model of the forces on the beam yields a system of linear equations
  $A\mathbf{x} = \mathbf{b}$,
  where the $n\times n$ matrix $A$ has the banded form
  \begin{displaymath}
    \begin{bmatrix}
      9 & -4 & 1 & 0 & \cdots & \cdots & 0 \\
      -4 & 6 & -4 & 1 & \ddots & & \vdots \\
      1 & -4 & 6 & -4 & 1 & \ddots & \vdots \\
      0 & \ddots & \ddots & \ddots & \ddots & \ddots & 0 \\
      \vdots & \ddots & 1 & -4 & 6 & -4 & 1 \\
      \vdots &  & \ddots & 1 & -4 & 5 & -2 \\
      0 & \cdots & \cdots & 0 & 1 & -2 & 1
    \end{bmatrix},
  \end{displaymath}
  the $n$-vector $\mathbf{b}$ is the known load on the bar
  (including its own weight),
  and the $n$-vector $\mathbf{x}$ represents
  the resulting deflection of the bar that is to be determined.
  We will take the bar to be uniformly loaded,
  with $b_i = \frac{1}{n^4}$ for each component of the load vector.
  \begin{itemize}
  \item[(a)]
    Letting $n=100$, solve this linear system using
    both a standard library routine for dense linear systems
    and a library routine designed for banded (or more general sparse) systems.
    How do the two routines compare in the time required to compute the solution?
    How well do the answers obtained agree with each other?

  \item[(b)]
    Verify that the matrix $A$ has the UL factorization $A = RR^T$,
    where $R$ is an upper triangular matrix of the form
    \begin{displaymath}
      \begin{bmatrix}
        2 & -2 & 1 & 0 & \cdots & 0 \\
        0 & 1 & -2 & 1 & \ddots & \vdots \\
        \vdots & \ddots & \ddots & \ddots & \ddots & 0 \\
        \vdots & & \ddots & 1 & -2 & 1 \\
        \vdots & & & \ddots & 1 & -2 \\
        0 & \cdots & \cdots & \cdots & 0 & 1
      \end{bmatrix}.
    \end{displaymath}
    Letting $n=1000$,
    solve the linear system using this factorization
    (two triangular solves will be required).
    Also solve the system in its original form using a banded (or general sparse) system solver
    as in part (a).
    How well do the answers obtained agree with each other?
    Which approach seems more accurate?
    What is the condition number of $A$,
    and what accurary does it suggest that you should expect?
    Try iterative refinement to see if the accurary  or residual improves
    for the less accurate method.
  \end{itemize}
\end{pro}

\begin{sol}
    \begin{itemize}
    \item[(a)]
      The matlab routines are as follows.
      The time for the standard library routine for dense linear systems and
      the standard routines designed for banded systems we use
      are given approximately as follows.
\begin{verbatim}
Dense: 0.000226 seconds;
Sparse: 0.000163 seconds.
\end{verbatim}
      The answers obtained almost agree with each other, 
      since the $\infty$-norm of the difference between the two results we get is
\begin{verbatim}
6.01e-11.
\end{verbatim}
  \lstinputlisting[firstnumber=1]{matlab/beamDense.m}
  \lstinputlisting[firstnumber=1]{matlab/beamSparse.m}

  \item[(b)]
    Verifying that $A = RR^T$ is straightforward.

      The matlab routines are as follows.
      The $\infty$-norm of the difference between the two results we get is
\begin{verbatim}
1.59e-07.
\end{verbatim}
      The condition number of the matrix $A$ is
\begin{verbatim}
1.31e+08
\end{verbatim}
  \lstinputlisting[firstnumber=1]{matlab/triangularSolver.m}
  \end{itemize}
\end{sol}
%%% Local Variables:
%%% mode: latex
%%% TeX-master: "../hw1"
%%% End:


\begin{pro}
  If $\lambda$ is an eigenvalue of an $n\times n$ matrix $A$,
  show that $\lambda^2$ is an eigenvalue of $A^2$.
\end{pro}
\begin{proof}
  Let $\mathbf{x}$ be the eigenvector of $A$ corresponding to $\lambda$,
  i.e., $A\mathbf{x}=\lambda\mathbf{x}$,
  therefore
  \begin{displaymath}
    A^2\mathbf{x} = A(\lambda\mathbf{x}) = \lambda A\mathbf{x}
    = \lambda^2\mathbf{x},
  \end{displaymath}
  which shows that $\lambda^2$ is an eigenvalue of $A^2$
  with eigenvector $\mathbf{x}$.
\end{proof}
%%% Local Variables:
%%% mode: latex
%%% TeX-master: "../hw2"
%%% End:


\begin{pro}
  Prove that the formula using divided differences given in Section
  7.3.3,
  \begin{displaymath}
    x_j = f[t_1, t_2, \ldots, t_j],
  \end{displaymath}
  indeed gives the coefficient of the $j$th basis function
  in the Newton polynomial interpolant.
\end{pro}

\begin{proof}
  We utilize a mathematical induction on $j$.
  \begin{itemize}
  \item
    For $j=1$, the interpolating polynomial is
    \begin{displaymath}
      p_1(t) = f(t_1),
    \end{displaymath}
    and hence $x_1 = f(t_1) = f[t_1]$.

  \item
    Suppose the conclusion is true for all integers less than $n$,
    we show that it holds for $n+1$ as well.
    
    By our inductive hypothesis, we know that
    the polynomial interpolating
    \begin{displaymath}
      p_n(t_1) = f(t_1), \quad p_n(t_2) = f(t_2), \quad p_n(t_n) = f(t_n)
    \end{displaymath}
    is given by
    \begin{displaymath}
      p_n(t) = \sum_{i=1}^nx_i\prod_{j=1}^{i-1}(t-t_j)
      = \sum_{i=1}^nf[t_1, \ldots, t_i]\prod_{j=1}^{i-1}(t-t_j).
    \end{displaymath}
    And the polynomial interpolating
    \begin{displaymath}
      q_n(t_2) = f(t_2), \quad q_n(t_3) = f(t_3), \quad q_n(t_{n+1}) = f(t_{n+1})
    \end{displaymath}
    is given by
    \begin{displaymath}
      q_n(t) = \sum_{i=1}^nx_i\prod_{j=2}^i(t-t_j)
      = \sum_{i=1}^nf[t_2, \ldots, t_{i+1}]\prod_{j=2}^i(t-t_j).
    \end{displaymath}
    Therefore from the uniqueless of the interpolating polynomial,
    we know that the polynomial for interpolating
    \begin{displaymath}
      p_{n+1}(t_1) = f(t_1), \quad p_{n+1}(t_2) = f(t_2), \quad
      p_{n+1}(t_n) = f(t_n), \quad p_{n+1}(t_{n+1}) = f(t_{n+1})
    \end{displaymath}
    is given by
    \begin{displaymath}
      p_{n+1}(t) = \frac{t-t_{n+1}}{t_1-t_{n+1}}p_n(t)
      + \frac{t-t_1}{t_{n+1}-t_1}q_n(t)
    \end{displaymath}
    Comparing the coefficient of the highest-order term of the above
    two polynomials yields
    \begin{displaymath}
      x_{n+1} = \frac{f[t_2, t_3, \ldots, t_{n+1}]-f[t_1, t_2, \ldots,
        t_n]}{t_{n+1} - t_1} = f[t_1, t_2, \ldots, t_{n+1}],
    \end{displaymath}
    where the second equality follows from the definition of divided differences.
    Therefore we have shown that the conclusion holds for $n+1$,
    which completes the inductive proof.
  \end{itemize}
\end{proof}
%%% Local Variables:
%%% mode: latex
%%% TeX-master: "../hw3"
%%% End:


\begin{pro}
  Consider the two-point BVP
  \begin{displaymath}
    u'' = u^3 + t, \quad a < t < b,
  \end{displaymath}
  with boundary conditions
  \begin{displaymath}
    u(a) = \alpha, \quad u(b) = \beta.
  \end{displaymath}
  To use the shooting method to solve this problem,
  one needs a starting guess for the initial slope $u'(a)$.
  One way to obtain such a starting guess for the initial slope is,
  in effect,
  to do a ``preliminary shooting'' in which we take a single step
  of Euler's method with $h=b-a$.
  \begin{itemize}
  \item[(a)]
    Using this approach,
    write out the resulting algebraic equation for the initial slope.

  \item[(b)]
    What starting value for the inital slope results from this approach?
  \end{itemize}
\end{pro}

\begin{sol}
  \begin{itemize}
  \item[(a)]
    \begin{displaymath}
      u(b) = u(a) + hu'(a) \Rightarrow hu'(a) = u(b)-u(a).
    \end{displaymath}

  \item[(b)]
    \begin{displaymath}
      u'(a) = \frac{u(b)-u(a)}{h} = \frac{\beta-\alpha}{b-a}.
    \end{displaymath}
  \end{itemize}
\end{sol}
%%% Local Variables:
%%% mode: latex
%%% TeX-master: "../hw4"
%%% End:


\end{document}
%%% Local Variables:
%%% mode: latex
%%% TeX-master: t
%%% End:
