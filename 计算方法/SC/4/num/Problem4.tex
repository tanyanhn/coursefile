\begin{pro}
  Why is Monte Carlo \emph{not} a practical method for
  computing one-dimensional integrals?
\end{pro}

\begin{sol}
  The error goes to zero as $1/\sqrt{n}$,
  which means, for example,
  that to gain an additional decimal digit of accuracy
  the number of sample points must be increased by a factor of $100$.
  Therefore,
  for computing one-dimensional integrals,
  Monte Carlo method is so inefficient,
  which may require millions of evaluations of the integrand.
\end{sol}
%%% Local Variables:
%%% mode: latex
%%% TeX-master: "../hw4"
%%% End:
