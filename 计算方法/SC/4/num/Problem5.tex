\begin{pro}
  Explain how a quadrature rule can be used
  to solve an integral equation numerically.
  What type of computational problem results?
\end{pro}

\begin{sol}
  We approximate the integral equation
  \begin{displaymath}
    \int_a^bK(s, t)u(t)\dif t = f(s),
  \end{displaymath}
  by
  \begin{displaymath}
    \sum_{j=1}^nw_jK(s_i, t_j)u(t_j) = f(s_i), \quad i = 1, \ldots, n,
  \end{displaymath}
  where $t_j$ and $w_j$ ($j=1, \ldots,n$) are the nodes and
  weights of a quadrature rule.

  Now we can solve the above
  \emph{system of linear algebratic equations}
  $A\mathbf{x}=\mathbf{b}$,
  where $a_{ij} = w_jK(s_i, t_j), b_i=f(s_i)$,
  and $x_j=u(t_j)$,
  which can be solved for $\mathbf{x}$ to obtain
  a discrete sample of approximate values of the solution function $u$.

  As we have seen,
  the result is solving a system of linear algebraic equations.
\end{sol}
%%% Local Variables:
%%% mode: latex
%%% TeX-master: "../hw4"
%%% End:
