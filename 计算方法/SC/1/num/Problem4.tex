\begin{pro}
  If $A, B$, and $C$ are $n\times n$ matrices,
  with $B$ and $C$ nonsingular,
  and $\mathbf{b}$ an $n$-vector,
  how would you implement the formula
  \begin{displaymath}
    \mathbf{x} = B^{-1}(2A+I)(C^{-1}+A)\mathbf{b}
  \end{displaymath}
  without computing any matrix inverses?
\end{pro}

\begin{sol}
  The basic idea is that whenever we see a matrix inverse
  in a formula, we should think ``solve a system'' rather than
  ``invert a matrix.'' So we can implement the formula using the following steps.
  \begin{itemize}
  \item
    solve $C\mathbf{y} = b$ for $\mathbf{y}$;
  \item
    $\mathbf{y}\leftarrow \mathbf{y} + A\mathbf{b}$;
  \item
    $\mathbf{w} \leftarrow (2A+I)\mathbf{y}$;
  \item
    solve $B\mathbf{x} = \mathbf{w}$ for $\mathbf{x}$.
  \end{itemize}
\end{sol}
%%% Local Variables:
%%% mode: latex
%%% TeX-master: "../hw1"
%%% End:
