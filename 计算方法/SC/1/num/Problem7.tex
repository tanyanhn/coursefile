\begin{pro}[Programming assignment]
  Consider a horizontal cantilevered beam that is
  clamped at one end but free along the remainder of its length.
  A discrete model of the forces on the beam yields a system of linear equations
  $A\mathbf{x} = \mathbf{b}$,
  where the $n\times n$ matrix $A$ has the banded form
  \begin{displaymath}
    \begin{bmatrix}
      9 & -4 & 1 & 0 & \cdots & \cdots & 0 \\
      -4 & 6 & -4 & 1 & \ddots & & \vdots \\
      1 & -4 & 6 & -4 & 1 & \ddots & \vdots \\
      0 & \ddots & \ddots & \ddots & \ddots & \ddots & 0 \\
      \vdots & \ddots & 1 & -4 & 6 & -4 & 1 \\
      \vdots &  & \ddots & 1 & -4 & 5 & -2 \\
      0 & \cdots & \cdots & 0 & 1 & -2 & 1
    \end{bmatrix},
  \end{displaymath}
  the $n$-vector $\mathbf{b}$ is the known load on the bar
  (including its own weight),
  and the $n$-vector $\mathbf{x}$ represents
  the resulting deflection of the bar that is to be determined.
  We will take the bar to be uniformly loaded,
  with $b_i = \frac{1}{n^4}$ for each component of the load vector.
  \begin{itemize}
  \item[(a)]
    Letting $n=100$, solve this linear system using
    both a standard library routine for dense linear systems
    and a library routine designed for banded (or more general sparse) systems.
    How do the two routines compare in the time required to compute the solution?
    How well do the answers obtained agree with each other?

  \item[(b)]
    Verify that the matrix $A$ has the UL factorization $A = RR^T$,
    where $R$ is an upper triangular matrix of the form
    \begin{displaymath}
      \begin{bmatrix}
        2 & -2 & 1 & 0 & \cdots & 0 \\
        0 & 1 & -2 & 1 & \ddots & \vdots \\
        \vdots & \ddots & \ddots & \ddots & \ddots & 0 \\
        \vdots & & \ddots & 1 & -2 & 1 \\
        \vdots & & & \ddots & 1 & -2 \\
        0 & \cdots & \cdots & \cdots & 0 & 1
      \end{bmatrix}.
    \end{displaymath}
    Letting $n=1000$,
    solve the linear system using this factorization
    (two triangular solves will be required).
    Also solve the system in its original form using a banded (or general sparse) system solver
    as in part (a).
    How well do the answers obtained agree with each other?
    Which approach seems more accurate?
    What is the condition number of $A$,
    and what accurary does it suggest that you should expect?
    Try iterative refinement to see if the accurary  or residual improves
    for the less accurate method.
  \end{itemize}
\end{pro}

\begin{sol}
    \begin{itemize}
    \item[(a)]
      The matlab routines are as follows.
      The time for the standard library routine for dense linear systems and
      the standard routines designed for banded systems we use
      are given approximately as follows.
\begin{verbatim}
Dense: 0.000226 seconds;
Sparse: 0.000163 seconds.
\end{verbatim}
      The answers obtained almost agree with each other, 
      since the $\infty$-norm of the difference between the two results we get is
\begin{verbatim}
6.01e-11.
\end{verbatim}
  \lstinputlisting[firstnumber=1]{matlab/beamDense.m}
  \lstinputlisting[firstnumber=1]{matlab/beamSparse.m}

  \item[(b)]
    Verifying that $A = RR^T$ is straightforward.

      The matlab routines are as follows.
      The $\infty$-norm of the difference between the two results we get is
\begin{verbatim}
1.59e-07.
\end{verbatim}
      The condition number of the matrix $A$ is
\begin{verbatim}
1.31e+08
\end{verbatim}
  \lstinputlisting[firstnumber=1]{matlab/triangularSolver.m}
  \end{itemize}
\end{sol}
%%% Local Variables:
%%% mode: latex
%%% TeX-master: "../hw1"
%%% End:
