\begin{pro}
  Let
  \begin{displaymath}
    A =
    \begin{bmatrix}
      1 & 1+\epsilon \\
      1-\epsilon & 1
    \end{bmatrix}.
  \end{displaymath}
\end{pro}
\begin{itemize}
\item[(a)]
  What is the determinant of $A$?

\item[(b)]
  In floating-point arithmetic,
  for what range of values of $\epsilon$
  will the computed value of the determinant be zero?

\item[(c)]
  What is the LU factorization of $A$?

\item[(d)]
  In floating-point arithmetic,
  for what range of values of $\epsilon$
  will the computed value of $U$ be singular?
\end{itemize}

\begin{sol}
  \begin{itemize}
  \item[(a)]
    \begin{displaymath}
      \det(A) = 1 - (1+\epsilon)(1-\epsilon) = \epsilon^2.
    \end{displaymath}

  \item[(b)]
    For $\epsilon\in(-\sqrt{\epsilon_{\text{mach}}},
    \sqrt{\epsilon_{\text{mach}}})$,
    the computed value of the determinant will be zero,
    where $\epsilon_{\text{mach}}$ denotes the unit roundoff.

  \item[(c)]
    The LU factorization of A is as follows
    \begin{displaymath}
      L =
      \begin{bmatrix}
        1 & 0 \\
        1-\epsilon & 1
      \end{bmatrix},
      \quad
      U =
      \begin{bmatrix}
        1 & 1+\epsilon \\
        0 & \epsilon^2
      \end{bmatrix}.
    \end{displaymath}

  \item[(d)]
    For $\epsilon\in(-\sqrt{\epsilon_{\text{mach}}},
    \sqrt{\epsilon_{\text{mach}}})$,
    the computed value of $U$ will be singular,
    where $\epsilon_{\text{mach}}$ denotes the unit roundoff.
  \end{itemize}
\end{sol}
%%% Local Variables:
%%% mode: latex
%%% TeX-master: "../hw1"
%%% End:
