\begin{pro}
  Given the three data points $(-1, 1), (0, 0), (1, 1)$,
  determine the interpolating polynomial of degree two:
  \begin{itemize}
  \item[(a)]
    Using the monomial basis

  \item[(b)]
    Using the Lagrange basis

  \item[(c)]
    Using the Newton basis
  \end{itemize}
  Show that the three representations give the same polynomial.
\end{pro}
\begin{sol}
  \begin{itemize}
  \item[(a)]
    Solving the following system of linear equations
    \begin{displaymath}
      \begin{bmatrix}
        1 & -1 & 1 \\
        1 & 0 & 0 \\
        1 & 1 & 1
      \end{bmatrix}
      \begin{bmatrix}
        x_1 \\
        x_2 \\
        x_3
      \end{bmatrix}
      =
      \begin{bmatrix}
        1 \\
        0 \\
        1
      \end{bmatrix}
    \end{displaymath}
    yields $x_1 = 0, x_2 = 0, x_3 = 1$,
    so that the interpolating polynomial is
    \begin{displaymath}
      p_2(t) = t^2.
    \end{displaymath}

  \item[(b)]
    Apply Lagrange interpolation polynomial, and we have
    \begin{align*}
      p_2(t) &= 1\cdot \frac{(t-0)(t-1)}{(-1-0)(-1-1)}
      + 0\cdot \frac{(t+1)(t-1)}{(0+1)(0-1)}
               + 1\cdot \frac{(t+1)(t-0)}{(1+1)(1-0)} \\
             &= \frac{t(t-1)}{2} + \frac{t(t+1)}{2} \\
      &= t^2.
    \end{align*}

  \item[(c)]
    (a) From the table of divided differences
\begin{center}
\begin{tabular}{c|cccc}
$t$  & $y$ \\
\hline 
-1 & 1 \\
0 & 0 & -1 \\
1 & 1 & 1 & 1
\end{tabular}
\end{center}
one obtains by Newton's formula
\begin{displaymath}
  p_2(t) = 1 - (t+1) + (t+1)t = t^2.
\end{displaymath}

We can see that the above three representations give the same polynomial
\begin{displaymath}
  p_2(t) = t^2.
\end{displaymath}
  \end{itemize}
\end{sol}
%%% Local Variables:
%%% mode: latex
%%% TeX-master: "../hw3"
%%% End:
