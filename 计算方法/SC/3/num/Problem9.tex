\begin{pro}
  Prove that the formula using divided differences given in Section
  7.3.3,
  \begin{displaymath}
    x_j = f[t_1, t_2, \ldots, t_j],
  \end{displaymath}
  indeed gives the coefficient of the $j$th basis function
  in the Newton polynomial interpolant.
\end{pro}

\begin{proof}
  We utilize a mathematical induction on $j$.
  \begin{itemize}
  \item
    For $j=1$, the interpolating polynomial is
    \begin{displaymath}
      p_1(t) = f(t_1),
    \end{displaymath}
    and hence $x_1 = f(t_1) = f[t_1]$.

  \item
    Suppose the conclusion is true for all integers less than $n$,
    we show that it holds for $n+1$ as well.
    
    By our inductive hypothesis, we know that
    the polynomial interpolating
    \begin{displaymath}
      p_n(t_1) = f(t_1), \quad p_n(t_2) = f(t_2), \quad p_n(t_n) = f(t_n)
    \end{displaymath}
    is given by
    \begin{displaymath}
      p_n(t) = \sum_{i=1}^nx_i\prod_{j=1}^{i-1}(t-t_j)
      = \sum_{i=1}^nf[t_1, \ldots, t_i]\prod_{j=1}^{i-1}(t-t_j).
    \end{displaymath}
    And the polynomial interpolating
    \begin{displaymath}
      q_n(t_2) = f(t_2), \quad q_n(t_3) = f(t_3), \quad q_n(t_{n+1}) = f(t_{n+1})
    \end{displaymath}
    is given by
    \begin{displaymath}
      q_n(t) = \sum_{i=1}^nx_i\prod_{j=2}^i(t-t_j)
      = \sum_{i=1}^nf[t_2, \ldots, t_{i+1}]\prod_{j=2}^i(t-t_j).
    \end{displaymath}
    Therefore from the uniqueless of the interpolating polynomial,
    we know that the polynomial for interpolating
    \begin{displaymath}
      p_{n+1}(t_1) = f(t_1), \quad p_{n+1}(t_2) = f(t_2), \quad
      p_{n+1}(t_n) = f(t_n), \quad p_{n+1}(t_{n+1}) = f(t_{n+1})
    \end{displaymath}
    is given by
    \begin{displaymath}
      p_{n+1}(t) = \frac{t-t_{n+1}}{t_1-t_{n+1}}p_n(t)
      + \frac{t-t_1}{t_{n+1}-t_1}q_n(t)
    \end{displaymath}
    Comparing the coefficient of the highest-order term of the above
    two polynomials yields
    \begin{displaymath}
      x_{n+1} = \frac{f[t_2, t_3, \ldots, t_{n+1}]-f[t_1, t_2, \ldots,
        t_n]}{t_{n+1} - t_1} = f[t_1, t_2, \ldots, t_{n+1}],
    \end{displaymath}
    where the second equality follows from the definition of divided differences.
    Therefore we have shown that the conclusion holds for $n+1$,
    which completes the inductive proof.
  \end{itemize}
\end{proof}
%%% Local Variables:
%%% mode: latex
%%% TeX-master: "../hw3"
%%% End:
