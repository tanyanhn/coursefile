\begin{pro}
  Consider the nonlinear equation
  \begin{displaymath}
    f(x) = x^2 - 2 = 0.
  \end{displaymath}
  \begin{itemize}
  \item[(a)]
    With $x_0=1$ as a starting point,
    what is the value of $x_1$ if you use Newton's method for solving this problem?

  \item[(b)]
    With $x_0=1$ and $x_1=2$ as starting points,
    what is the value of $x_2$ if you use the secant method for the same problem?
  \end{itemize}
\end{pro}

\begin{sol}
  \begin{itemize}
  \item[(a)]
    \begin{displaymath}
      x_1 = x_0 - \frac{f(x_0)}{f'(x_0)} = 1 - \frac{1-2}{2} = 1.5.
    \end{displaymath}

  \item[(b)]
    \begin{displaymath}
      x_2 = x_1 - f(x_1)\frac{x_1-x_0}{f(x_1)-f(x_0)}
    = 2 - 2 \frac{2-1}{4-(-1)} = 1.6.
    \end{displaymath}
  \end{itemize}
\end{sol}
%%% Local Variables:
%%% mode: latex
%%% TeX-master: "../hw3"
%%% End:
