\documentclass[a4paper]{book}

\usepackage{geometry}
% make full use of A4 papers
\geometry{margin=1.5cm, vmargin={0pt,1cm}}
\setlength{\topmargin}{-1cm}
\setlength{\paperheight}{29.7cm}
\setlength{\textheight}{25.1cm}

% auto adjust the marginals
\usepackage{marginfix}

\usepackage{amsfonts}
\usepackage{amsmath}
\usepackage{amssymb}
\usepackage{amsthm}
%\usepackage{CJKutf8}   % for Chinese characters
\usepackage{ctex}
\usepackage{enumerate}
\usepackage{graphicx}  % for figures
\usepackage{layout}
\usepackage{multicol}  % multiple columns to reduce number of pages
\usepackage{mathrsfs}  
\usepackage{fancyhdr}
\usepackage{subfigure}
\usepackage{tcolorbox}
\usepackage{tikz-cd}
\usepackage{listings}
\usepackage{xcolor} %代码高亮
\usepackage{braket}
\usepackage{algorithm} 
\usepackage{algorithmicx}  
\usepackage{algpseudocode}  
\usepackage{amsmath} 
\usepackage{wrapfig} 

\usepackage{color}
\usepackage{listings}
\definecolor{mygreen}{rgb}{0,0.6,0}
\definecolor{mygray}{rgb}{0.5,0.5,0.5}
\definecolor{mymauve}{rgb}{0.58,0,0.82}
\lstset{ %
  backgroundcolor=\color{white},   % choose the background color; you must add \usepackage{color} or \usepackage{xcolor}
  basicstyle=\footnotesize,        % the size of the fonts that are used for the code
  breakatwhitespace=false,         % sets if automatic breaks should only happen at whitespace
  breaklines=true,                 % sets automatic line breaking
  captionpos=b,                    % sets the caption-position to bottom
  commentstyle=\color{mygreen},    % comment style
  deletekeywords={...},            % if you want to delete keywords from the given language
  escapeinside={\%*}{*)},          % if you want to add LaTeX within your code
  extendedchars=true,              % lets you use non-ASCII characters; for 8-bits encodings only, does not work with UTF-8
  %frame=single,                    % adds a frame around the code
  keepspaces=true,                 % keeps spaces in text, useful for keeping indentation of code (possibly needs columns=flexible)
  keywordstyle=\color{blue},       % keyword style
  language=Matlab,                 % the language of the code
  morekeywords={*,...},            % if you want to add more keywords to the set
  numbers=left,                    % where to put the line-numbers; possible values are (none, left, right)
  numbersep=5pt,                   % how far the line-numbers are from the code
  numberstyle=\tiny\color{mygray}, % the style that is used for the line-numbers
  rulecolor=\color{black},         % if not set, the frame-color may be changed on line-breaks within not-black text (e.g. comments (green here))
  showspaces=false,                % show spaces everywhere adding particular underscores; it overrides 'showstringspaces'
  showstringspaces=false,          % underline spaces within strings only
  showtabs=false,                  % show tabs within strings adding particular underscores
  stepnumber=1,                    % the step between two line-numbers. If it's 1, each line will be numbered
  stringstyle=\color{mymauve},     % string literal style
  tabsize=2,                       % sets default tabsize to 2 spaces
  %title=\lstname,                   % show the filename of files included with \lstinputlisting; also try caption instead of title
  flexiblecolumns=true
}


\floatname{algorithm}{算法}  
\renewcommand{\algorithmicrequire}{\textbf{输入:}}  
\renewcommand{\algorithmicensure}{\textbf{输出:}}  
\renewcommand{\algorithmicrequire}{\textbf{Input : }}
\renewcommand{\algorithmicrequire}{\textbf{Precondition : }}
\renewcommand{\algorithmicensure}{\textbf{Output : }}
\renewcommand{\algorithmicensure}{\textbf{Postcondition : }}
%------------------
% common commands %
%------------------
% differentiation
\newcommand{\gen}[1]{\left\langle #1 \right\rangle}
\newcommand{\dif}{\mathrm{d}}
\newcommand{\difPx}[1]{\frac{\partial #1}{\partial x}}
\newcommand{\difPy}[1]{\frac{\partial #1}{\partial y}}
\newcommand{\Dim}{\mathrm{D}}
\newcommand{\avg}[1]{\left\langle #1 \right\rangle}
\newcommand{\sgn}{\mathrm{sgn}}
\newcommand{\Span}{\mathrm{span}}
\newcommand{\dom}{\mathrm{dom}}
\newcommand{\Arity}{\mathrm{arity}}
\newcommand{\Int}{\mathrm{Int}}
\newcommand{\Ext}{\mathrm{Ext}}
\newcommand{\Cl}{\mathrm{Cl}}
\newcommand{\Fr}{\mathrm{Fr}}
% group is generated by
\newcommand{\grb}[1]{\left\langle #1 \right\rangle}
% rank
\newcommand{\rank}{\mathrm{rank}}
\newcommand{\Iden}{\mathrm{Id}}

% this environment is for solutions of examples and exercises
\newenvironment{solution}%
{\noindent\textbf{Solution.}}%
{\qedhere}
% the following command is for disabling environments
%  so that their contents do not show up in the pdf.
\makeatletter
\newcommand{\voidenvironment}[1]{%
\expandafter\providecommand\csname env@#1@save@env\endcsname{}%
\expandafter\providecommand\csname env@#1@process\endcsname{}%
\@ifundefined{#1}{}{\RenewEnviron{#1}{}}%
}
\makeatother

%---------------------------------------------
% commands specifically for complex analysis %
%---------------------------------------------
% complex conjugate
\newcommand{\ccg}[1]{\overline{#1}}
% the imaginary unit
\newcommand{\ii}{\mathbf{i}}
%\newcommand{\ii}{\boldsymbol{i}}
% the real part
\newcommand{\Rez}{\mathrm{Re}\,}
% the imaginary part
\newcommand{\Imz}{\mathrm{Im}\,}
% punctured complex plane
\newcommand{\pcp}{\mathbb{C}^{\bullet}}
% the principle branch of the logarithm
\newcommand{\Log}{\mathrm{Log}}
% the principle value of a nonzero complex number
\newcommand{\Arg}{\mathrm{Arg}}
\newcommand{\Null}{\mathrm{null}}
\newcommand{\Range}{\mathrm{range}}
\newcommand{\Ker}{\mathrm{ker}}
\newcommand{\Iso}{\mathrm{Iso}}
\newcommand{\Aut}{\mathrm{Aut}}
\newcommand{\ord}{\mathrm{ord}}
\newcommand{\Res}{\mathrm{Res}}
%\newcommand{\GL2R}{\mathrm{GL}(2,\mathbb{R})}
\newcommand{\GL}{\mathrm{GL}}
\newcommand{\SL}{\mathrm{SL}}
\newcommand{\Dist}[2]{\left|{#1}-{#2}\right|}

\newcommand\tbbint{{-\mkern -16mu\int}}
\newcommand\tbint{{\mathchar '26\mkern -14mu\int}}
\newcommand\dbbint{{-\mkern -19mu\int}}
\newcommand\dbint{{\mathchar '26\mkern -18mu\int}}
\newcommand\bint{
{\mathchoice{\dbint}{\tbint}{\tbint}{\tbint}}
}
\newcommand\bbint{
{\mathchoice{\dbbint}{\tbbint}{\tbbint}{\tbbint}}
}





%----------------------------------------
% theorem and theorem-like environments %
%----------------------------------------
\numberwithin{equation}{chapter}
\theoremstyle{definition}

\newtheorem{thm}{Theorem}[chapter]
\newtheorem{axm}[thm]{Axiom}
\newtheorem{alg}[thm]{Algorithm}
\newtheorem{asm}[thm]{Assumption}
\newtheorem{defn}[thm]{Definition}
\newtheorem{prop}[thm]{Proposition}
\newtheorem{rul}[thm]{Rule}
\newtheorem{coro}[thm]{Corollary}
\newtheorem{lem}[thm]{Lemma}
\newtheorem{exm}{Example}[chapter]
\newtheorem{rem}{Remark}[chapter]
\newtheorem{exc}[exm]{Exercise}
\newtheorem{frm}[thm]{Formula}
\newtheorem{ntn}{Notation}

% for complying with the convention in the textbook
\newtheorem{rmk}[thm]{Remark}


%----------------------
% the end of preamble %
%----------------------

\begin{document}
% \pagestyle{plain}

%\tableofcontents
%\clearpage

\pagestyle{fancy}
\pagenumbering{roman}
%\fancyhead{}
%\lhead{Qinghai Zhang}
%\chead{Notes on Algebraic Topology}
%\rhead{Fall 2018}

\chapter{hw1 12235005 谭焱}

\paragraph*{1.14.}
\begin{solution}
    \begin{itemize}
        \item Truncation or discretization:

              Some features of a mathematical model may be omitted or simplified
              (e.g., replacing  derivatives by finite differences or using only
              a finite number of terms in an infinite series).

        \item Rounding:

              Whether in hand in computation, a calculator, or digital computer,
              the representation of real numbers and arithmetic operations upon
              them is ultimately limited to some finite amount of precision
              and thus is generally inexact.

    \end{itemize}
\end{solution}

\paragraph*{1.51.}
\begin{solution}
    \begin{itemize}
        \item If the coefficients are very large or very small,
              then $b^2$ or $4ac$ may overflow or underflow.

        \item Cancellation insider the square root, when the
              discriminant is small relative to the coefficients.
    \end{itemize}
\end{solution}

\paragraph*{1.10. }
\begin{solution}
    Code as follow,
    \begin{itemize}
        \item When $-b \pm \sqrt{b^2 - 4ac}$ or $2 * a$ too small,
              we should use second formula.

        \item Same as $-b$ and square, $2 * c$ too small, consider
              first formula.
    \end{itemize}

    \lstinputlisting[firstnumber=1]{exRoot.m}
\end{solution}

\paragraph*{2.39. }
\begin{solution}
    (a) 4 ,(b) 6 ,(c) -10.
\end{solution}

\paragraph*{2.40. }
\begin{solution}
    \begin{itemize}
        \item In finite-precision arithmetic the choice should be
              made with some care to minimize propagation of numerical
              error. In particular, we wish to limit the magnitudes of
              the multipliers so that previous rounding errors will not
              be amplified when remain portion of the matrix and right-hand
              side are multiplied by each elementary elimination matrix.

        \item     Let $\mathit{E}$ is the backward error in the matrix $\mathit{A}$.
              For LU factorization by Gaussian elimination, a bound of form
              \[ \frac{\| \mathit{E}\|}{\mathit{A}} \leq \rho n^2 \epsilon_{mach}\]
              holds, where $\rho$, called the growth factor, is  the ratio of
              the largest entry of $\mathit{A}$ in magnitude. Without pivoting
              $\rho$ can be arbitrary large, and hence Gaussian elimination
              without pivoting is unstable.
    \end{itemize}

\end{solution}

\paragraph*{2.61. }
\begin{solution}
    \begin{enumerate}
        \item [(a)] $cond_1 =  10^20$, is ill-conditioned.
        \item [(b)] $cond_1 = 1$, is well-conditioned.
        \item [(c)] $cond_1 = 1$, is well-conditioned.
        \item [(d)] $cond = \infty $, is ill-conditioned.
    \end{enumerate}
\end{solution}

\paragraph*{2.17. }
\begin{solution}
    \begin{align*}
        \left[ \begin{array}{rrr}
                1  & -1 & 0  \\
                -1 & 2  & -1 \\
                0  & -1 & 1  \\
            \end{array} \right]
         & = \left[ \begin{array}{rrr}
                1  & 0 & 0 \\
                -1 & 1 & 0 \\
                0  & 0 & 1 \\
            \end{array}\right]
        \left[ \begin{array}{rrr}
                1 & -1 & 0  \\
                0 & 1  & -1 \\
                0 & -1 & 1  \\
            \end{array}\right]                                                                \\
         & = \left[ \begin{array}{rrr}
                1  & 0 & 0 \\
                -1 & 1 & 0 \\
                0  & 0 & 1 \\
            \end{array}\right]
        \left[ \begin{array}{rrr}
                1 & 0  & 0 \\
                0 & 1  & 0 \\
                0 & -1 & 1 \\
            \end{array}\right]
        \left[ \begin{array}{rrr}
                1 & -1 & 0  \\
                0 & 1  & -1 \\
                0 & 0  & 0  \\
            \end{array}\right]                                                               \\
         & =   \left[ \begin{array}{rrr}
                1  & 0  & 0 \\
                -1 & 1  & 0 \\
                0  & -1 & 0 \\
            \end{array}\right]\left[ \begin{array}{rrr}
                1 & -1 & 0  \\
                0 & 1  & -1 \\
                0 & 0  & 0  \\
            \end{array}\right] =: \mathit{LU}
    \end{align*}
\end{solution}

\paragraph*{2.28. }
\begin{solution}
    \begin{align*}
          & (A - UV^T)(A^{-1} +A^{-1}U(I - V^T A^{-1} U)^{-1} V^T A^{-1})                                   \\
        = & I + U(I - V^T A^{-1}U)^{-1}V^TA^{-1} -UV^TA^{-1} -UV^TA^{-1}U(I - V^T A^{-1} U)^{-1} V^T A^{-1} \\
        = & I + UV^TA^{-1} - UV^RA^{-1}                                                                     \\
        = & I.
    \end{align*}
    Similarly, $(A^{-1} +A^{-1}U(I - V^T A^{-1} U)^{-1} V^T A^{-1})(A -UV^T) = I$.

    So, $(A^{-1} +A^{-1}U(I - V^T A^{-1} U)^{-1} V^T A^{-1}) = (A - UV^T)^{-1}$.
\end{solution}

\paragraph*{2.34. }
\begin{solution}
    \begin{itemize}
        \item [(a)] Suppose $\mathit{A}$ is singular, there is exist
              $x$ such that $Ax = 0$. Hence $x^TAx = 0$, which conflict
              with $\mathit{A}$ is positive definite matrix.

        \item [(b)] Suppose $\mathit{A}^{-1}$ is not positive definite.
              There is exist $x$ such that $x^T A^{-1} x <= 0$.

              Since $\mathit{A}$ is not singular, $Ay = x$ have a solution $y$.
              Which means $0 >= x^T A^{-1} x = y^T A^T A^{-1} A y = y^T A^T y$.
              That is conflict with $A, A^T$ is positive definite.
    \end{itemize}
\end{solution}

\paragraph*{2.7. }
\begin{solution}
    \begin{itemize}
        \item [(a)]
              When partial pivoting is used, nothing differences with
              no pivoting using.
              During eliminating the matrix,
              entries of the transformed matrix not grow.
              When complete pivoting is used, some right col permutation
              matrix $\mathit{Q}$ will be generated, number of transformed matrix won't
              change.
              \begin{align*}
                  \left[ \begin{array}{rrrrr}
                          1  & 0  & 0  & 0  & 1 \\
                          -1 & 1  & 0  & 0  & 1 \\
                          -1 & -1 & 1  & 0  & 1 \\
                          -1 & -1 & -1 & 1  & 1 \\
                          -1 & -1 & -1 & -1 & 1 \\
                      \end{array}\right] =
                  \left[ \begin{array}{rrrrr}
                          1 & 0 & 0 & 0 & 0 \\
                          1 & 1 & 0 & 0 & 0 \\
                          1 & 0 & 1 & 0 & 0 \\
                          1 & 0 & 0 & 1 & 0 \\
                          1 & 0 & 0 & 0 & 1 \\
                      \end{array}\right]
                  \left[ \begin{array}{rrrrr}
                          1 & 0  & 0  & 0  & 1 \\
                          0 & 1  & 0  & 0  & 2 \\
                          0 & -1 & 1  & 0  & 2 \\
                          0 & -1 & -1 & 1  & 2 \\
                          0 & -1 & -1 & -1 & 2 \\
                      \end{array}\right] \\
                  =\left[ \begin{array}{rrrrr}
                          1 & 0 & 0 & 0 & 0 \\
                          1 & 1 & 0 & 0 & 0 \\
                          1 & 0 & 1 & 0 & 0 \\
                          1 & 0 & 0 & 1 & 0 \\
                          1 & 0 & 0 & 0 & 1 \\
                      \end{array}\right]
                  \left[ \begin{array}{rrrrr}
                          1 & 1 & 0  & 0  & 0  \\
                          0 & 2 & 0  & 0  & 1  \\
                          0 & 2 & 1  & 0  & -1 \\
                          0 & 2 & -1 & 1  & -1 \\
                          0 & 2 & -1 & -1 & -1 \\
                      \end{array}\right]
                  \left[ \begin{array}{rrrrr}
                          1 & 0 & 0 & 0 & 0 \\
                          0 & 0 & 0 & 0 & 1 \\
                          0 & 0 & 1 & 0 & 0 \\
                          0 & 0 & 0 & 1 & 0 \\
                          0 & 1 & 0 & 0 & 0 \\
                      \end{array}\right] \\
                  ...
              \end{align*}

        \item[(b)] using program as following
            \lstinputlisting[firstnumber=1]{GaussEliminateSolver.m}
            \lstinputlisting[firstnumber=1]{test.m}.

            We get result in table
            \begin{table}[htb]
                \centering
                \resizebox{.5\textwidth}{!}{%
                    \begin{tabular}{cccc}
                        $n$ & $\log \|\mathit{E}\|$ & $\log \|\mathit{R}\|$ & $cond(\mathit{A})$     \\
                        5   & -34.72412472 & -33.9800862  & 2.22392344                                \\
                        7   & -34.40508102 & -34.0399868  & 3.075424567                               \\
                        9   & -34.16596879 & -33.90532033 & 3.945777218                               \\
                        11  & -31.96870812 & -31.62615731 & 4.825981291                               \\
                        13  & -32.13881932 & -31.68500675 & 5.711914731                               \\
                        15  & -28.68323595 & -28.30497086 & 6.601454215                               \\
                        17  & -27.98571253 & -27.25548417 & 7.493403319                               \\
                        19  & -25.99587455 & -25.60176018 & 8.387039353                               \\
                    \end{tabular}%
                }
                \caption{matrix size, error ,residual and condition number.}
            \end{table},
            Which imply about linear relation between $\log(error)$ and size $n$.
    \end{itemize}
\end{solution}
\end{document}
