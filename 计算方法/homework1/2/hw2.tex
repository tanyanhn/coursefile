\documentclass[a4paper]{book}

\usepackage{geometry}
% make full use of A4 papers
\geometry{margin=1.5cm, vmargin={0pt,1cm}}
\setlength{\topmargin}{-1cm}
\setlength{\paperheight}{29.7cm}
\setlength{\textheight}{25.1cm}

% auto adjust the marginals
\usepackage{marginfix}

\usepackage{amsfonts}
\usepackage{amsmath}
\usepackage{amssymb}
\usepackage{amsthm}
%\usepackage{CJKutf8}   % for Chinese characters
\usepackage{ctex}
\usepackage{enumerate}
\usepackage{graphicx}  % for figures
\usepackage{layout}
\usepackage{multicol}  % multiple columns to reduce number of pages
\usepackage{mathrsfs}  
\usepackage{fancyhdr}
\usepackage{subfigure}
\usepackage{tcolorbox}
\usepackage{tikz-cd}
\usepackage{listings}
\usepackage{xcolor} %代码高亮
\usepackage{braket}
\usepackage{algorithm} 
\usepackage{algorithmicx}  
\usepackage{algpseudocode}  
\usepackage{amsmath} 
\usepackage{wrapfig} 
\usepackage{bm}

\usepackage{color}
\usepackage{listings}
\definecolor{mygreen}{rgb}{0,0.6,0}
\definecolor{mygray}{rgb}{0.5,0.5,0.5}
\definecolor{mymauve}{rgb}{0.58,0,0.82}
\lstset{ %
  backgroundcolor=\color{white},   % choose the background color; you must add \usepackage{color} or \usepackage{xcolor}
  basicstyle=\footnotesize,        % the size of the fonts that are used for the code
  breakatwhitespace=false,         % sets if automatic breaks should only happen at whitespace
  breaklines=true,                 % sets automatic line breaking
  captionpos=b,                    % sets the caption-position to bottom
  commentstyle=\color{mygreen},    % comment style
  deletekeywords={...},            % if you want to delete keywords from the given language
  escapeinside={\%*}{*)},          % if you want to add LaTeX within your code
  extendedchars=true,              % lets you use non-ASCII characters; for 8-bits encodings only, does not work with UTF-8
  %frame=single,                    % adds a frame around the code
  keepspaces=true,                 % keeps spaces in text, useful for keeping indentation of code (possibly needs columns=flexible)
  keywordstyle=\color{blue},       % keyword style
  language=Matlab,                 % the language of the code
  morekeywords={*,...},            % if you want to add more keywords to the set
  numbers=left,                    % where to put the line-numbers; possible values are (none, left, right)
  numbersep=5pt,                   % how far the line-numbers are from the code
  numberstyle=\tiny\color{mygray}, % the style that is used for the line-numbers
  rulecolor=\color{black},         % if not set, the frame-color may be changed on line-breaks within not-black text (e.g. comments (green here))
  showspaces=false,                % show spaces everywhere adding particular underscores; it overrides 'showstringspaces'
  showstringspaces=false,          % underline spaces within strings only
  showtabs=false,                  % show tabs within strings adding particular underscores
  stepnumber=1,                    % the step between two line-numbers. If it's 1, each line will be numbered
  stringstyle=\color{mymauve},     % string literal style
  tabsize=2,                       % sets default tabsize to 2 spaces
  %title=\lstname,                   % show the filename of files included with \lstinputlisting; also try caption instead of title
  flexiblecolumns=true
}


\floatname{algorithm}{算法}  
\renewcommand{\algorithmicrequire}{\textbf{输入:}}  
\renewcommand{\algorithmicensure}{\textbf{输出:}}  
\renewcommand{\algorithmicrequire}{\textbf{Input : }}
\renewcommand{\algorithmicrequire}{\textbf{Precondition : }}
\renewcommand{\algorithmicensure}{\textbf{Output : }}
\renewcommand{\algorithmicensure}{\textbf{Postcondition : }}
%------------------
% common commands %
%------------------
% differentiation
\newcommand{\gen}[1]{\left\langle #1 \right\rangle}
\newcommand{\dif}{\mathrm{d}}
\newcommand{\difPx}[1]{\frac{\partial #1}{\partial x}}
\newcommand{\difPy}[1]{\frac{\partial #1}{\partial y}}
\newcommand{\Dim}{\mathrm{D}}
\newcommand{\avg}[1]{\left\langle #1 \right\rangle}
\newcommand{\sgn}{\mathrm{sgn}}
\newcommand{\Span}{\mathrm{span}}
\newcommand{\dom}{\mathrm{dom}}
\newcommand{\Arity}{\mathrm{arity}}
\newcommand{\Int}{\mathrm{Int}}
\newcommand{\Ext}{\mathrm{Ext}}
\newcommand{\Cl}{\mathrm{Cl}}
\newcommand{\Fr}{\mathrm{Fr}}
% group is generated by
\newcommand{\grb}[1]{\left\langle #1 \right\rangle}
% rank
\newcommand{\rank}{\mathrm{rank}}
\newcommand{\Iden}{\mathrm{Id}}

% this environment is for solutions of examples and exercises
\newenvironment{solution}%
{\noindent\textbf{Solution.}}%
{\qedhere}
% the following command is for disabling environments
%  so that their contents do not show up in the pdf.
\makeatletter
\newcommand{\voidenvironment}[1]{%
\expandafter\providecommand\csname env@#1@save@env\endcsname{}%
\expandafter\providecommand\csname env@#1@process\endcsname{}%
\@ifundefined{#1}{}{\RenewEnviron{#1}{}}%
}
\makeatother

%---------------------------------------------
% commands specifically for complex analysis %
%---------------------------------------------
% complex conjugate
\newcommand{\ccg}[1]{\overline{#1}}
% the imaginary unit
\newcommand{\ii}{\mathbf{i}}
%\newcommand{\ii}{\boldsymbol{i}}
% the real part
\newcommand{\Rez}{\mathrm{Re}\,}
% the imaginary part
\newcommand{\Imz}{\mathrm{Im}\,}
% punctured complex plane
\newcommand{\pcp}{\mathbb{C}^{\bullet}}
% the principle branch of the logarithm
\newcommand{\Log}{\mathrm{Log}}
% the principle value of a nonzero complex number
\newcommand{\Arg}{\mathrm{Arg}}
\newcommand{\Null}{\mathrm{null}}
\newcommand{\Range}{\mathrm{range}}
\newcommand{\Ker}{\mathrm{ker}}
\newcommand{\Iso}{\mathrm{Iso}}
\newcommand{\Aut}{\mathrm{Aut}}
\newcommand{\ord}{\mathrm{ord}}
\newcommand{\Res}{\mathrm{Res}}
%\newcommand{\GL2R}{\mathrm{GL}(2,\mathbb{R})}
\newcommand{\GL}{\mathrm{GL}}
\newcommand{\SL}{\mathrm{SL}}
\newcommand{\Dist}[2]{\left|{#1}-{#2}\right|}

\newcommand\tbbint{{-\mkern -16mu\int}}
\newcommand\tbint{{\mathchar '26\mkern -14mu\int}}
\newcommand\dbbint{{-\mkern -19mu\int}}
\newcommand\dbint{{\mathchar '26\mkern -18mu\int}}
\newcommand\bint{
{\mathchoice{\dbint}{\tbint}{\tbint}{\tbint}}
}
\newcommand\bbint{
{\mathchoice{\dbbint}{\tbbint}{\tbbint}{\tbbint}}
}





%----------------------------------------
% theorem and theorem-like environments %
%----------------------------------------
\numberwithin{equation}{chapter}
\theoremstyle{definition}

\newtheorem{thm}{Theorem}[chapter]
\newtheorem{axm}[thm]{Axiom}
\newtheorem{alg}[thm]{Algorithm}
\newtheorem{asm}[thm]{Assumption}
\newtheorem{defn}[thm]{Definition}
\newtheorem{prop}[thm]{Proposition}
\newtheorem{rul}[thm]{Rule}
\newtheorem{coro}[thm]{Corollary}
\newtheorem{lem}[thm]{Lemma}
\newtheorem{exm}{Example}[chapter]
\newtheorem{rem}{Remark}[chapter]
\newtheorem{exc}[exm]{Exercise}
\newtheorem{frm}[thm]{Formula}
\newtheorem{ntn}{Notation}

% for complying with the convention in the textbook
\newtheorem{rmk}[thm]{Remark}


%----------------------
% the end of preamble %
%----------------------

\begin{document}
% \pagestyle{plain}

%\tableofcontents
%\clearpage

\pagestyle{fancy}
\pagenumbering{roman}
%\fancyhead{}
%\lhead{Qinghai Zhang}
%\chead{Notes on Algebraic Topology}
%\rhead{Fall 2018}

\setcounter{chapter}{1}
\chapter{hw2 12235005 谭焱}

\paragraph*{3.3 }
\begin{solution}
    By  substituting the data point $(1, 2), (2, 3), (3, 5)$ have 
    \begin{align*}
        Ax = \left[
            \begin{array}{cc}
            1 & e \\
            2 & e^2 \\
            3 & e^3 \\
        \end{array}
        \right]
        \left[\begin{array}{c}
            x_1 \\
            x_2 
        \end{array}\right]
        \cong \left[ \begin{array}{c}
            2 \\
            3 \\
            5
        \end{array} \right] = b
    \end{align*}
\end{solution}

\paragraph*{3.7 }
\begin{solution}
    \begin{itemize}
        \item [(a)] The function $\phi(\bm y) = \| \bm b - 
        \bm y \|_2$ is continuous and coercive on $\mathbb{R}^m$,
        so $\phi$ has a minimum on the closed, unbounded set span
        ($\bm A$), i.e., there is an $m$-vector $\bm y \in $ span$ 
        (\bm A)$ closest to $\bm b$ in the Euclidean norm. 

        \item [(b)] Suppose $\bm x_1$ and $\bm x_2$ are such 
        solutions, and let \bm{$z = x_2 - x_1$}. Then since \bm 
        {$Ax_1 = y = Ax_2$}, we have \bm{$Az = 0$}. Now if 
        \bm{$z \neq 0 \Leftrightarrow x_1 \neq x_2$}, then the 
        columns of \bm$A$ must be linearly dependent. We conclude
        that teh solution to an $m \times n$ least squares problem 
        \bm{$Ax \cong b$} is  unique if, and only if, $\bm A$ has 
        full column rank, i.e., rank($\bm A$) = n.
    \end{itemize}
\end{solution}

\paragraph*{3.17 }
\begin{solution}
    From definition, we have 
    \begin{align*}
        \alpha = - \text{sign} (a_1)\|\bm a\|_2 = -2 \\
        \bm v = \bm a - \alpha \bm e_1 = [3\ 1\ 1\ 1]^T
    \end{align*}
\end{solution}
\paragraph*{3.20 }
\begin{solution}\begin{itemize}
        \item [(a)] It's possible to annihilate $a_2$ with Givens rotation
        \begin{align*}
            G = \left[
                \begin{array}{cc}
                    0 & 1 \\
                    1 & 0 \\
                \end{array}
            \right]
        \end{align*}
        \item [(b)] It's not possible, since in  elimination matrix
        calculating, $a_2 / a_1 = a_2 
        / 0$ is meaningless.
    \end{itemize}
\end{solution}
    
\paragraph*{3.28 }
\begin{solution}
    \begin{itemize}
        \item [(a)] Since $\bm q_k$ is orthogonal, imply $\bm {q_i^T q_j} = 0, i \neq j$
        \begin{align*}
            &\bm{(I - P_k)(I - P_{k-1})\ldots(I - P_{1})}
            =\bm{I - \sum P_m + \sum q_i (q_i^T q_j) q_j^T (\ldots)} \\
            =&\bm{I - \sum P_m + 0 * \sum q_i q_j^T (\ldots)}
            =\bm{I - \sum P_m}
        \end{align*}
        
        \item [(b)] In the classical Gram-Schmidt procedure 
        \begin{align*}
            \bm {q_k} =& \bm{a_k - \sum_j (q_j^T a_k) q_j}
            =\bm { a_k - \sum_j q_j (q_j^T a_k)} 
            =\bm { (I - \sum_j P_j)a_k}
        \end{align*}

        \item [(c)] In the modified Gram-Schmidt procedure,
        assume $M_j(a_k) = a_k - (q_j^T a_k)q_j = \bm{(I - P_j)}a_k$ 
        \begin{align*}
            \bm {q_k} =& \bm{M_{k -1}(M_{k - 1}( \ldots M_1(a_k) \ldots)) }
            =\bm {(I - P_{k - 1})(M_{k - 1}( \ldots M_1(a_k) \ldots))} 
            =\bm { (I - P_{k - 1})\ldots (I - P_1) a_k}
        \end{align*}

        \item [(d)] It's obvious that is same as (a) like
        \begin{align*}
            &\bm{(I - \sum P_i)(I - \sum P_{j})}
            =\bm{I - 2* \sum P_m +  \sum P_{m}^2 + \sum q_i (q_i^T q_j) q_j^T (\ldots)} \\
            =&\bm{I - 2* \sum P_m + \sum P_{m} + 0 * \sum q_i q_j^T (\ldots)}
            =\bm{I - \sum P_m}
        \end{align*}
    \end{itemize}
\end{solution}

\paragraph*{4.2 }
\begin{solution}
    Since the matrix is upper triangular matrix, the eigenvalues
    and corresponding eigenvectors are 
    \begin{align*}
        A \left[\begin{array}{c}
            v_1 \\
            v_2 \\ 
            v_3\\
        \end{array}\right] = A \left[ \begin{array}{ccc}
            1& 0 & 0 \\
            0& 1 & 0 \\
            0 & 0 & 1 \\
        \end{array}\right] = 
        \left[\begin{array}{c}
            1 v_1 \\
            2 v_2\\ 
            3 v_3\\
        \end{array}\right]
        = [1\ 2\ 3]^T [v_1\ v_2\ v_3] = e^T [v_1\ v_2\ v_3]
    \end{align*}
\end{solution}

\paragraph*{4.14 }
\begin{solution}
    \begin{itemize}
        \item [(a)] Let $\alpha = 0$ the matrix is lower 
        triangular matrix and eigenvalues is diagonal $[1\ 2\ 3]$
        which is all real values.
        
        \item [(b)] It's impossible that since real matrix eigenvalues 
        has nonzero imaginary part exist as pair. Which is coming from when 
        $a + bi$ and $v$ is eigenvalue and eigenvector, It's easy get 
        $a - bi$ and $\bar{v}$ is another eigenvalue and eigenvector
        by substituting.
        However, the matrix have odd eigenvalue that conflicted with 
        all nonzero imaginary part complex eigenvalue.
    \end{itemize}
\end{solution}

\paragraph*{4.17 }
\begin{solution}
    Assuming $v$ is eigenvector of eigenvalue $\lambda$. It's 
    easy verify that $A^2 v = A \lambda v = \lambda (Av) = 
    \lambda^2 v$, so $\lambda^2$ is $A^2$'s eigenvalue.
\end{solution}

\paragraph*{4.22 }
\begin{solution}
    \begin{itemize}
        \item [(a)] \begin{align*}
            A \left[\begin{array}{c}
                \bm u \\
                \bm 0 \\
            \end{array}\right] =
            \left[\begin{array}{cc}
                A_{11} & A_{12} \\
                \bm O      & A_{22} \\
            \end{array}\right] 
            \left[\begin{array}{c}
                \bm u \\
                \bm 0 \\    
            \end{array}\right] =
            \left[\begin{array}{c}
                A_{11} \bm u \\
                \bm 0   
            \end{array}\right] = 
             \lambda \left[\begin{array}{c}
                \bm u \\
                \bm 0 \\
            \end{array}\right]
        \end{align*}
        
        \item [(b)] \begin{align*}
            A \left[\begin{array}{c}
                \bm u \\
                \bm v \\
            \end{array}\right] =
            \left[\begin{array}{cc}
                A_{11} & A_{12} \\
                \bm O  & A_{22} \\
            \end{array}\right] 
            \left[\begin{array}{c}
                \bm u \\
                \bm v \\    
            \end{array}\right] =
            \left[\begin{array}{c}
                A_{11} \bm u + A_{12} \bm v \\
                A_{22} \bm v   
            \end{array}\right] = 
             \lambda \left[\begin{array}{c}
                \bm u \\
                \bm v \\
            \end{array}\right]
        \end{align*}
        , we need satisfy $ A_{11} \bm u + A_{12} \bm v = \lambda u$, 
        which is equal to $\bm u = (A_{11} - \lambda I)^{-1} A_{12} \bm v$.
        Since $\lambda$ is not $A_{11}$'s eigenvalue, $ (A_{11} - \lambda I)^{-1}$
        is exist, so $\lambda$ and $[(A_{11} - \lambda I)^{-1}A_{12} \bm v \ \bm v]^{T}$
        is eigenvalue and eigenvector for $A$.

        \item [(c)] By result in (b), 
        \begin{align*}
            A_{11} \bm u + A_{12} \bm v = \lambda u \\ 
            A_{22} \bm v = \lambda \bm v.
        \end{align*}
        When $\bm v \neq \bm 0$ we have $A_{22} \bm v = \lambda \bm v$,
        When $\bm b = \bm 0$, we have $A_{11} \bm u = \lambda \bm u$. 
        So $\lambda$ is eigenvalue of $A_{11}, \bm u$ or $A_{22}, \bm v$.

        \item [(d)] The sufficiency follows from (a) and (b) while the necessity follows from (c).
    \end{itemize}
\end{solution}

\paragraph*{4.32 }
\begin{solution}
    \begin{itemize}
        \item [(a)] Assume a orthogonal basis contain $v$ is 
        $U = [v, u_0, \ldots, u_k]$, so $v^Tu_i = 0$ means 
        \begin{align*}
            Hv &= Iv - 2 \frac{v v^T v}{v^T v} = v - 2v = -v \\
            Hu_i&= Iu_i - 2 \frac{v v^T u_i}{v^T v} = u_i.
        \end{align*}
        And $U$ is a basis, so eigenvalues is -1 with $v$ and 1 with $u_i$.

        \item [(b)] The characteristic polynomial of $H$ is 
        \[p(\lambda) =\left| \begin{array}{cc}
            c - \lambda & s \\
            -s & c - \lambda \\
        \end{array} \right| = \lambda^2 - 2\lambda c +c^2 +s^2. \]
        The eigenvalues is solution of characteristic polynomial zero points 
        $c \pm is$.
    \end{itemize}
\end{solution}
\end{document}
