\documentclass[a4paper]{book}

\usepackage{geometry}
% make full use of A4 papers
\geometry{margin=1.5cm, vmargin={0pt,1cm}}
\setlength{\topmargin}{-1cm}
\setlength{\paperheight}{29.7cm}
\setlength{\textheight}{25.1cm}

% auto adjust the marginals
\usepackage{marginfix}

\usepackage{amsfonts}
\usepackage{amsmath}
\usepackage{amssymb}
\usepackage{amsthm}
%\usepackage{CJKutf8}   % for Chinese characters
\usepackage{ctex}
\usepackage{enumerate}
\usepackage{graphicx}  % for figures
\usepackage{layout}
\usepackage{multicol}  % multiple columns to reduce number of pages
\usepackage{mathrsfs}  
\usepackage{fancyhdr}
\usepackage{subfigure}
\usepackage{tcolorbox}
\usepackage{tikz-cd}
\usepackage{listings}
\usepackage{xcolor} %代码高亮
%------------------
% common commands %
%------------------
% differentiation
\newcommand{\gen}[1]{\left\langle #1 \right\rangle}
\newcommand{\dif}{\mathrm{d}}
\newcommand{\difPx}[1]{\frac{\partial #1}{\partial x}}
\newcommand{\difPy}[1]{\frac{\partial #1}{\partial y}}
\newcommand{\Dim}{\mathrm{D}}
\newcommand{\avg}[1]{\left\langle #1 \right\rangle}
\newcommand{\sgn}{\mathrm{sgn}}
\newcommand{\Span}{\mathrm{span}}
\newcommand{\dom}{\mathrm{dom}}
\newcommand{\Arity}{\mathrm{arity}}
\newcommand{\Int}{\mathrm{Int}}
\newcommand{\Ext}{\mathrm{Ext}}
\newcommand{\Cl}{\mathrm{Cl}}
\newcommand{\Fr}{\mathrm{Fr}}
% group is generated by
\newcommand{\grb}[1]{\left\langle #1 \right\rangle}
% rank
\newcommand{\rank}{\mathrm{rank}}
\newcommand{\Iden}{\mathrm{Id}}

% this environment is for solutions of examples and exercises
\newenvironment{solution}%
{\noindent\textbf{Solution.}}%
{\qedhere}
% the following command is for disabling environments
%  so that their contents do not show up in the pdf.
\makeatletter
\newcommand{\voidenvironment}[1]{%
  \expandafter\providecommand\csname env@#1@save@env\endcsname{}%
  \expandafter\providecommand\csname env@#1@process\endcsname{}%
  \@ifundefined{#1}{}{\RenewEnviron{#1}{}}%
}
\makeatother

%---------------------------------------------
% commands specifically for complex analysis %
%---------------------------------------------
% complex conjugate
\newcommand{\ccg}[1]{\overline{#1}}
% the imaginary unit
\newcommand{\ii}{\mathbf{i}}
%\newcommand{\ii}{\boldsymbol{i}}
% the real part
\newcommand{\Rez}{\mathrm{Re}\,}
% the imaginary part
\newcommand{\Imz}{\mathrm{Im}\,}
% punctured complex plane
\newcommand{\pcp}{\mathbb{C}^{\bullet}}
% the principle branch of the logarithm
\newcommand{\Log}{\mathrm{Log}}
% the principle value of a nonzero complex number
\newcommand{\Arg}{\mathrm{Arg}}
\newcommand{\Null}{\mathrm{null}}
\newcommand{\Range}{\mathrm{range}}
\newcommand{\Ker}{\mathrm{ker}}
\newcommand{\Iso}{\mathrm{Iso}}
\newcommand{\Aut}{\mathrm{Aut}}
\newcommand{\ord}{\mathrm{ord}}
\newcommand{\Res}{\mathrm{Res}}
%\newcommand{\GL2R}{\mathrm{GL}(2,\mathbb{R})}
\newcommand{\GL}{\mathrm{GL}}
\newcommand{\SL}{\mathrm{SL}}
\newcommand{\Dist}[2]{\left|{#1}-{#2}\right|}



%----------------------------------------
% theorem and theorem-like environments %
%----------------------------------------
\numberwithin{equation}{chapter}
\theoremstyle{definition}

\newtheorem{thm}{Theorem}[chapter]
\newtheorem{axm}[thm]{Axiom}
\newtheorem{alg}[thm]{Algorithm}
\newtheorem{asm}[thm]{Assumption}
\newtheorem{defn}[thm]{Definition}
\newtheorem{prop}[thm]{Proposition}
\newtheorem{rul}[thm]{Rule}
\newtheorem{coro}[thm]{Corollary}
\newtheorem{lem}[thm]{Lemma}
\newtheorem{exm}{Example}[chapter]
\newtheorem{rem}{Remark}[chapter]
\newtheorem{exc}[exm]{Exercise}
\newtheorem{frm}[thm]{Formula}
\newtheorem{ntn}{Notation}

% for complying with the convention in the textbook
\newtheorem{rmk}[thm]{Remark}


%\lstset{
%	backgroundcolor=\color{red!50!green!50!blue!50},%代码块背景色为浅灰色
%	rulesepcolor= \color{gray}, %代码块边框颜色
%	breaklines=true,  %代码过长则换行
%	numbers=left, %行号在左侧显示
%	numberstyle= \small,%行号字体
%	keywordstyle= \color{blue},%关键字颜色
%	commentstyle=\color{gray}, %注释颜色
%	frame=shadowbox%用方框框住代码块
%}
\lstset{
	columns=fixed,       
	numbers=left,                                        % 在左侧显示行号
	numberstyle=\tiny\color{gray},                       % 设定行号格式
	frame=none,                                          % 不显示背景边框
	backgroundcolor=\color[RGB]{245,245,244},            % 设定背景颜色
	keywordstyle=\color[RGB]{40,40,255},                 % 设定关键字颜色
	numberstyle=\footnotesize\color{darkgray},           
	commentstyle=\it\color[RGB]{0,96,96},                % 设置代码注释的格式
	stringstyle=\rmfamily\slshape\color[RGB]{128,0,0},   % 设置字符串格式
	showstringspaces=false,                              % 不显示字符串中的空格
	language=c++,                                        % 设置语言
}

%----------------------
% the end of preamble %
%----------------------

\begin{document}
	
	马克思主义部分
	自然科学,社会科学与思想科学交叉的马克思主义哲学
	
	哲学
	爱智,追求明智,明智——从复杂情况中寻找出路。
	
	
	自然辩证法:是马克思主义观关于自然科学和科学技术发展的一般规律。人类认识自然改造自然的一般方法。以及科学技术和人类社会相互作用的一场原理。
	
	
	
	课程目标
	自然观
	更理智地理解人与自然的关系。
	
	科学技术观
	提高科学精神,增强创造能力。
	
	方法论
	掌握严谨的思维方法
	要学会控制自己的思想,使他们成熟,以便从一千个思想中选出其一,然后从一千个可能的地方选出它最适合的一个地方。
	
	对理论范畴的批判,是通过将理论范畴和实际的经验事实进行对比来实现的。
	
	科学技术社会观
	增强历史幸福感和社会责任感。
	
	
	
	回答问题的流程
	发现问题之所在1确定用以回答问题的知识域。
	理清问题的由来2追溯问题的的来源与解题的历史,联系实际,确定观点。
	寻找解题之思路3设计论证结构:演绎推理,由一般知识推出个别知识的结论的推理 例子
	问题 苏格拉底会死的吗
	所有人都是会死的
	苏格拉底是人
	所以苏格拉底是会死的
	建构解题之语言4叙述解题过程,阐述对问题的理解,提出观点,论证观点,(反驳对立观点)
	结论。
	
	
	1.1人工自然与生态危机
	人与自然的对象性关系  相互依存,相互制约
	人工自然界  人类运用科学和技术制造的系统自然界.
	
	1.2反思人与自然的关系
	
	
	1.3可持续发展及其建设
	
	生态危机:
	因人类的不合理活动,在全球规模或局部区域导致生态系统的结构性和功能性的损害,和生命系统的瓦解,从而危害到人的生命和发展的现象.
	
	生态危机的表现,
	1, 自然资源消耗过快
	2, 环境污染严重
	
	反思人相对于自然的态度
	1, 悲观主义
	"快速的工业化","人口数量的增长","日益严重的食物匮乏","不可再生的资源储备枯竭",以及"恶化的自然环境"是制约人类增长的五个极限.
	如果当前的五个趋势持续下去的话,世界将会在今后100年达到人类增长的极限,最可能的结果是人类数量突然不可控制地急速下降和生产量的急剧下滑.
	
	改变这种趋势和建立稳定的生态和经济条件以支持遥远的未来是可能的.全球均衡状态可以这样来设计,使地球上每个人的基本物质需要得到满足,而且每个人有实现他个人潜力的平等机会.
	2, 乐观主义
	在过去到未来的前后两百年中,人类将会从困境中摆脱出来,自然因素不构成限制,科学技术的发展将会使社会和自然环境充满活力.
	自然资源无限,人类资源的短缺,土地,粮食,污染问题,完全可以通过技术进步来解决.
	当穷国富起来的时候,人口增长就会自动停止.
	3, 现实主义
	世界明天的好坏不是命运决定的,也不是科学技术的本性决定的,它取决于人类在今后二十年左右做的决策是否明智.
	
	
	
	2.3 可持续发展及其途径
	1987    既满足现代人的需求,又不损害后代人满足需求的能力的发展.
	1993    一部分人的发展不应损害另一部分人的利益.
	
	
	生态危机原因
	1, 人口数量快速增长
	2, 科学技术革命,极大扩大了人开发自然的能力.
	3, 思维方式:未能意识到环境问题不愿及时采取措施解决环境问题.
	
	厄里齐等式
	I = P × A × T
	I : 环境影响
	P : 人口
	A : 富裕程度
	T : 科学技术
	
	
	绿色技术 : 旨在减少污染,降低消耗,治理污染和改善生态的技术系统.
	
	内部经济价值 : 绿色技术开发者能直接获益
	->直接的外部经济价值 : 绿色技术使用者和消费这获得的价值
	->间接的外部经济价值 : 没有使用绿色技术的人获得的价值
	
	
	\section{题目}
	\heiti\large 面对生态危机,是在地球上实现可持续发展还是离开地球定居其他星系?还是带着地球去流浪?
	\subsection{答}
	1,
%\pagestyle{empty}
%\pagenumbering{roman}
%
%\tableofcontents
%\clearpage
%
%\pagestyle{fancy}
%\fancyhead{}
%\lhead{Qinghai Zhang}
%\chead{Notes on Algebraic Topology}
%\rhead{Fall 2018}
%
%\setcounter{chapter}{-1}
%\pagenumbering{arabic}
%% \setcounter{page}{0}
%
%% --------------------------------------------------------
%% uncomment the following to remove these environments 
%%  to generate handouts for students.
%% --------------------------------------------------------
%% \begingroup
%% \voidenvironment{rem}%
%% \voidenvironment{proof}%
%% \voidenvironment{solution}%
%
%
%% each chapter is factored into a separate file.
%
%\chapter{Preliminaries}
%%\begin{lstlisting}
%%int main(){
%%  double d;
%%  int i;
%%  return i;
%%}
%%\end{lstlisting}
%% The main ingredients of snacks are sugar and fat;
%%  the main ingredients of math are logic and set theory.
%We collect concepts and results
% in a coherent manner to form a solid foundation
% for our study of computational homology.
%Every math major should master the English glossary
% as well as the math in this chapter.
%
%\begin{multicols}{2}
%\setlength{\columnseprule}{0.2pt}  
%
%\section{First-order logic}
%%\label{sec:logic}
%%\input{sec/logic.tex}
%
%\section{Ordered sets}
%%\label{sec:sets}
%%\input{sec/orderedSets.tex}
%
%\section{Linear algebra}
%%\label{sec:line-algebra}
%%\input{sec/linearAlgebra.tex}
%
%\subsubsection{}
%\paragraph{契约}
%\subparagraph{input}
%
%\subparagraph{output}
%
%\subparagraph{precondition}
%
%\subparagraph{postcondition}
%
%\paragraph{算法实现}
%\begin{lstlisting}
%
%\end{lstlisting}
%\paragraph{证明}
%
%\end{multicols}


\end{document}


%%% Local Variables: 
%%% mode: latex
%%% TeX-master: t
%%% End: 

% LocalWords:  FPN underflows denormalized FPNs matlab eps IEEE iff
% LocalWords:  cardinality significand quadratically bijection unary
%  LocalWords:  contractive bijective postcondition invertible arity
%  LocalWords:  subspaces surjective injective monomials additivity
%  LocalWords:  nullary Abelian abelian finitary eigenvectors adjoint
%  LocalWords:  eigenvector nullspace Hermitian unitarily multiset
%  LocalWords:  nonsingular nonconstant homomorphism homomorphisms
%  LocalWords:  isomorphically indeterminates subfield isomorphism
%  LocalWords:  nondefective diagonalizable contrapositive cofactor
%  LocalWords:  submatrix nilpotent positivity orthonormal extremum
%  LocalWords:  Jacobian nonsquare semidefinite nonnegative RHS LLS
%  LocalWords:  roundoff closedness
