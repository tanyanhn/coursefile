\documentclass[a4paper]{book}

\usepackage{geometry}
% make full use of A4 papers
\geometry{margin=1.5cm, vmargin={0pt,1cm}}
\setlength{\topmargin}{-1cm}
\setlength{\paperheight}{29.7cm}
\setlength{\textheight}{25.1cm}

% auto adjust the marginals
\usepackage{marginfix}

\usepackage{amsfonts}
\usepackage{amsmath}
\usepackage{amssymb}
\usepackage{amsthm}
%\usepackage{CJKutf8}   % for Chinese characters
\usepackage{ctex}
\usepackage{enumerate}
\usepackage{graphicx}  % for figures
\usepackage{layout}
\usepackage{multicol}  % multiple columns to reduce number of pages
\usepackage{mathrsfs}  
\usepackage{fancyhdr}
\usepackage{subfigure}
\usepackage{tcolorbox}
\usepackage{tikz-cd}
\usepackage{listings}
\usepackage{xcolor} %代码高亮
%------------------
% common commands %
%------------------
% differentiation
\newcommand{\gen}[1]{\left\langle #1 \right\rangle}
\newcommand{\dif}{\mathrm{d}}
\newcommand{\difPx}[1]{\frac{\partial #1}{\partial x}}
\newcommand{\difPy}[1]{\frac{\partial #1}{\partial y}}
\newcommand{\Dim}{\mathrm{D}}
\newcommand{\avg}[1]{\left\langle #1 \right\rangle}
\newcommand{\sgn}{\mathrm{sgn}}
\newcommand{\Span}{\mathrm{span}}
\newcommand{\dom}{\mathrm{dom}}
\newcommand{\Arity}{\mathrm{arity}}
\newcommand{\Int}{\mathrm{Int}}
\newcommand{\Ext}{\mathrm{Ext}}
\newcommand{\Cl}{\mathrm{Cl}}
\newcommand{\Fr}{\mathrm{Fr}}
% group is generated by
\newcommand{\grb}[1]{\left\langle #1 \right\rangle}
% rank
\newcommand{\rank}{\mathrm{rank}}
\newcommand{\Iden}{\mathrm{Id}}

% this environment is for solutions of examples and exercises
\newenvironment{solution}%
{\noindent\textbf{Solution.}}%
{\qedhere}
% the following command is for disabling environments
%  so that their contents do not show up in the pdf.
\makeatletter
\newcommand{\voidenvironment}[1]{%
  \expandafter\providecommand\csname env@#1@save@env\endcsname{}%
  \expandafter\providecommand\csname env@#1@process\endcsname{}%
  \@ifundefined{#1}{}{\RenewEnviron{#1}{}}%
}
\makeatother

%---------------------------------------------
% commands specifically for complex analysis %
%---------------------------------------------
% complex conjugate
\newcommand{\ccg}[1]{\overline{#1}}
% the imaginary unit
\newcommand{\ii}{\mathbf{i}}
%\newcommand{\ii}{\boldsymbol{i}}
% the real part
\newcommand{\Rez}{\mathrm{Re}\,}
% the imaginary part
\newcommand{\Imz}{\mathrm{Im}\,}
% punctured complex plane
\newcommand{\pcp}{\mathbb{C}^{\bullet}}
% the principle branch of the logarithm
\newcommand{\Log}{\mathrm{Log}}
% the principle value of a nonzero complex number
\newcommand{\Arg}{\mathrm{Arg}}
\newcommand{\Null}{\mathrm{null}}
\newcommand{\Range}{\mathrm{range}}
\newcommand{\Ker}{\mathrm{ker}}
\newcommand{\Iso}{\mathrm{Iso}}
\newcommand{\Aut}{\mathrm{Aut}}
\newcommand{\ord}{\mathrm{ord}}
\newcommand{\Res}{\mathrm{Res}}
%\newcommand{\GL2R}{\mathrm{GL}(2,\mathbb{R})}
\newcommand{\GL}{\mathrm{GL}}
\newcommand{\SL}{\mathrm{SL}}
\newcommand{\Dist}[2]{\left|{#1}-{#2}\right|}



%----------------------------------------
% theorem and theorem-like environments %
%----------------------------------------
\numberwithin{equation}{chapter}
\theoremstyle{definition}

\newtheorem{thm}{Theorem}[chapter]
\newtheorem{axm}[thm]{Axiom}
\newtheorem{alg}[thm]{Algorithm}
\newtheorem{asm}[thm]{Assumption}
\newtheorem{defn}[thm]{Definition}
\newtheorem{prop}[thm]{Proposition}
\newtheorem{rul}[thm]{Rule}
\newtheorem{coro}[thm]{Corollary}
\newtheorem{lem}[thm]{Lemma}
\newtheorem{exm}{Example}[chapter]
\newtheorem{rem}{Remark}[chapter]
\newtheorem{exc}[exm]{Exercise}
\newtheorem{frm}[thm]{Formula}
\newtheorem{ntn}{Notation}

% for complying with the convention in the textbook
\newtheorem{rmk}[thm]{Remark}


%\lstset{
%	backgroundcolor=\color{red!50!green!50!blue!50},%代码块背景色为浅灰色
%	rulesepcolor= \color{gray}, %代码块边框颜色
%	breaklines=true,  %代码过长则换行
%	numbers=left, %行号在左侧显示
%	numberstyle= \small,%行号字体
%	keywordstyle= \color{blue},%关键字颜色
%	commentstyle=\color{gray}, %注释颜色
%	frame=shadowbox%用方框框住代码块
%}
\lstset{
	columns=fixed,       
	numbers=left,                                        % 在左侧显示行号
	numberstyle=\tiny\color{gray},                       % 设定行号格式
	frame=none,                                          % 不显示背景边框
	backgroundcolor=\color[RGB]{245,245,244},            % 设定背景颜色
	keywordstyle=\color[RGB]{40,40,255},                 % 设定关键字颜色
	numberstyle=\footnotesize\color{darkgray},           
	commentstyle=\it\color[RGB]{0,96,96},                % 设置代码注释的格式
	stringstyle=\rmfamily\slshape\color[RGB]{128,0,0},   % 设置字符串格式
	showstringspaces=false,                              % 不显示字符串中的空格
	language=c++,                                        % 设置语言
}

%----------------------
% the end of preamble %
%----------------------

\begin{document}
	
3.科学技术观
科学,science
反映客观事实和规律的知识与知识体系及其相关的研究活动.
自然学家,scientist
技术,technology
人为满足自身需求,根据实际经验或科学原理所创造或发明的各种手段,方法的总和.
工程,engineering

技术的自然性
人们在运用技术改变和利用自然的过程中,必须顺应自然规律.
任何物质手段都是天然自然的人工自然的产物.
技术的应用要以相应的自然后果为代价.

技术的社会性
技术的开发和利用具有特定的社会目的.
技术发展收社会条件的制约,又反映不同时期的人类发展状况.
技术的应用还会产生相应的社会后果.

工程
创造性地运用自然科学原理设计或研制结构,机器,设备,生产工艺,部分或整体地对它们加以利用;并运用完善的工业品艺术品设计知识对它们本身进行设计和管理;在一定可行条件下预测它们的性状.

工程师
"工程师是以一定水平的专门知识和技能为人类服务的职业名称,创新能力的成功表现和专业知识的应用是这种职业的主要回报.这就意味着这门职业的先决条件是要求具有非常良好的早期教育,并体现在从业人员以后的服务业务及伦理操行中."


科学方法论
1,科学问题与科学选题
2,建构假说
3,获取事实证据
4,建构理论
5,检验科学结论

科学问题  --research  观察,实验
--新知识  --旧知识  --假说  --科学问题


\heiti 作业


你如何看待工业4.0时代的技术与需求?
作为工程师,面对工业4,0如何处理新技术(人工智能)与满足社会需求(生产,就业)的关系问题?你做好准备了吗?


所谓工业4.0(Industry4.0), [1]  是基于工业发展的不同阶段作出的划分。 [1]  按照目前的共识, [1]  工业1.0是蒸汽机时代, [1]  工业2.0是电气化时代,工业3.0是信息化时代, [1]  工业4.0则是利用信息化技术促进产业变革的时代, [1]  也就是智能化时代。 [1] 
这个概念最早出现在德国, [1]  2013年的汉诺威工业博览会上正式推出, [1]  其核心目的是为了提高德国工业的竞争力, [1]  在新一轮工业革命中占领先机。 [1]  随后由德国政府列入《德国2020高技术战略》中所提出的十大未来项目之一。该项目由德国联邦教育局及研究部和联邦经济技术部联合资助,投资预计达2亿欧元。旨在提升制造业的智能化水平,建立具有适应性、资源效率及基因工程学的智慧工厂,在商业流程及价值流程中整合客户及商业伙伴。其技术基础是网络实体系统及物联网。
德国所谓的工业4.0是指利用物联信息系统(Cyber—Physical System简称CPS)将生产中的供应,制造,销售信息数据化、智慧化,最后达到快速,有效,个人化的产品供应。

“工业4.0”这一名称的含义是人类历史上的第四次工业革命。

工业4.0驱动新一轮工业革命,核心特征是互联 [9]  。互联网技术降低了产销之间的信息不对称,加速两者之间的相互联系和反馈,因此,催生出消费者驱动的商业模式,而工业4.0是实现这一模式关键环节。工业4.0代表了“互联网+制造业”的智能生产,孕育大量的新型商业模式,真正能够实现“C2B2C”的商业模式。 [9]

1.缺乏足够的技能来加快第四次工业革命的进程。
2.企业的信息技术部门有冗余的威胁。
3.利益相关者普遍不愿意改变。
%\pagestyle{empty}
%\pagenumbering{roman}
%
%\tableofcontents
%\clearpage
%
%\pagestyle{fancy}
%\fancyhead{}
%\lhead{Qinghai Zhang}
%\chead{Notes on Algebraic Topology}
%\rhead{Fall 2018}
%
%\setcounter{chapter}{-1}
%\pagenumbering{arabic}
%% \setcounter{page}{0}
%
%% --------------------------------------------------------
%% uncomment the following to remove these environments 
%%  to generate handouts for students.
%% --------------------------------------------------------
%% \begingroup
%% \voidenvironment{rem}%
%% \voidenvironment{proof}%
%% \voidenvironment{solution}%
%
%
%% each chapter is factored into a separate file.
%
%\chapter{Preliminaries}
%%\begin{lstlisting}
%%int main(){
%%  double d;
%%  int i;
%%  return i;
%%}
%%\end{lstlisting}
%% The main ingredients of snacks are sugar and fat;
%%  the main ingredients of math are logic and set theory.
%We collect concepts and results
% in a coherent manner to form a solid foundation
% for our study of computational homology.
%Every math major should master the English glossary
% as well as the math in this chapter.
%
%\begin{multicols}{2}
%\setlength{\columnseprule}{0.2pt}  
%
%\section{First-order logic}
%%\label{sec:logic}
%%\input{sec/logic.tex}
%
%\section{Ordered sets}
%%\label{sec:sets}
%%\input{sec/orderedSets.tex}
%
%\section{Linear algebra}
%%\label{sec:line-algebra}
%%\input{sec/linearAlgebra.tex}
%
%\subsubsection{}
%\paragraph{契约}
%\subparagraph{input}
%
%\subparagraph{output}
%
%\subparagraph{precondition}
%
%\subparagraph{postcondition}
%
%\paragraph{算法实现}
%\begin{lstlisting}
%
%\end{lstlisting}
%\paragraph{证明}
%
%\end{multicols}


\end{document}


%%% Local Variables: 
%%% mode: latex
%%% TeX-master: t
%%% End: 

% LocalWords:  FPN underflows denormalized FPNs matlab eps IEEE iff
% LocalWords:  cardinality significand quadratically bijection unary
%  LocalWords:  contractive bijective postcondition invertible arity
%  LocalWords:  subspaces surjective injective monomials additivity
%  LocalWords:  nullary Abelian abelian finitary eigenvectors adjoint
%  LocalWords:  eigenvector nullspace Hermitian unitarily multiset
%  LocalWords:  nonsingular nonconstant homomorphism homomorphisms
%  LocalWords:  isomorphically indeterminates subfield isomorphism
%  LocalWords:  nondefective diagonalizable contrapositive cofactor
%  LocalWords:  submatrix nilpotent positivity orthonormal extremum
%  LocalWords:  Jacobian nonsquare semidefinite nonnegative RHS LLS
%  LocalWords:  roundoff closedness
