\documentclass[UTF8]{ctexbeamer}	% Compile at least twice!
%\setbeamertemplate{navigation symbols}{}
\usetheme{Antibes}
\usecolortheme{beaver}
\setbeamertemplate{navigation symbols}{}
% \useinnertheme{rectangles}
% \useoutertheme{infolines}
% \useoutertheme[title,section,subsection=true]{smoothbars}
% \useoutertheme{split}
\useinnertheme{rounded}


% \usecolortheme{default}
% \usecolortheme{whale}
 
% -------------------
% Packages
% -------------------
\usepackage{
    amsmath,			% Math Environments
    amssymb,			% Extended Symbols
    enumerate,		    % Enumerate Environments
    graphicx,			% Include Images
    lastpage,			% Reference Lastpage
    multicol,			% Use Multi-columns
    multirow,			% Use Multi-rows
    pifont,			    % For Checkmarks
    stmaryrd,            % For brackets
    listings,
}
\usepackage[english]{babel}
\usepackage{graphicx}
% \usepackage{CJK}
\lstset{language=C++}
\lstset{extendedchars=false}
\lstset{breaklines}


% -------------------
% Colors
% -------------------
% \definecolor{UniOrange}{RGB}{212,69,0}
% \definecolor{UniGray}{RGB}{62,61,60}
% \definecolor{UniRed}{HTML}{B31B1B}
% \definecolor{UniGray}{HTML}{222222}
% \setbeamercolor{title}{fg=UniGray}
% \setbeamercolor{frametitle}{fg=UniOrange}
% \setbeamercolor{structure}{fg=UniOrange}
% \setbeamercolor{section in head/foot}{bg=UniGray}
% \setbeamercolor{author in head/foot}{bg=UniGray}
% \setbeamercolor{date in head/foot}{fg=UniGray}
% \setbeamercolor{structure}{fg=UniOrange}
% \setbeamercolor{local structure}{fg=black}
% \beamersetuncovermixins{\opaqueness<1>{0}}{\opaqueness<2->{15}}


% -------------------
% Fonts & Layout
% -------------------
\useinnertheme{default}
\usefonttheme{serif}
\usepackage{palatino}
\setbeamerfont{title like}{shape=\scshape}
\setbeamerfont{frametitle}{shape=\scshape}
\setbeamertemplate{itemize items}[circle]
%\setbeamertemplate{enumerate items}[default]


% -------------------
% Commands
% -------------------

% Special Characters
\newcommand{\N}{\mathbb{N}}
\newcommand{\Z}{\mathbb{Z}}
\newcommand{\Q}{\mathbb{Q}}
\newcommand{\R}{\mathbb{R}}
%\newcommand{\C}{\mathbb{C}}

% Math Operators
\DeclareMathOperator{\im}{im}
\DeclareMathOperator{\Span}{span}

% Special Commands
\newcommand{\pf}{\noindent\emph{Proof. }}
\newcommand{\ds}{\displaystyle}
\newcommand{\defeq}{\stackrel{\text{def}}{=}}
\newcommand{\ov}[1]{\overline{#1}}
\newcommand{\ma}[1]{\stackrel{#1}{\longrightarrow}}
\newcommand{\twomatrix}[4]{\begin{pmatrix} #1 & #2 \ #3 & #4 \end{pmatrix}}


% -------------------
% Tikz & PGF
% -------------------
\usepackage{tikz}
\usepackage{tikz-cd}
\usetikzlibrary{
    calc,
    decorations.pathmorphing,
    matrix,arrows,
    positioning,
    shapes.geometric
}
\usepackage{pgfplots}
\pgfplotsset{compat=newest}

\usepackage{wrapfig}
\usepackage{cite}


% -------------------
% Theorem Environments
% -------------------
\theoremstyle{plain}
\newtheorem{sit}{Situation}[section]
\newtheorem{prop}{Proposition}[section]
\newtheorem{rtm}{Theorem}[section]
\newtheorem{cor}{Corollary}[section]
\theoremstyle{definition}
\newtheorem{das}{Data structure}[section]
\newtheorem{nex}{Non-Example}[section]
\newtheorem{cla}{class}[section]
\newtheorem{emt}{}[section]
\newtheorem{defn}{Definition}[section]
\theoremstyle{remark}
\newtheorem{rem}{Remark}[section] 
\numberwithin{equation}{section}

\newcommand\caesura{$\mkern -8.5mu\raise -.2ex\hbox{\rotatebox[]{180}{\`}}\ $}

% -------------------
% Title Page
% -------------------
\title{\textcolor{red}{2022年硕博连读综合面试报告}}
%\subtitle{\textcolor{white}{Mathematics Conference for the Mysterious and dMagical}}  
\author{谭焱\, \newline   \newline \small{专业: 计算数学}\, 
\newline \newline
 \small{硕士导师: 王何宇\, \\
 申请博士导师: 张庆海}}

\institute{浙江大学数学科学学院}
\date{\today} 


% -------------------
% Content
% -------------------
\begin{document}
% \begin{CJK}{GBK}{kai}

% Title Page
\begin{frame}
\titlepage
\end{frame}


\begin{frame}
    \frametitle{目录}
    \tableofcontents
  \end{frame}

% Motivation
\section{个人基本情况介绍}


% \begin{frame}
%     \begin{emt}[过往受教育经历]
%         \begin{enumerate}
%             \item 高中在湖南师范大学附属中学就读时参与数学奥林匹克竞赛.
%         \end{enumerate}
        
%     \end{emt}
% \end{frame}


% Definitions & Examples
\begin{frame}[fragile]
    \frametitle{学习情况}
\begin{enumerate}
    \item 研究生课程
    \begin{itemize}
        \item 已修满硕士学位要求的学分,在与科研项目相关的课程中取得良好成绩.
        \begin{itemize}
            \item 非线性问题的数学方法(92), 图形学的新进展(90) 等.
        \end{itemize}
        \item 英语阅读及写作方面
        \begin{itemize}
            \item 六级489分(阅读205)可以流畅阅读英文文献.
            \item 通过课程研究生论文写作指导(92)打下坚实写作基础.
        \end{itemize}
    \end{itemize}
    \item 编程学习
    \begin{itemize}
        \item 熟悉Cpp14之前的大部分特性,使用Cpp完成张老师多个项目.
        \item 能够阅读Cpp, Fortran, Java, Python, Shell等语言的代码.
        \item 独立AC LeetCode中200+道Hard题,能运用常见数据结构和高效算法.
    \end{itemize}
\end{enumerate}
\end{frame}

\begin{frame}[fragile]
   \frametitle{研究项目参与情况}
    \small{\begin{enumerate}
        \item 2019年春学期,独立完成程序.实现张老师的论文中的二维空间内殷集上的布尔代数.
        为之后的三维空间内殷集之间的布尔代数的研究做好铺垫.
        \includegraphics[width = \linewidth]{fig/articlename1.png}
        \item 2020年秋学期,在张老师的指导下,和学长合作完成项目微地震反问题.最终得到一个能根据
        检波器接收到的地震波信号输出合理的震源位置的程序.
        \item 2021年春学期,与学弟分工推进三维殷集的表示,及通过编程实现在计算机上计算
        三维殷集之间的布尔运算.
        \item 2021年秋学期,尝试使用matlab在张老师提出的CubicMars方法追踪动边界
        的过程中加入高精度追踪拓扑变化的时间点和发生位置.
    \end{enumerate}}
\end{frame}

\section{研究生项目详细介绍}
\subsection{微地震的反问题}

\begin{frame}[fragile]
    \frametitle{微地震探测反问题研究}
    \begin{itemize}
        \item  微地震通常是由于地质勘探或者一些开采活动导致地下裂缝错位,从而形成的
        低频率弹性波.
        \item \begin{columns}
            \column{0.5\linewidth}<1->
                \centering
                \includegraphics[width = \textwidth]{fig/s2p1.jpg}
            \column{0.5\linewidth}<1->
        \end{columns}
        \item 因为地质结构的复杂,地震波非直线传播,并且微地震常发生于地下深处,能获取的信息少
        想高精度反演非常困难. 
    \end{itemize}
\end{frame}

\begin{frame}
    \begin{emt}[背景及意义]
      
    \begin{itemize}
        \begin{columns}
            \column{0.8\linewidth}<1->
    % \begin{wrapfigure}[r]{5cm}
    %    \centering
    %    \includegraphics[width = \textwidth]{fig/s2p1.jpg}
    %    \caption{\footnote{ 油气田中的生产井和监测井}}
    % \end{wrapfigure}
    \column{0.2\linewidth}<1->
    \item  微地震通常是由于地质勘探或者一些开采活动导致地下裂缝错位,从而形成的
低频率弹性波.\cite{wu_1991}
\end{columns}
\item 作为一种人为产生的地震,其产生的信号可以用于石油工程作业.
近些年来,国际上众多的研究机构与微震公司已经证明了微地震监测方法在油气田
规划与开发方面的指导意义.

    \end{itemize}

    
\end{emt}
\end{frame}


% \end{CJK}
\end{document}