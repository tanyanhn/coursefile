\documentclass[UTF8]{ctexbeamer}	% Compile at least twice!
%\setbeamertemplate{navigation symbols}{}
\usetheme{AnnArbor}
\usecolortheme{beaver}
\setbeamertemplate{navigation symbols}{}
% \useinnertheme{rectangles}
% \useoutertheme{infolines}
% \useoutertheme[title,section,subsection=true]{smoothbars}
% \useoutertheme{split}
\useinnertheme{rounded}


% \usecolortheme{default}
% \usecolortheme{whale}
 
% -------------------
% Packages
% -------------------
\usepackage{
    amsmath,			% Math Environments
    amssymb,			% Extended Symbols
    enumerate,		    % Enumerate Environments
    graphicx,			% Include Images
    lastpage,			% Reference Lastpage
    multicol,			% Use Multi-columns
    multirow,			% Use Multi-rows
    pifont,			    % For Checkmarks
    stmaryrd,            % For brackets
    listings,
}
\usepackage[english]{babel}
\usepackage{graphicx}
\usepackage{xeCJK}
\usepackage{fontspec} 
\setCJKmainfont{gkai00mp.ttf}
% \usepackage{CJK}
\lstset{language=C++}
\lstset{extendedchars=false}
\lstset{breaklines}


% -------------------
% Colors
% -------------------
% \definecolor{UniOrange}{RGB}{212,69,0}
% \definecolor{UniGray}{RGB}{62,61,60}
% \definecolor{UniRed}{HTML}{B31B1B}
% \definecolor{UniGray}{HTML}{222222}
% \setbeamercolor{title}{fg=UniGray}
% \setbeamercolor{frametitle}{fg=UniOrange}
% \setbeamercolor{structure}{fg=UniOrange}
% \setbeamercolor{section in head/foot}{bg=UniGray}
% \setbeamercolor{author in head/foot}{bg=UniGray}
% \setbeamercolor{date in head/foot}{fg=UniGray}
% \setbeamercolor{structure}{fg=UniOrange}
% \setbeamercolor{local structure}{fg=black}
% \beamersetuncovermixins{\opaqueness<1>{0}}{\opaqueness<2->{15}}


% -------------------
% Fonts & Layout
% -------------------
\useinnertheme{default}
\usefonttheme{serif}
\usepackage{palatino}
\setbeamerfont{title like}{shape=\scshape}
\setbeamerfont{frametitle}{shape=\scshape}
\setbeamertemplate{itemize items}[circle]
%\setbeamertemplate{enumerate items}[default]


% -------------------
% Commands
% -------------------

% Special Characters
\newcommand{\N}{\mathbb{N}}
\newcommand{\Z}{\mathbb{Z}}
\newcommand{\Q}{\mathbb{Q}}
\newcommand{\R}{\mathbb{R}}
%\newcommand{\C}{\mathbb{C}}

% Math Operators
\DeclareMathOperator{\im}{im}
\DeclareMathOperator{\Span}{span}

% Special Commands
\newcommand{\pf}{\noindent\emph{Proof. }}
\newcommand{\ds}{\displaystyle}
\newcommand{\defeq}{\stackrel{\text{def}}{=}}
\newcommand{\ov}[1]{\overline{#1}}
\newcommand{\ma}[1]{\stackrel{#1}{\longrightarrow}}
\newcommand{\twomatrix}[4]{\begin{pmatrix} #1 & #2 \ #3 & #4 \end{pmatrix}}


% -------------------
% Tikz & PGF
% -------------------
\usepackage{tikz}
\usepackage{tikz-cd}
\usetikzlibrary{
    calc,
    decorations.pathmorphing,
    matrix,arrows,
    positioning,
    shapes.geometric
}
\usepackage{pgfplots}
\pgfplotsset{compat=newest}

\usepackage{wrapfig}
\usepackage{cite}


% -------------------
% Theorem Environments
% -------------------
\theoremstyle{plain}
\newtheorem{sit}{Situation}[section]
\newtheorem{prop}{Proposition}[section]
\newtheorem{rtm}{Theorem}[section]
\newtheorem{cor}{Corollary}[section]
\theoremstyle{definition}
\newtheorem{das}{Data structure}[section]
\newtheorem{nex}{Non-Example}[section]
\newtheorem{cla}{class}[section]
\newtheorem{emt}{}[section]
\newtheorem{defn}{Definition}[section]
\theoremstyle{remark}
\newtheorem{rem}{Remark}[section] 
\numberwithin{equation}{section}

\newcommand\caesura{$\mkern -8.5mu\raise -.2ex\hbox{\rotatebox[]{180}{\`}}\ $}

% -------------------
% Title Page
% -------------------
\title{\textcolor{red}{2022年硕博连读综合面试报告}}
%\subtitle{\textcolor{white}{Mathematics Conference for the Mysterious and dMagical}}  
\author{谭焱(张庆海)\, \newline   \newline 专业: 计算数学\, 
% \newline \newline
%  \small{硕士导师: 王何宇\, \\
%  申请博士导师: 张庆海}
 }

% \institute{浙江大学数学科学学院}
\date{\today} 


% -------------------
% Content
% -------------------
\begin{document}
% \begin{CJK}{GBK}{kai}

% Title Page
\begin{frame}
\titlepage
\end{frame}


\begin{frame}
    \frametitle{目录}
    \tableofcontents
  \end{frame}

% Motivation
\section{个人基本情况介绍}


% \begin{frame}
%     \begin{emt}[过往受教育经历]
%         \begin{enumerate}
%             \item 高中在湖南师范大学附属中学就读时参与数学奥林匹克竞赛.
%         \end{enumerate}
        
%     \end{emt}
% \end{frame}


% Definitions & Examples
\begin{frame}[fragile]
    \frametitle{学习情况}
\begin{enumerate}
    \item 我是在王何宇老师名下的研究生,因为王老师和张庆海教授的合作关系,硕士生
    阶段跟张教授做项目.
    \item 研究生课程
    \begin{itemize}
        \item 已修满硕士学位要求的学分,在与科研项目相关的课程中取得良好成绩.
        \begin{itemize}
            \item 非线性问题的数学方法(92), 图形学的新进展(90) 等.
        \end{itemize}
        \item 英语阅读及写作方面
        \begin{itemize}
            \item 六级489分(阅读205)可以流畅阅读英文文献.
            \item 通过研究生论文写作指导(92)打下坚实写作基础.
        \end{itemize}
    \end{itemize}
    % \item 编程学习
    % \begin{itemize}
    %     \item 熟悉Cpp14之前的大部分特性,使用Cpp完成张庆海老师多个项目.
    %     \item 流畅阅读Fortran, Python, Shell等语言的项目和使用make等工具.
    %     \item 独立AC LeetCode中200+道Hard题,能运用常见数据结构和高效算法.
    % \end{itemize}
\end{enumerate}
\end{frame}

\begin{frame}[fragile]
   \frametitle{研究项目参与情况}
    \small{\begin{enumerate}
        \item 实现张庆海教授的论文(MATH COMP, 2020)中的二维空间上殷集的布尔代数.
        % 为之后的三维空间内殷集之间的布尔代数的研究做好铺垫.
        % \includegraphics[width = \linewidth]{fig/articlename1.png}
        \item 和黎颖学长合作完成项目微地震反问题.最终得到一个能根据
        检波器接收到的地震波信号输出合理的震源位置的程序.
        \item 与邱云昊学弟分工推进三维殷集的表示,及编程实现在计算机上高效计算
        三维殷集的布尔代数.
        \item 独立尝试使用matlab在张庆海教授提出的CubicMARS方法追踪动边界
        的过程中加入高精度追踪拓扑变化的时间点和发生位置.
    \end{enumerate}}
\end{frame}

\section{研究生阶段的科研工作}
% \subsection{微地震的反问题}

% \begin{frame}[fragile]
%     \frametitle{研究背景}
%     \begin{itemize}
%         \item  微地震通常是由于地质勘探或者一些开采活动导致地下裂缝错位,从而形成的
%         低频率弹性波.
%         \item 微地震经常出现在石油天然气工
%         程作业. 近些年来, 国际上众多的研究机构与微震公司已经证明了
%         微地震监测方法在油气田规划与开发方面的指导意义.
%         \begin{columns}
%             \column{0.4\linewidth}<1->
%                 \includegraphics[width = \textwidth]{fig/s2p1.jpg}
%             \column{0.6\linewidth}<1->
%             \includegraphics[width = \textwidth]{fig/layer.png}
%         \end{columns}
%         \item 地质结构的复杂,地震波非直线传播,微地震常发生于地下深处,检波器获取的信息少
%         ,导致高精度反演非常困难. 
%     \end{itemize}
% \end{frame}

% \begin{frame}
%     \frametitle{解决思路}
%     \begin{itemize}
%         \item 检波器获取的到时是地震波最早到时,计算假设震源到检波器
%         的最短时间,得到超定方程组.
%         \begin{equation}
%             V^lS^k - V^kS^l - V^lV^k(t^l-t^k) =0, \quad l \neq k.
%         \end{equation}
%         \item 采用多轮循环迭代的方式计算震源到检波器的时间.如下图处理二维情况
%         \begin{columns}
%             \column{0.5\linewidth}<1->
%                 \includegraphics[width = \textwidth]{fig/layer1.png}
%             \column{0.5\linewidth}<1->
%             \includegraphics[width = \textwidth]{fig/layer2.png}
%         \end{columns}
%     \end{itemize}
% \end{frame}

% \begin{frame}
%     \frametitle{反思与学习}
%     \begin{itemize}
%         \item 计算时间的迭代可能因为地层复杂结构和断层的存在出现局部解.(加入模拟退火
%         等随机因素)
%         \item 震源位置的迭代会因为检波器过于集中在监测井这条直线上导致精确得到xy坐标
%         很困难.(输出多个疑似震源)
%     \end{itemize}
%     当运用数学到工程实际中时,有时无法得到充分
%     的已知条件从而不能给出准确的标准答案,及时的沟通修改契约.
%     \begin{center}
%     \includegraphics[width = 0.5\textwidth]{fig/sol122.png}
%     \end{center}
% \end{frame}

\subsection{\textcolor{red}{三维殷集和布尔代数(军工项目)}}
\begin{frame}
    \frametitle{研究背景}
    \begin{itemize}
        \item 流体建模相关研究少,数学模型和计算机算法都设计成避免在数值模拟时对流体
        进行几何建模.
        % \begin{enumerate}
        %     \item VOF方法中使用各个单元的体积分数重建边界.
        %     \item FT方法追踪边界上的示踪点,按顺序连接得到边界.
        %     \item LS方法求解隐式函数的边界.
        % \end{enumerate}
        主流的VOF方法, FT方法和LS方法舍弃了界面上的拓扑信息,几何问题转化为求解微分方程.过去几十年这些方法取得
        了大量的成果.
        \item 我们想在复杂拓扑区域上进行数值模拟,这些方法避免了几何建模使得很难严格地处理拓扑变化.
        \item 张庆海教授2020年的论文中为流体建模建立了一个理论基础,在二维空间中
        提出了数学模型殷集给任意有物理意义的区域建模.
    \end{itemize}
\end{frame}

\begin{frame}
    \frametitle{二维殷集}
    \begin{itemize}
        \item 二维空间中,任一个殷集可以唯一表示为
        \[\mathcal{Y} = \cup_j^{\bot \bot}\cap_i \text{int}(\gamma_{j, i} ),\]
        Jordan Curve $\gamma_{j, i}$是$\mathcal{Y}$内第$j$个连通分量
        的第$i$条边界.
        \item 高效实现了殷集上的布尔代数. 
        \begin{columns}
            \column{0.3\linewidth}<1->
                \includegraphics[width = \textwidth]{fig/p.png}
            \column{0.3\linewidth}<1->
            \includegraphics[width = \textwidth]{fig/m.png}
            \column{0.3\linewidth}<1->
            \includegraphics[width = \textwidth]{fig/pm.png}
        \end{columns}
    \end{itemize}
\end{frame}

\begin{frame}
    \frametitle{三维殷集的表示}
    \begin{itemize}
        \item 三维殷集:三维空间中边界有界的正则半解析开集.所有三维殷集构
        成的集合被称为殷空间,记为 $\mathbb{Y}$。
        \item 任一个殷集$\mathcal{Y} \in \mathbb{Y}$可以唯一表示为
        \[\mathcal{Y} = \cup_j^{\bot \bot} \cap_i \text{int}(\Gamma_{j, i}),\]
        黏合紧曲面$\Gamma_{j, i}$是$\mathcal{Y}$的第$j$个连通分量的第$i$个边界.
        \item 黏合紧曲面是互相之间没有恰当交的闭合有向曲面.
    \end{itemize}
    \begin{columns}
        \column{0.5\linewidth}<1->
            \includegraphics[width = \textwidth]{fig/s1.png}
        \column{0.5\linewidth}<1->
        \includegraphics[width = \textwidth]{fig/s2.png}
    \end{columns}
\end{frame}

\begin{frame}
    \frametitle{布尔代数}
    \begin{itemize}
        \item 求布尔代数的两个殷集
        \begin{columns}
            \column{0.4\linewidth}<1->
                \includegraphics[width = \textwidth]{fig/s3.png}
            \column{0.4\linewidth}<1->
            \includegraphics[width = \textwidth]{fig/s4.png}
        \end{columns}
        \item 交
        \begin{columns}
            \column{0.4\linewidth}<1->
                \includegraphics[width = \textwidth]{fig/s5.png}
            \column{0.4\linewidth}<1->
            \includegraphics[width = \textwidth]{fig/s6.png}
        \end{columns}
    \end{itemize}
\end{frame}

\begin{frame}
    \begin{itemize}
        \item 并 \begin{columns}
            \column{0.5\linewidth}<1->
                \includegraphics[width = \textwidth]{fig/s7.png}
            \column{0.5\linewidth}<1->
            \includegraphics[width = \textwidth]{fig/s8.png}
        \end{columns}
        \item 复杂几何结构
        \begin{columns}
            \column{0.3\linewidth}<1->
                \includegraphics[width = \textwidth]{fig/s9.png}
            \column{0.3\linewidth}<1->
            \includegraphics[width = \textwidth]{fig/s10.png}
            \column{0.3\linewidth}<1->
            \includegraphics[width = \textwidth]{fig/s11.png}
        \end{columns}
    \end{itemize}
\end{frame}

\begin{frame}
    \frametitle{三维殷集建模的意义}
    \begin{itemize}
        \item 殷空间是一个跨领域的为有意义的物理区域恰当地建模的拓扑空间。
        \item 我们提供了殷集的简单表示方法,并且可以
        从殷集的表示方法中常数时间复杂度提取拓扑信息.
        \item 在殷空间上实现了高效的布尔代数,布尔代数是研究流相拓扑的核心工具之一.
        \item 为处理移动区域的拓扑变化提供了理论支持和程序接口.
        \item 与CubicMARS方法结合,可以提高界面追踪的精度,进而提高
        数值计算方法的精度.
        \end{itemize}
\end{frame}

\begin{frame}
    \frametitle{论文在投}
\end{frame}

% \subsection{CubicMARS方法追踪拓扑变化}
% \begin{frame}
%     \frametitle{研究背景}
%     \begin{itemize}
%         \item 多相流的研究在军事国防,医学仿生,核能工业,海洋工程,国民经济等许多重大
%         领域中都占有举足轻重的地位; 而界面追踪问题是多相流数值计算中最基本的子问题之一.
%         % 其重要性体现在
%         % \begin{enumerate}
%         %     \item 界面追踪不可避免的影响到流相计算精度.
%         %     \item 在表面张力不可忽略的多相流问题中,界面追踪准确性低会导致数值模拟
%         %     结果脱离物理实际.
%         % \end{enumerate}
%         \item 现有的界面追踪方法在过去都取得了巨大成功,但是随着多相流研究
%         的进展,这些方法逐渐捉襟见肘.
%         \begin{enumerate}
%             \item 现有的方法计算精度不够高.
%             \item 在处理流相拓扑结构变化时随意性大.
%             \item 现有的显式界面追踪方法缺乏严格的系统理论支持.
%         \end{enumerate}
%         \item 张庆海教授18年提出的CubicMARS方法是一套用于对界面追踪问题的分析
%         的普适理论,
%         是时空一致四阶以上精度的界面追踪方法.
%     \end{itemize}
    
% \end{frame}

% \begin{frame}
%     \frametitle{研究的方向} 
%     \begin{itemize}
%         \item 在CubicMARS方法的基础上,结合二维殷集,增添对拓扑变化的处理.
%         \item 需要在有拓扑变化的流相追踪过程中保持计算精度.
%         \item 期望同样能高精度捕捉流相拓扑变化的时间点和位置.
%     \end{itemize}
%     \centering
%     \includegraphics[width = 0.7\textwidth]{fig/s13.png}
% \end{frame}

\section{博士阶段的研究计划}
\begin{frame}
    \frametitle{计划的博士研究项目}
    \begin{itemize}
        \item 潜艇湍流尾迹的海洋表面特征等内波现象的研究, 用于潜艇追踪和隐身.
        (\textcolor{red}{军科委基础加强重点项目}).
        \begin{figure}
            \includegraphics[width = 0.7\textwidth]{fig/s12.png};
            \caption{水下潜艇产生的各种尾迹示意图, 包括开尔文尾迹, 内波, 湍流尾迹, 涡尾迹, 煎
            饼旋涡.}
        \end{figure}
    \end{itemize}
\end{frame}

\section*{}
\begin{frame}
    \centering\huge
    \textcolor{red}{请各位老师批评指正!}
\end{frame}
% \end{CJK}
\end{document}